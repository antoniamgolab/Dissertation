\chapter*{Abstract}
\hl{\textbf{TODO:} think about the title? maybe origin-destination-flow based? }
The objective of the thesis is to advance the analytical understanding of the demand-driven geographic allocation of charging infrastructure of future electrified vehicle fleets in different road transport segments. The core contribution to the scientific literature is the extension of charging infrastructure capacity planning to consider diverse drivers of charging demand. This includes the long-term battery-electric vehicle (BEV) adoption, consumer-specific utilization patterns and inter-regional traffic flows. This is addressed through four research papers. Three of the four methodologies applied are optimizations, one a simulation-based approach. The central feature of the analytical methods is the use of traffic flow data, leveraging geographic relations of traffic flow to model the spatial distribution of charging demand. 

For passenger cars, we first address the planning of fast charging for long-distance passenger car travel at the national level under different decarbonization scenarios for the transport sector. We observe that the highest reduction in required charging infrastructure capacities is achieved in scenarios with high charging efficiencies, while the driving range of BEVs has no significant impact due to the dense network topology of the analysed highway network. Furthermore, we analyze public charging infrastructure for shorter-distance traffic and BEV adoption within and between three regions. Here, the analysis shows income-class differences in charger utilization preferences which are further shaped by the local spatial density of charging stations. The local expansion of rapid public charging infrastructure has a moderate impact on BEV adoption in neighboring regions. 

For commercial fleets, the analysis focuses on the spatial heterogeneity of charging loads and flexibility potential. The sensitivity of charging load allocation to spatially varying electricity prices and network fees is analyzed along international road transport routes. Furthermore, this study identifies segments in an international transport corridor where rail infrastructure is cost-effective in contributing to the decarbonization of road freight and reducing the local concentration of charging loads. The temporal flexibility of commercial fleets is analyzed, considering light-duty vehicles, buses, and heavy-duty vehicles at the federal and state levels. The spatial heterogeneity results from geographical variations in regional fleet sizes and in national and international truck traffic flows. Moreover, high temporal flexibility in charging demand allocation is evident on weekends, which, in cases of cost-optimal charging, may result in substantial peak loads in the electricity system.

% A main learning of this thesis is that, to support the large-scale adoption of BEVs with cost-efficient charging, the cross-regional considerations of adoption patterns and mobility requirements can support long-distance transport of both passenger cars and trucks.
A central insight of this thesis is that planning of charging infrastructure—accounting for both adoption patterns and cross-regional vehicle movements — is necessary to support the electrification of particularly long-distance BEV application for passenger cars and trucks alike.
% the geographic allocation of charger capacities needs to be planned with consideration for the impact on charging patterns across larger geographic scopes and on the long-term BEV uptake of different consumer groups, while maintaining sufficient geographic resolution in the capacity allocation. 


% The applied methods vary primarily by geographic scope (national, transnational, sub-national) and by the temporal scale of analysis (static, long-term annual, hourly), tailored to the geographic extent of the mobility patterns of different road segments. Two research papers address charging infrastructure planning for long-distance passenger cars, while the other two are dedicated to the charging of commercial fleets. 
% of local charging loads to 



%In this study, we study the requirements for fast-charging capacities along the highway network under different decarbonization scenarios. 
%Later, the focus shifts to regional passenger car transport. Here,  For commercial vehicles, we explore two dimensions of charging flexibility. First, the spatial flexibility in the allocation of charging demand between different electricity market zones for long-haul trucks is analysed. At last, the temporal flexibility potential is quantified for buses, light-duty vehicles and heavy-duty vehicle of short- and long-haul transport is quantified. 

%The main learnings /outcomes of the modeling results point towards the need for the cross-sectoral perspective between road transport networks and energy systems with the consideration of spatio-temporal dynamics in the long-term, i.e. the adoption, but also in the short-term, i.e., the hourly utilization of charging infrastructure.