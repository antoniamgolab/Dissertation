\section{Research questions}\label{sec:researchquestion}
As described in Section \ref{sec:motivation}, several flexibility options could be integrated into the electricity market. For this purpose the following flexibility types are integrated in the European electricity market:
\begin{itemize}
 \setlength\itemsep{-0.5em}
    \item Producer flexibility: Storage expansion
    \item Infrastructure flexibility: Transmission line expansion
    \item Consumer flexibility: Demand-side flexibility provided by \ac{EV}s
    \item Consumer- and producer flexibility: Sector coupling to hydrogen markets
\end{itemize}
This thesis models and quantifies the flexibility options mentioned above in the European electricity market. 


The contributions are based on the peer-reviewed articles: \textit{Synergies and competition: Examining flexibility options in the European electricity market} \cite{LoschanFlexibility}, \textit{Hydrogen as Short-Term Flexibility and Seasonal Storage in a Sector-Coupled Electricity Market} \cite{LoschanH2} and \textit{Flexibility potential of aggregated electric vehicle fleets to reduce transmission congestions and redispatch needs: A case study in Austria} \cite{LoschanEV}.


Particular emphasis is given on the integration of hydrogen markets and individual transport by \ac{EV}s. The integration of these flexibility options into the market clearing allows to find an optimum without the need for several model iterations. The economic effects of the above mentioned flexibility options and competition and synergy between them are evaluated. Further, a developed redispatch model quantifies congestion and the need for redispatch measures after the market clearing. With this two-stage model the influence of flexibility not only on the electricity market but also on the physical transmission grid is shown. Each research question is answered in a peer-reviewed article published as the main author. Hence, the following research questions will be answered: 


\textbf{Research question 1:} \textit{Are there competitive or synergistic effects between different types of flexibility when different technologies provide them?}

This research question is answered in the publication \textit{Synergies and competition: Examining flexibility options in the European electricity market} \cite{LoschanFlexibility}. This paper includes the flexibility options: demand-side flexibility, storage, transmission grid and hydrogen sector coupling. Competitive and synergistic welfare effects between them are qualitative and quantitative analysed. The effects on prices and surpluses are analysed between the bidding zones Austria and Germany. Their different generation fleets with, on the one hand, a very high storage capacity in the alpine arc in Austria and, on the other hand, massive wind turbine capacities in Germany serve as a representative case study.


\textbf{Research question 2:} \textit{How do electricity and hydrogen prices affect \ac{SEW} shifts? On the one hand, between consumers and producers, and on the other hand, between the electricity and hydrogen markets. This includes hydrogen production, seasonal storage, and reconversion to electricity.}

This research question is answered in the publication \textit{Hydrogen as Short-Term Flexibility and Seasonal Storage in a Sector-Coupled Electricity Market} \cite{LoschanH2}. This paper analyses the effects of sector coupling between the European electricity market and national hydrogen markets in detail. For this purpose, economic effects on consumers and producers, such as fossil fired thermal power plants, renewable energies, nuclear power and storages are analysed.


\textbf{Research question 3:} \textit{What are the influences of customer integration as aggregated \ac{EV} fleet demand-side flexibility on congestion and needed redispatch measures? On the one hand, as flexible demand using smart-charging to minimise electricity costs, and on the other hand to balance redispatch needs.}

This research question is answered in the publication \textit{Flexibility potential of aggregated electric vehicle fleets to reduce transmission congestions and redispatch needs: A case study in Austria} \cite{LoschanEV}. This paper analyses the effects of demand-side flexibility provided by electric vehicles on transmission line congestion and redispatch needs in Austria. The electrified individual transport sector and the associated flexibility potential is integrated in the European electricity market. After the market clearing the redispatch model calculates the cost otimal redispatch to overcome transmission line congestion.  


Figure \ref{fig:research_Questions} illustrates the three research questions and their scope within the European electricity market model. The top of the figure depicts research question 1, dealing with several flexibility options and their interaction. The bottom of the figure describes the related research questions 2 and 3, focusing on hydrogen sector coupling and demand-side flexibility. 


To answer these research questions, a two-stage fundamental European electricity market model is developed. This includes a dispatch model that calculates the market clearing while maximising \ac{SEW}. This makes it possible to evaluate competitive and synergistic effects between different flexibility types evident in the European electricity market. Moreover, interactions between the electricity and hydrogen markets are evaluated. This includes hydrogen production, the use of hydrogen as seasonal storage and reconversion to electricity. Following the dispatch, the cost-optimal redispatch is calculated. Using this, the effects of flexibility on transmission grid congestion and redispatch needs are evaluated. For this purpose, on the one hand, aggregated \ac{EV} fleets and \ac{RESE} curtailment are used to balance redispatch needs. On the other hand, the use of demand-side flexibility during dispatch and the consequences for redispatch measures are analysed.
\begin{landscape}
\begin{figure}[h]
\centering
\includegraphics[angle=90,width=1.25\textwidth]{graphics/RQ_overview_detail.pdf}
\caption{Graphic illustration of the three research questions}
\label{fig:research_Questions}
\end{figure}
\end{landscape}
\FloatBarrier