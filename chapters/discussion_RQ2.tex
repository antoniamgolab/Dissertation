\subsection{Hydrogen sector coupling}\label{sec:discussion_RQ2}
\textbf{Research question 2:} \textit{How do electricity and hydrogen prices affect \ac{SEW} shifts? On the one hand, between consumers and producers, and on the other hand, between the electricity and hydrogen markets. This includes hydrogen production, seasonal storage, and reconversion to electricity.}


The results show that a change in electricity prices caused by sector coupling directly causes \ac{PS} and \ac{CS} changes. Low electricity prices stimulate hydrogen production, which raises electricity prices. High electricity prices trigger the dispatch of fuel cells, resulting in a price reduction. As a result, the extent of extremely low and extremely high electricity prices becomes lower. The average and standard deviation of electricity prices are reduced, while the median electricity price remains unchanged. As a result, observing the median price in isolation is not a reliable method for analysing the effects of sector coupling. It is critical to consider the respective surpluses. 


Hydrogen production drastically increases nuclear power plant \ac{PS}. They stand to gain significantly from hydrogen production if the public accepts pink hydrogen as pink hydrogen sets up a high share of overall production. However, the results indicate that hydrogen from nuclear power plants could play an important role depending on political and regulatory decisions. Generation units with higher \ac{SRMC}, such as coal, lignite, oil or gas, can only achieve additional \ac{PS} if the hydrogen price is sufficiently high. Using fuel cells reduces their \ac{PS} because they compensate for price peaks.


Because of the \ac{RESE} marginally low \ac{SRMC}, an increase in electricity prices always increases their \ac{PS}. Even in the fuel cell scenarios, price peaks and spillage reduction lead to an increase in \ac{PS} over the year. As a result, the increase in \ac{PS} through higher clearing prices is greater than the decrease in \ac{PS} due to lower clearing prices. In particular, hydrogen integration benefits wind and \ac{PV}. 


According to the sector coupling definition, the electrolyser \ac{PS} increases in scenarios without hydrogen-to-electricity conversion. When fuel cells are used in conjunction with a hydrogen price of 0 \euro{}/MWh\textsubscript{H\textsubscript{2}}, hydrogen production is only triggered by the use of fuel cells. This results in a negative electrolyser \ac{PS}. Higher hydrogen prices result in higher electrolyser \ac{PS} but lower fuel cell \ac{PS}. Only if the fuel cell \ac{SRMC} are lower than those of other generation units they are used instead of them. This demonstrates the fuel cell \ac{PS}'s strong reliance on primary energy prices such as natural gas and hydrogen. 


In all scenarios in 2030, the short-term flexibility of battery storage competes with the flexibility provided by sector coupling. Hence, the battery storage \ac{PS} reduces. \ac{PHS} typically have a higher storage capacity, a natural inflow and can be used for longer periods. As a result of higher electricity prices, their \ac{PS} rises. Sector coupling reduces the \ac{PS} in all scenarios in 2040 for both storage types due to the increased availability of flexibility through sector coupling.


Sector coupling reduces the \ac{CS} in all scenarios by 2030, as electricity prices rise due to electrolysers. In 2040, \ac{CS} increases through significant price peaks. Otherwise, from the standpoint of the electricity market, consumers do not benefit from sector coupling. This does not necessarily imply that there is no overall \ac{CS} increase if other sectors, such as heating and transportation, are also considered. However, a reduction in \ac{CS} generally disadvantages consumers who do not require hydrogen. This entails a significant social redistribution of costs and benefits. 


In all scenarios, the overall \ac{SEW} rises. Thermal power plants and \ac{RESE} benefit equally from sector coupling in 2030. There is a surplus shift from consumers to electricity producers. In 2040, there is a significant difference between scenarios with hydrogen production and complete sector coupling, including storage units and reconversion. In the scenarios with hydrogen production, there is a welfare shift from consumers to electricity producers. Fuel cell utilisation reverses this welfare shift. If significant electricity peaks can be compensated by fuel cells, the consumers will benefit the most from sector coupling. This, however, is not limited to fuel cells. In addition to other flexibility options, \ac{CCGT} can also use hydrogen.