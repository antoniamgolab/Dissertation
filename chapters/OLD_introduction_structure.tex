\section{Structure of the thesis}\label{sec:structure}
The remainder of this thesis is organised as follows:


In Chapter \ref{SOTA}, the review of relevant literature regarding this thesis's scope and three research questions is examined. This includes possible flexibility options that could be integrated into the European electricity market. Both their potential and possible applications are examined. Furthermore, the organisation of these flexibility options in flexibility markets is described. Subsequently, their beneficial effects and synergy and competition between them are outlined. This relies on the main conclusions of existing literature and modelling approaches. In addition, congestion in the European electricity market and methods to overcome them are analysed. Finally, the contribution of this thesis beyond state of the art is described.


In Chapter \ref{methods}, the model's methodology and mathematical formulation are described. Furthermore, the use cases and scenarios examined to answer the research questions are outlined. The methodology to implement the two-stage electricity market model is examined comprehensively. This includes the integration of \ac{EV}s, sector coupling to the hydrogen market, \ac{SEW} calculation and the redispatch model. Following this, the mathematical formulation of the model is done. The model is separated into three parts, the dispatch model and the subsequent redispatch model and the ex-post calculation of \ac{SEW} and its components. Finally, an overview of the use cases and scenarios to answer the research questions is given.


In Chapter \ref{results}, the results to answer the three research questions are presented. The presented results are based on \textit{Synergies and competition: Examining flexibility options in the European electricity market} \cite{LoschanFlexibility}, \textit{Hydrogen as Short-Term Flexibility and Seasonal Storage in a Sector-Coupled Electricity Market} \cite{LoschanH2} and \textit{Flexibility potential of aggregated electric vehicle fleets to reduce transmission congestions and redispatch needs: A case study in Austria} \cite{LoschanEV}. Their discussion and synthesis are done in Chapter \ref{synthesis}. In addition, the limitations and strengths of the methodology are examined.


In Chapter \ref{conclusions}, the thesis is concluded by giving emphasis to ideas for future work and upcoming research questions. 
