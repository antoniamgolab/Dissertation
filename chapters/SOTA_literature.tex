\chapter{State of the art and progress beyond}\label{SOTA}\
This chapter examines the relevant literature regarding the scope and the three research questions of this thesis. While the five Sections \ref{sec:SOTA_flexibility_options} - \ref{sec:SOTA_congestion_management} focus on already existing literature, Section \ref{sec:SOTA_progress_beyond} presents the contribution of this thesis beyond state of the art. 
Section \ref{sec:SOTA_flexibility_options} presents possible flexibility options that could be integrated into the European electricity market. Both their potential and possible applications are examined. The concept of flexibility markets is described in Section \ref{sec:SOTA_flexibility_markets}. Section \ref{SOTA:effects} examines the benefits of flexibility integration and synergistic and competitive effects between them. Existing modelling approaches and their main conclusions from operators and system perspectives are reviewed in Section \ref{sec:SOTA_flexibility_models}. Congestions and measures to prevent and overcome them are examined in Section \ref{sec:SOTA_congestion_management}.  
\section{Flexibility options}\label{sec:SOTA_flexibility_options}
\subsection{Overview}
Different stakeholders can provide several types of flexibility in the electricity market \cite{Kara2022}. Fossil fuel-fired power plants and controllable \ac{RESE} are used on the supply side. Furthermore, different types of storages can offer flexibility. Regional and local grids can contribute to integrating flexibility into the system through a spatial connection between generation and demand \cite{Moradi2023}. The demand side can provide additional flexibility using flexible loads such as heat pumps, \ac{EV}s, industrial processes or sector coupling \cite{Golmohamadi2022}. The characteristics of the flexibility used: transmission grid, storage, demand-side flexibility, and hydrogen sector coupling are described in this section.
\subsection{Transmission grid}
The significance of transmission grid expansion has been demonstrated in numerous studies \cite{Tejada2015,Gomes2019}, both on the distribution grid level \cite{Moradi2023,Herding2021} and on the transmission grid level \cite{Burholzer2016,Herding2021}. Market coupling between bidding zones through sufficient interconnection capacities can lead to a system-wide optimum \cite{Ringler2017}. However, the rapid expansion of \ac{RESE} stimulates considerable grid investment needs \cite{Allard2020}. This reduces \ac{RESE} curtailment and supports electricity trade between areas with high renewable energy potential and those that rely on electricity import \cite{Herding2021}. Furthermore, additional electricity demand from sector coupling can trigger additional grid investment needs \cite{Lieberwirth2023}.
\subsection{Storage}
As independent market participants, storage is crucial in advancing electricity market system flexibility \cite{Bjorndal2023}. These act on the supply side while discharging and on the demand side while charging. However, their use leads to a higher total electricity demand due to the conversion losses. 


One of the most investigated applications of storage units is using arbitrage as a business model \cite{Brijs2016}. This operational strategy employs price peaks to generate individual profit by buying electricity for charging at times with low prices and selling electricity when prices are high \cite{Nunez2022}. A side effect of this strategy is the reduction of price peaks; however, battery storage units can face significant regulatory barriers that drastically reduce revenue \cite{Gissey2018,Bashi2022,Forrester2017}. Installing storage units on a large scale can also reduce these revenues \cite{Chyong2022,Zafirakis2016}. This is due to the induced price convergence between peak and off-peak electricity prices \cite{Ehler2011}. 


In addition to applying arbitrage as a business model, revenue can be generated on ancillary service markets \cite{Rangarajan2023,Naemi2022}, but both markets can be served simultaneously \cite{Biggins2022}.
\subsection{Demand-side flexibility}
Demand-side flexibility can impact the balance between generation and demand considerably \cite{Saffari2023,Gade2022}. In particular, in an electricity system in which decentralised variable \ac{RESE} replaces large centralised power plants, this flexibility is needed \cite{Plaum2022}. This approach can potentially reduce system-wide $CO_2$ emissions and encourage \ac{RESE} integration through increased investment incentives and reduced curtailment \cite{Li2018,Herre2022}. This can be provided by integrating \ac{EV}s \cite{Arpanahi2022}, heat pumps \cite{Teng2016}, the development of household \cite{Plaum2022,Ettorre2022}, office building \cite{Aduda2016}, and energy communities \cite{Barnes2022}. Whereby the flexibility potential of \ac{EV}s is higher than that of heat pumps \cite{Kroeger2023}. Moreover, as the largest electricity consumer, the industry sector can also provide demand-side flexibility \cite{Heffron2020,Golmohamadi2022}.
\subsection{Sector coupling}
Sector coupling is another strategy for achieving climate neutrality \cite{Cambini2020}. Exemplary, \ac{EV}s or fuel cell \ac{EV}s \cite{Trapp2022} couple the transportation sector with the electricity sector. Heat pumps integrate the heating sector on an individual consumer basis or through local district heating grids \cite{Bernath2021,Jokinen2022}. Hydrogen has a possible role for all these sectors through indirect electrification \cite{Hesel2022}.
\subsubsection{Hydrogen trade and market}
Access to hydrogen in all European countries and unrestricted cross-border hydrogen trade is an essential goal of the European hydrogen strategy \cite{european2020hydrogen} and the European hydrogen backbone \cite{EHB}. This should compensate for the uneven distribution of hydrogen demand and production capacity across countries. Furthermore, it affects hydrogen production in countries based on their installed electricity generation capacity and the resulting electricity price \cite{Ikonnikova2023,Frischmuth2022}. However, pipeline infrastructure for hydrogen trade is expensive, resulting in higher end-user hydrogen prices. As a result, only large hydrogen producers and consumers can use it economically without subsidies \cite{Nastasi2019}. 


Furthermore, the achieved hydrogen price influences investment decisions in generation capacity \cite{VomScheidt2022}. A price of at least 100 \euro{}/MWh\textsubscript{H2} is required to achieve sufficient revenues for hydrogen storage units like pipes or salt caverns \cite{Budny2015}. The general conditions for achieving sufficient profitability in the heating and transport sectors are similar, whereby these are largely determined by the prices of alternative forms of energy cross-price elasticity \cite{Frischmuth2022}. Otherwise, surplus electricity is insufficient to generate significant amounts of hydrogen \cite{Lux2020}. 


Although these studies promote the need for sufficiently high hydrogen prices, the natural gas price must still be competitive \cite{NavasAnguita2020}. Otherwise, natural gas steam reformation would be more profitable than electrolysers, resulting in significantly higher CO\textsubscript{2} emissions, that must be captured and stored. This demonstrates a strong relationship between the hydrogen price, hydrogen production, and the competitiveness of the electrolyser.
\subsubsection{Hydrogen types}
Various methods of producing hydrogen result in varying CO\textsubscript{2} emissions. Their production capacity is constrained. A color scheme differentiates between these methods, \cite{Hydrogen_colors}. 
\begin{itemize}
\item \textcolor{gray}{\textbf{Gray}} hydrogen is produced by a steam methane reforming process that uses natural gas, resulting in high CO\textsubscript{2} emissions.
\item \textcolor{blue}{\textbf{Blue}} hydrogen created through a steam methane reforming process that uses natural gas, followed by carbon capture and storage, resulting in low CO\textsubscript{2} emissions.
\item \textcolor{cyan}{\textbf{Turquoise}} hydrogen is produced by methane pyrolysis using natural gas, resulting in medium CO\textsubscript{2} emissions. The process results in hydrogen and solid carbon, which can be used in various applications, including steel production and battery storage units.
\item \textcolor{orange}{\textbf{Yellow}} hydrogen is created by electrolysing water with electricity from any available power plant. The associated CO\textsubscript{2} emissions depend highly on the dispatch and generation mix. 
\item \textcolor{magenta}{\textbf{Pink}} hydrogen is produced by electrolysing water with nuclear power plant electricity. This process does not lead to CO\textsubscript{2} emissions.
\item \textcolor{green}{\textbf{Green}} hydrogen is produced by electrolysing water with \ac{RESE}. There are no direct CO\textsubscript{2} emissions from this process. However, every technology produces a certain level of CO\textsubscript{2equ} emissions throughout its lifecycle. 
\cite{Frischmuth2022}
\end{itemize}
The hydrogen types do not only influence the resulting CO\textsubscript{2} emissions but also the revenues of the market participants and different electricity generation technologies. However, generated hydrogen types are mainly determined by policy and regulatory framework. The European hydrogen strategy defines preferred renewable hydrogen as hydrogen produced using \ac{RESE} \cite{european2020hydrogen}. In some countries where the electricity supply is heavily based on other technologies, alternative strategies predominate. For example, in France, the national hydrogen strategy includes green and pink hydrogen \cite{H2_Strat_France}.
\section{Flexibility markets}\label{sec:SOTA_flexibility_markets}
While this thesis focuses on implicit flexibility controlled by price signals, there is the possibility to implement explicit flexibility traded on markets \cite{FREIREBARCELO2022107953}. Alternative market schemes can offer a solution for their integration if flexibility providers' revenue is insufficient to trigger investment incentives in the energy-only market \cite{Rious2015}. An option is to integrate a strategic reserve to the energy-only market \cite{Keles2016}. Another way to achieve this goal to encourage the integration of flexibility are flexibility markets \cite{Ostovar2023}.


Such markets have already been established in several countries; for example, "flexiramp" in California and "ramp capability" in the Midcontent Independent System Operator in the central US \cite{Kara2022}. The market scheme with centralised flexibility scheduling can establish more efficient price structures and improve individual revenue \cite{Ela2016}. But also, the introduction of flexibility markets needs a cheap technical solution, without market entry barriers, to integrate these flexibility markets \cite{Newbery2018,Hoeschle2017}. Especially price signals must be provided. A uniform smart meter rollout is necessary to transmit these signals and to perform and monitor the corresponding interactions \cite{Newbery2018}. 


Moreover, flexibility must be defined as a commodity. This can be done by the committed capacities available to reduce demand peaks \cite{Wang2016}. Based on this, commitment costs will influence, on the one hand, flexibility provision and, on the other hand, the wholesale market \cite{Orvis2018}. This leads not only to welfare shifts between flexibility and wholesale markets but there are also market shifts to the balancing and redispatch markets \cite{GLISMANN2021117697}.
\section{System-wide effects of flexibility integration}\label{SOTA:effects}
Several effects on the electricity market are triggered by flexibility integration. While there are several beneficial effects for the whole system, several types of flexibility compete, while others can generate synergistic effects if implemented together. 
\subsection{Beneficial effects}  
\subsubsection{Hydrogen system benefits}
Hydrogen can be used in industry and sectors that are difficult or impossible to electrify. These face significant decarbonisation challenges \cite{Nurdiawati2021}. Using hydrogen is one promising option for those \cite{Frischmuth2022,Orths2019}. It is possible to decarbonise transportation by using fuel cell electric vehicles, trucks, ships and airplanes and replace natural gas in the heating sector. Further, numerous industrial applications exist, such as steel or chemistry production \cite{Griffiths2021}.


In addition to this, there are positive effects on the electricity market as well. Hydrogen integration can aid in decarbonising the energy system \cite{Eise2022}, while \ac{RESE} curtailment, and negative electricity prices can be avoided \cite{Roach2020}. As a result, profits from wind turbines and \ac{PV} systems can be increased \cite{Ruhnau2022}, whereas the volatility of electricity prices can be reduced \cite{Li2021,Hesel2022}. Moreover, electrolysers can provide ancillary services \cite{Dadkhah2020, Samani2020}. 
\subsubsection{Electric vehicle system benefits}
However, also the integration of \ac{EV}s into the electricity market can generate beneficial effects. At the local level \ac{EV}s can be used for peak shaving \cite{LiX2020,Ioakimidis2018,VanK2021,ZhengY2021} and valley-filling \cite{Ioakimidis2018,Jian2017,Zhang2014,ZhangL2014}. This can avoid the curtailment of \ac{RESE} \cite{Haddadian2015,Haddadian2016,Schuller2015}. Furthermore, grid utilisation can be reduced by using \ac{EV}s to transport electricity \cite{Khodayar2012,Nikoobakht2019,Verzi2014}. In addition, they can improve the matching of generation and demand \cite{Bibak2022} and reduce peak loads \cite{Tan2017}. 


A different use case for \ac{EV}s is the provision of ancillary services and balancing reserve provision \cite{Osorio2021,Thingvad2019,Figgener2022,XiaoG2013}. Especially for fast frequency containment reserve and the even faster spinning reserve \cite{Pavic2015}.
\subsection{Competitive effects}
In general, the profitability of flexibility decreases in electricity markets when it is installed on a large scale due to a price-smoothing effect \cite{Chyong2022}. Competition between flexibility options such as battery storage units, sector coupling and demand-side flexibility can lead to decreased revenue. Sector coupling has a significant impact on the storage systems' market viability \cite{Scheller2020}. Transmission grid expansion reduces the use of storage and demand-side flexibility \cite{Allard2020}. Demand-side flexibility, in turn, reduces the need for transmission grid expansion \cite{Quintero2022}.


Combining different types of flexibility can reduce the value of electricity storage, leading to reduced investment incentives \cite{Pudjianto2014}. This will result in \ac{PS} and \ac{CR} reductions. Storage units compete with demand-side flexibility when they can flatten demand peaks. An insufficient grid capacity can generate congestion between two bidding zones, resulting in price divergence. This changes the potential revenue from storage flexibility \cite{Hurta2022}; however, the \ac{PS} may increase in one area while decreasing in another. 
\subsection{Synergistic effects}
A joint expansion of several flexibility options can generate higher individual revenue \cite{Nagel2022}, although individual revenue is strongly dependent on the other available flexibility options. Combining different flexibility options can garner results from a systems perspective as fewer local grid investments are needed \cite{Kara2022_market_design}. This effect can be achieved in the distribution and transmission grid if a proper market-based framework is established \cite{Khodabakhsh2023}; however, the chosen application of battery storage can impact the expected revenue significantly \cite{Parra2016,Klinger2018}. Combining other flexibility options, such as heat pumps, can further differentiate between synergy and competition, reducing revenue. The flexibility provided by transmission grid expansion and demand-side flexibility can increase \ac{SEW} and the value of demand-side flexibility \cite{Nouicer2023}; however, the regulatory environment can impact these results and the underlying investment incentives considerably \cite{Grimm2020,Bergaentzle2022}. Therefore, it is essential to develop a market environment that fosters the simultaneous integration of different flexibility types instead of only selected types \cite{Gissey2018}.
\section{Flexibility models}\label{sec:SOTA_flexibility_models}
Several existing modelling approaches and the main conclusion of flexibility integration are described in this section.
\subsection{Overview}
While the implementation of storage and the transmission grid is standardised in most electricity market models, there are numerous options for integrating other sectors through sector coupling. This is referred as the integration and interaction of two or more energy systems, which is a promising option for achieving climate neutrality \cite{Cambini2020}. The demand, generation, and storage possibilities in other sectors provide flexibility to the electricity market \cite{Fridgen2020}. This increases the amount of flexible demand that is met by inflexible generation \cite{Gea2021}. 


To provide demand-side flexibility, several sectors can be linked with the electricity market. \ac{EV}s and fuel cell \ac{EV}s \cite{Trapp2022} are used in the transportation sector. Heating is provided by decentralised heat pumps and by district heating grids \cite{Bernath2021,Jokinen2022}. Industry, heating, and transport can be integrated by the use of hydrogen \cite{Hesel2022}.
\subsection{Hydrogen}
As sector coupling between the electricity and hydrogen markets includes components of transportation, heating and industrial sectors \cite{Frischmuth2022}, it is chosen as one important flexibility option in this thesis. This section differentiates between effects from the system- and operators' perspectives. 
\subsubsection{System perspective}
On the one hand, hydrogen’s increased flexibility reduces the demand for peak-load power plants, but on the other hand, its use for reconversion to electricity results in increased hydrogen demand. Furthermore, there is a strong link between the use of battery storage and the use of hydrogen to meet peak demand \cite{Frischmuth2022}. The resulting effects of hydrogen generation on electricity prices \cite{Budny2015,Sijm2021} have a strong influence on investment decisions in other flexibility options \cite{Frischmuth2022}. However, overall \ac{SEW} benefits can be investigated, for example, in Germany \cite{Budny2015} and in the Netherlands \cite{Sijm2021}. 


On the one hand, synergistic effects between hydrogen use, \ac{PHS}, and battery storages occur on European scale can occur \cite{Panos2019}. On the other hand, using hydrogen as flexibility can increase the market revenue of storages, wind turbines, and \ac{PV} systems \cite{Schlund2021} significantly because high \ac{RESE} shares will lower electricity prices and prevent further investment decisions without additional demand or flexibility. By integrating the hydrogen sector coupling, the market value of these generation types will stabilise even in 2050, which makes further investments feasible  \cite{Ruhnau2022}. Hence, subsidising electrolysers can help to reduce the costs of \ac{RESE} subsidies \cite{Roach2020}. 


In addition, there is also potential for hydrogen as seasonal storage. These are crucial for decarbonised electricity systems \cite{Eise2022}. They are required to compensate for seasonal variations in generation and demand. While \ac{PHS} with large storage capacities are ideal for seasonal storage, specific geographic conditions must be met. This limits their potential for growth \cite{Samani2020}. In general, battery storage units with high round-trip efficiency but high capacity investment costs cannot be used for this application. Hydrogen storages with lower round-trip efficiency but also lower investment costs are a possibility \cite{Arsalis2022}. Using hydrogen as seasonal storage in this future electricity market can abate the final 5-10\% of CO\textsubscript{2} emissions towards a zero-emission energy system \cite{Petkov2020}. 
\subsubsection{Operator perspective}
Aside from the perspective of energy system planning, many publications deal with specific hydrogen business models, including storage options, infrastructure and electrolysers as generation technology. 


Hydrogen as seasonal storage can be a sufficient business model for an energy provider if the local community aims to be self-sufficient \cite{Eise2022}. In this case, hydrogen can help to balance the seasonal variations in generation and demand. However, a sufficient grid capacity is required to maximise profits because using electrolysers increases the grid capacity requirements \cite{Rabiee2021}. A comparison of different storage technologies reveals that hydrogen systems can be used for seasonal storage while batteries are helpful for household sizing \cite{Arsalis2022}. In particular, using hydrogen as seasonal storage combined with fuel cells and heat pumps helps achieve a fully decarbonised energy system \cite{Petkov2020}. Whereby relatively high hydrogen prices are required to render hydrogen storage units such as pipes or salt caverns economically viable \cite{Budny2015}.


While these studies focus on hydrogen storage units, the economic potential of electrolysers is critical for the integration of hydrogen on a large scale. For example, in the Netherlands, high hydrogen demand combined with an electricity market with a remarkably high share of \ac{RESE} reduces electricity price volatility. A CO\textsubscript{2} price of at least 150 \euro{}/t is needed to operate economically \cite{Li2021}. Aside from the CO\textsubscript{2} price, the electricity price significantly impacts investments in electrolysers \cite{VomScheidt2022}. An extremely low exogenous defined CO\textsubscript{2} emission target \cite{Oberg2022} is another possibility for increasing the competitiveness of investments in hydrogen generation technology. Furthermore, the competitiveness between steam methane reforming and electrolysers is crucial in an unregulated hydrogen market, because the relationship between natural gas- and hydrogen prices significantly impacts the produced hydrogen types \cite{NavasAnguita2020}. 


Moreover, participating in the reserve market lowers the break-even price of hydrogen \cite{Dadkhah2020}. This possible application of electrolysers providing ancillary services has no negative impact on their lifetime \cite{Samani2020}. However, switching the electrolysers completely on and off frequently in stand-alone systems is not recommended. Hence, to provide an adequate response to grid frequency variation and to meet the technical specifications, the electrolyser is expected to operate continuously in a variable manner to avoid start-up and shut-down.  
\subsection{Electric vehicle}
This section describes the effects of \ac{EV} integration from the system- and operator perspective. The results are mainly from studies that model \ac{EV}s as bidirectional electricity storage systems. Because this thesis focuses on demand-side flexibility, the calculation of charging demands is also examined.
\subsubsection{System perspective}
The demand-side flexibility and the use of \ac{EV}s as bidirectional storage can contribute several benefits to the electricity market. They can reduce frequency deviation, minimise local peak demand \cite{Tan2017} and transfer electricity throughout the network whilst reducing grid operation costs \cite{Khodayar2012,Nikoobakht2019}. Furthermore, they can reduce overall system operating costs, \ac{RESE} curtailment \cite{Haddadian2015,Haddadian2016} and CO\textsubscript{2} emissions \cite{Lund2008}. Price signals can change the temporal distribution of \ac{EV} charging events \cite{Azad-Farsani2021}.
\subsubsection{Operator perspective}
Numerous studies deal with the optimal marketing of \ac{EV} demand-side flexibility from perspective of individual market participants. They can maximise their economic benefits in electricity \cite{Cramer2021} and ancillary service markets \cite{Osorio2021}. This mainly includes charging with minimal costs \cite{Zheng2020,Ghasemi2015}.
\subsubsection{Electric vehicle charging profiles}
To calculate the effects of \ac{EV} integration on the electricity market \ac{SEW}, a realistic integration of \ac{EV}s is crucial. This includes estimating their charging and driving patterns based on measurement data. Statistical analyses of real-world data, the temporal distribution of charging processes and their flexibility potential are performed in the Netherlands \cite{Flammini2019}, Germany \cite{Schauble2017} and in the USA \cite{Tehrani2015,Harris2014}, which showed that a system-wide \ac{EV} rollout increases demand at certain hours \cite{Tehrani2015}. 


In addition to analysing measured data, simulating and predicting charging processes is possible. The charging behaviour could be predicted with machine learning algorithms using user and charging infrastructure data \cite{Majidpour2016} and simulated using Monte Carlo methods \cite{Harris2014}.
\subsection{Demand-side flexibility}
Integrating demand-side flexibility into a fully renewable fundamental electricity market model compensates for imbalances, reduces the need for electricity generation of thermal power plants and decreases \ac{RESE} curtailment \cite{Kies2016}. \ac{EV}s can manage imbalances resulting from volatile \ac{RESE} generation \cite{Ghasemi2016}. These positive effects of flexibility persist if conventional power plants dominate the generation mix. Hence, if the \ac{RESE} penetration is very low, flexibility can still contribute to achieving environmental goals \cite{Bergaentzle2014}. Furthermore, it leads to a change in price-setting marginal power plants, leading to a change in revenue and investment decisions.
\section{Congestion management}\label{sec:SOTA_congestion_management}
\subsection{Congestion management in the European electricity market design}
Since both dispatch and redispatch are modelled in this thesis, defining these terms is crucial. The dispatch of the available generation units is primarily determined by clearing the \ac{DA} electricity market. On the \ac{DA} market, electricity is traded anonymously for the following day. Cross-border trades between bidding zones require implicit allocation of cross-border capacities. These trades occur on the international power stock market or over the counter. The dispatch includes the optimal generation schedules for all thermal power plants, \ac{RESE} and the use of storage. Therefore, the demand and supply curves are intersected. The intersection directly results in the market clearing price for the respective hour. All market participants subsequently pay or get this \ac{DA} price. The supply bids are based on the \ac{SRMC} of the generation units and the merit-order function within a perfectly competitive market \cite{Khezr2021}. 


During dispatch, transmission line limitations within a bidding zone are not considered. Hence, congestion could occur. The dispatch is forwarded to the \ac{TSO} to identify them in advance. The latter subsequently performs a load flow calculation to identify transmission grid congestion. Redispatch measures must balance congestion that would occur. Such a balance is realised by modifying the power generation of several thermal power plants based on the instructions of the \ac{TSO}. A power plant will increase its power generation on one side of the congestion and reduce it on the other side to overcome this congestion. The corresponding power plants are financially equalised as if there had not been any redispatch measures. Aggregated consumers can participate in the congestion management market in Austria \cite{Osterreich2021}. This should encourage active customer participation. They can adjust their demand to the actual local generation to reduce the power exchange on the transmission grid. 
\subsection{Congestion management and redispatch measures}
Congestion management in electricity markets is often associated with costs because intervention in the cost-optimal dispatch is required. In addition to costly methods that prevent transmission line congestions, the high potential of specific technologies to reduce redispatch costs is examined in several studies. 


This includes, on the one hand, technical solutions, such as flexible AC transmission systems \cite{Pillay2015}, and, on the other hand, market designs and enhanced trading capacities through the implementation of  \ac{FBMC} \cite{Lang2020}. This approach considers several aspects of the grid topology within the European electricity market dispatch and reduces the necessary redispatch measures. If these methods are insufficient, redispatch measures that regulate thermal power plants must be used \cite{Linnemann2011}.


Besides already established market schemes and methods, aggregated prosumer participation can contribute to balancing redispatch needs \cite{Pantos2020}. This includes an extensive rollout of bidirectional \ac{EV}s \cite{Staudt2018} and demand-side flexibility to minimise redispatch costs \cite{Zaeim-Kohan2018}.


Moreover, the prevention of congestion can be realised by distributed regulation despite the absence of a central planner method if neighbouring nodes communicate with each other \cite{Xu2018}. This form of preventive regulation within the dispatch can raise \ac{CS} and cross-border trading capacity between bidding zones \cite{Poplavskaya2020}. However, also on \ac{TSO} level, communication and joint redispatch measures can influence redispatch costs \cite{Kunz2015} and the \ac{RESE} curtailment \cite{Xiong2021}. 