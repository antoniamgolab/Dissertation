\section{Case study and scenarios}\label{sec:methods_scenarios}
In order to answer this thesis's three main research questions, the described model uses different configurations and datasets for each of them. 
\subsection{Flexibility competition and synergy}
This section describes the case study that evaluates synergistic and competitive effects on \ac{SEW} between several flexibility options. For this purpose, different combinations of flexibility options are implemented using several scenarios. These options include storage expansion, transmission line expansion, \ac{EV} demand-side flexibility, and hydrogen sector coupling. These provide different system benefits constrained by each option's features. Flexibility characteristics of each option are presented in Table \ref{table:flexibility_character}. 

\begin{table}[h]
\centering
\begin{tabular}{llll}
\hline
\textbf{Flexibility option}  & \textbf{Energy}  & \textbf{Power} & \textbf{Duration} \\ \hline
Transmission line & not limited                   & limited & not limited       \\ \hline
Demand-side flexibility     & constant                      & limited & hours         \\ \hline
Sector coupling             & \begin{tabular}[c]{@{}l@{}}increases demand\end{tabular}  & limited  & not limited \\ \hline
Storage     & \begin{tabular}[c]{@{}l@{}}limited; increases demand\\due to conversion losses\end{tabular}                   & limited  & \begin{tabular}[c]{@{}l@{}}hourly\\to seasonal \end{tabular}     
\\ \hline
\end{tabular}
\caption{Flexibility characteristics}
\label{table:flexibility_character}
\end{table}
All 13 countries (Austria, Germany, Netherlands, Belgium, Luxembourg, Czech Republic, Slovenia, Switzerland, Poland, Slovakia, Hungary, Italy, and France) within the optimized area are expected to fulfil the European \ac{NECP} \cite{NECP2021} by 2030. This includes for example expanding \ac{RESE} and decommissioning coal-fired power plants. According to Austria's \ac{NECP}, \ac{RESE} compensates for 100\% of the yearly electricity demand. The used \textit{\ac{NT} 2030} dataset, provided by the \ac{ENTSOE}, is based on the fulfilment of these plans \cite{TYNDP2022}. It includes a power plant fleet with \ac{RESE}, fossil fuel-fired power plants (coal, oil, lignite and \ac{CCGT}) and other types of thermal power plants that do not emit CO\textsubscript{2} (nuclear, biogas and biomass).
\subsubsection{Scenarios}
A reference scenario is used to evaluate the changes in \ac{SEW}, \ac{PS}, \ac{CS}, and \ac{CR} of the whole system based on the integrated flexibility options. All scenarios evaluated for 2030 are listed in Appendix \ref{sec:scenario_overview} and described as follows:
\begin{itemize}
 \setlength\itemsep{-0.5em}
    \item The \textit{Baseline-1} and \textit{Baseline-2} serve as reference scenarios for the examined scenarios. The \textit{Baseline-1} scenario uses the current transmission grid, including a 4 900 MW \ac{NTC} between Austria and Germany (bold line in Figure \ref{fig:market_area}). This grid between Austria and Germany is expanded in several scenarios through the implementation of \ac{ENTSOE} \ac{TYNDP} 2022 projects TR47, TR187, and TR313 \cite{TYNDP2022}. These provide an overall capacity expansion of 4 100 MW. Based on these plans, the \textit{Baseline-2} scenario considers a total capacity of 9 000 MW. The case study between Austria and Germany's bidding zones was chosen as a representative European example. Centrally located in Europe, Austria has about 9 GW of turbine capacity from hydropower plants and \ac{PHS} due to the alpine arc. This compares to an annual electricity demand of 80 TWh in the chosen scenario. Conversely, Germany has an electricity demand of 584 TWh, which is more than seven times higher, but it has approximately the same turbine capacity of 10.3 GW. In addition, Germany possesses nearly 100 GW of installed wind power capacity. Thus, in windy periods, there is excess generation that should be exported or stored. Expanding the interconnectors between the two bidding zones can massively promote the storing and use of electricity generated by wind turbines in Europe. 
    \item A single storage expansion is evaluated in the scenarios \textit{Storage Austria-1} and \textit{Storage Germany-1}. Data published in the \ac{TYNDP} are used to apply grid and storage expansion with similar capital investment costs \cite{TYNDP2022}. This results in a representative \ac{PHS} with 700 MW pump- and turbine power. 7 GWh capacity is assumed as we use a typical relationship between capacity-to-power relation of 10 \cite{Diawuo2022}. In addition to storage, the \textit{Storage Austria-2} and \textit{Storage Germany-2} scenarios introduce interconnector expansion.
    \item The \ac{EV} demand-side flexibility combined with interconnector expansion is evaluated in the \textit{EVDSM only} scenario. The \textit{EVDSM Storage} scenario additionally includes the storage expansion. \ac{EV} demand-side flexibility does not change the overall electricity demand, but introduces the possibility to temporal shift demand. Hence, the benefits of the flexibility are evaluated. The 0.75 million \ac{EV}s charge using 11 kW of power. Overall, around 5.15 million passenger cars operate in Austria \cite{KFZAustria}. Thus, the electrification rate of Austria's private individual transport sector is 15\% \cite{Statharas2019}. The charging strategies prior to introducing flexibility are divided into 30\% charge immediately after the plug-in (\textit{immediately}) and 70\% charge with the lowest possible power to meet electricity demand (\textit{peak-shaving}).
    \item Sole sector coupling to the hydrogen market is evaluated in the \textit{Hydrogen only} scenario. In contrast, the \textit{Hydrogen Storage} scenario adds storage expansion and the \textit{Hydrogen EVDSM} scenario adds the \ac{EV} demand-side flexibility. A hydrogen price of 61.67 \euro{}/MWh and a $CO_2$ price of 70 \euro{}/t are assumed based on the \ac{TYNDP}2022 \ac{NT} dataset \cite{TYNDP2022}.
    \item The combined use of storage expansion and \ac{EV} demand-side flexibility is evaluated in the \textit{Flexibility-1 w/o H2} and \textit{EVDSM Storage} scenarios. Whereas the former does not include transmission grid expansion. All flexibility options are implemented in the \textit{Flexibility-1} scenario without transmission line expansion and in the \textit{Flexibility-2} scenario with transmission line expansion. 
\end{itemize}
\subsubsection{Model setup}

Figure \ref{fig:market_area_RQ1} illustrates the nodes and interconnectors used to answer this research question. The transmission line expansion in several scenarios is marked as a thick black line. The dispatch model uses all model components and flexibility options. This includes hydrogen production, storage and reconversion to electricity. The power exchange is calculated using the \ac{NTC} approach with one node per country. The redispatch calculation is not performed in this use case. 
\begin{figure}[ht]
\centering
      \includegraphics[width=0.48\textwidth]{graphics/RQ1/Europakarte_EDisOn.jpg}
      \caption{Nodes and grid of the electricity market model}
      \label{fig:market_area_RQ1}
\end{figure}
\FloatBarrier
\subsection{Hydrogen sector coupling}
This section describes the case study that evaluates the effects of the sector coupling between the European electricity and national hydrogen markets in detail. All 13 countries (Austria, Germany, Netherlands, Belgium, Luxembourg, Czech Republic, Slovenia, Switzerland, Poland, Slovakia, Hungary, Italy, and France) within the optimized area are expected to fulfil the European \ac{NECP} \cite{NECP2021} by 2030. This includes for example expanding \ac{RESE} and decommissioning coal-fired power plants. According to Austria's \ac{NECP}, \ac{RESE} compensates for 100\% of the yearly electricity demand. The used \textit{\ac{NT} 2030} and \textit{\ac{NT} 2040} scenarios provided by the \ac{ENTSOE} are based on the fulfilment of these plans \cite{TYNDP2022_scenarios}. The \ac{SRMC} of thermal power plants include a CO\textsubscript{2} price of 70 \euro{}/t in 2030 and 90 \euro{}/t in 2040.
\subsubsection{Scenarios}\label{sec:results_hydrogen_data}
The electrolyser and fuel cell capacities are based on the individual countries’ national hydrogen strategies. All installed electrolyser and fuel cell capacities are listed in Appendix \ref{sec:scenario_overview_RQ2}. As a result, the overall electrolyser capacity in 2030 is 26 GW and therefore slightly less than the European target of 40 GW \cite{su132313464}. All scenarios evaluated for 2030 and 2040 are listed in Appendix \ref{sec:scenario_overview_RQ2} and described as follows:
\begin{itemize}
 \setlength\itemsep{-0.5em}
    \item Each optimization is conducted with 8760 one-hour timesteps and is done for the years 2030 (\textit{2030}) and 2040 (\textit{2040}).
    \item  For both years, a distinction is made between using only electrolysers (\textit{Production}) or electrolysers, fuel cells and storage units (\textit{Complete}). This distinction allows to separately evaluate the impact of the two components, demand-side flexibility and seasonal storage. This results in four scenarios.
    \item Each scenario is optimised for a range of hydrogen prices as a sensitivity analysis (Table \ref{table:cost_assumptions}). Hence, five different hydrogen prices (\textit{0},\textit{50},\textit{100},\textit{150}, and \textit{200}) are assumed.
\end{itemize}
The \textit{Production-2030-0} and \textit{Production-2040-0} scenarios, with a hydrogen prices of 0 \euro{}/MWh\textsubscript{H\textsubscript{2}}, serve as a reference scenario for all other scenarios. There is no incentive to produce hydrogen in these scenarios. On the one hand, additional \ac{PS} or \ac{CS} cannot be obtained (Section \ref{sec:welfare_calculation}). On the other hand, there are no fuel cells to stimulate hydrogen production through sufficient electricity price fluctuations.
\subsubsection{Model setup}
Figure \ref{fig:market_area_RQ2} illustrates the nodes and interconnectors used to answer this research question. The dispatch model uses the hydrogen sector coupling as a flexibility option. This includes hydrogen production, storage and reconversion to electricity. \ac{EV}s and chargeable demand-side flexibility are not part of these evaluations. The power exchange is calculated using the \ac{NTC} approach with one node per country. The redispatch calculation is not performed in this use case. 
\begin{figure}[ht]
\centering
      \includegraphics[width=0.48\textwidth]{graphics/RQ2/Europakarte_EDisOn.jpg}
      \caption{Nodes and grid of the electricity market model}
      \label{fig:market_area_RQ2}
\end{figure}
\FloatBarrier
\subsection{Redispatch}
This section describes the case study that evaluates the influence of customer integration as aggregated \ac{EV} fleet demand-side flexibility on congestion and needed redispatch measures. All 13 countries (Austria, Germany, Netherlands, Belgium, Luxembourg, Czech Republic, Slovenia, Switzerland, Poland, Slovakia, Hungary, Italy, and France) within the optimized area are expected to fulfil the European \ac{NECP} \cite{NECP2021} by 2030. This includes for example expanding \ac{RESE} and decommissioning coal-fired power plants. This case study allows the evaluation of the effects of \ac{EV} demand-side flexibility in an electricity market with a high share of \ac{RESE}. The used \textit{\ac{NT} 2030} scenario provided by the \ac{ENTSOE} is based on the fulfilment of these plans \cite{NT2030}. The used parameters can be found in Appendix \ref{sec:appendix_parameters}. 


The used \ac{CT}s  and subsequently the \ac{CD}s of all \ac{EV}s are based on the statistical evaluation of real-world charging events \cite{Flammini2019} in combination with statistical data.
The charging strategies used are divided as follows: 70\% \textit{immediately}, 20\% \textit{partly peak-shaving} and 10\% \textit{peak-shaving}.
\subsubsection{Scenarios}
All scenarios evaluated for 2030 are presented in Appendix \ref{sec:appendix_parameters}.
The optimisation is conducted as a rolling horizon optimisation with 365 24 hour \ac{DA} optimisation steps. The market premium ($P^{Premium}$ ) as the price for \ac{RESE} curtailment as a redispatch measure is set to 60 \euro{}/MWh.
The redispatch is optimised with a different number of \ac{EV}s to demonstrate the influence of aggregated flexibilities as a congestion management compensation system. 


The cost-minimal redispatch is performed with 200 000 (\textit{Scen. 200k}), 600 000 (\textit{Scen. 600k}), and 2 000 000 (\textit{Scen. 2M}) \ac{EV}s as demand flexibility devices. Overall, around 5.15 million passenger cars operate in Austria \cite{KFZAustria}. Thus, the electrification rate of Austria's private individual transport sector is 4\%, 12\%, and 39\% \cite{KFZAustria}. Each transport electrification level is optimised with three different flexibility setups: 
\begin{itemize}
 \setlength\itemsep{-0.5em}
    \item Firstly, the flexibility is optimised in the redispatch. Austrian law allows aggregated consumers to participate in the congestion management market due to the increasing share of \ac{RESE} and to encourage active customer participation\cite{Osterreich2021}. Hence, the \ac{EV}s participate in the congestion management market to reduce redispatch needs (\textit{Scen. Redispatch DSM}).
    \item Secondly, \ac{EV}s do not actively participate in the electricity market. Demand-side flexibility can neither be used in the dispatch nor the redispatch (\textit{Scen. no DSM}). Hence, the summation of the \ac{CD}s serves as additional inflexible demand. This scenario forms the "baseline" to analyse the effect of the other two scenarios. 
    \item Thirdly, flexibility is available in the dispatch. Therefore, their \ac{CD} will be adapted to the \ac{DA} price (\textit{Scen. Dispatch DSM}). 
\end{itemize}
\FloatBarrier
\subsubsection{Model setup}
To answer this research question, the dispatch and the redispatch model are used. A reduced set of aggregated nodes is used in Austria to determine the optimal dispatch from an economic perspective. This set contains nodes that are connected via critical transport lines. Transmission lines within the new nodes in Austria are aggregated; however, the absence of congestion of these lines is not guaranteed. Meanwhile, transfer capacities between the bidding zones are allocated within the optimisation. Figure \ref{fig:Nodes_dispatch_RQ3} shows the nodes and transmission grid used within the market-based dispatch. 

\begin{figure}[h]
\centering
   \begin{minipage}[b]{.45\linewidth} % [b] => Ausrichtung an \caption
      \includegraphics[width=1\textwidth]{graphics/RQ3/Nodes_dispatch.jpg}
      \caption{Nodes and grid of the dispatch model}
      \label{fig:Nodes_dispatch_RQ3}
   \end{minipage}
   \hspace{0.05\linewidth}% Abstand zwischen Bilder
   \begin{minipage}[b]{0.45\linewidth} % [b] => Ausrichtung an \caption
        \includegraphics[width=1\textwidth]{graphics/RQ3/Nodes_redispatch.jpg}
        \caption{Nodes and grid of the redispatch model}
        \label{fig:Nodes_redispatch_RQ3}
   \end{minipage}
\end{figure}
After the dispatch model has minimised the overall electricity generation costs with an aggregated set of nodes ($N$) and transmission lines (Figure \ref{fig:Nodes_dispatch_RQ3}), these nodes are disaggregated (Figure \ref{fig:Nodes_redispatch_RQ3}) ($N^{All}$) before performing the cost minimal redispatch in Austria. Thus, the differences between the market-based optimum and the actual physical power flows that may lead to congested transmission lines and must be compensated are analysed. The redispatch needs are balanced cost-minimal by adjusting the generation of thermal power plants, additional \ac{RESE} curtailment, and the use of \ac{EV} demand-side flexibility. Hydrogen sector coupling and chargeable demand-side flexibility are not part of these evaluations to evaluate the effects of demand-side flexibility without competition with other flexibility options. The power exchange in both optimisations is calculated using the \ac{PTDF} approach with 9 (dispatch) and 17 (redispatch) nodes in Austria and one node per each other country.