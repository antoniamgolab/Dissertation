\chapter{Review of State-of-the-Art \& Progress Beyond}
    \section{The electrification of the private passenger cars sector}
        \subsection{Towards large-scale electrification of passenger car fleets}

        The pathways towards a carbon-emission-free passenger transport sector are most times determined by three complementary levers: the reduction of travel demand, the shifting of demand towards low-carbon modes and the implementation of zero-emission technologies [REF]. While the first two levers require substantial behavioral and structural changes, the technological transformation allows for maintaining the current comfort and flexibility provided by current private passenger car fleets [REF].

        For technological substitution, the transformation towards a passenger car fleet of battery-electric vehicles has been solidified to be the most cost-efficient way to decarbonize. This has been extensively demonstrated: Studies on the total cost of ownership show the cost benefits of battery-electric passenger cars that stem mainly from \dots \textit{?} [REF]. Life cycle analyses show significant reductions in carbon emissions due to the usage of battery-electric vehicles, best achieved with renewable electricity sources. This sector-overarching cost-efficiency has been also evident in studies applying energy system models to quantify decarbonization pathways in the energy system under various future scenarios of policies,  [REF]. The performance parameters of the battery-electric drive-train, particularly the driving range and charging speed, have significantly improved during the last decades and this, together with future trajectories, paint a clear picture on the system-wide benefit of the large-scale adoption of battery-electric vehicles.

        With the current fleet of \dots Mio. battery-electric passenger cars versus \dots fossil-fueled passenger cars world-wide, the battery-electric vehicle is still in the early adoption phase [REF]. Empirical research has shown that early adopters are characterized by high income and home ownership [REF]. The cost of purchase are currently a prominent barrier to the purchase of a battery-electric vehicles [REF]. In many countries, particularly, throughout European countries, this barrier has been tackled through subsidization [REF]. The costs of BEVs have significantly decreased and cost-parity has been achieved for different car models \hl{?}[REF]. Though, the purchase of a battery-electric vehicle remains unattainable by lower-income classes due to (a) the lack of availability in the range of car models, most battery-electric vehicles being luxury cars, (b) the subsidization of the purchase costs being increasingly removed from policies, and (c) most lower-income class people buy vehicles from the second-hand market, for BEVs this has not been established yet.

        [REFs] have reviewed socio-economic barriers to BEV adoption. These include \hl{\dots}. \textit{Mention/acknowledge here most prominent other barriers}. A prominent hinderer of the large-scale adoption of battery-electric vehicles, is the \textit{range anxiety}, which relates to the fear of the potential lack of charging infrastructure [REF 13]. This fear is also driven by the shorter driving ranges of BEVs than internal combustion engine cars. Further, the increased time for refueling the vehicle by the charging processes being substantially longer than a typical refueling of diesel or petrol, is perceived as inconvenient. These perceived inconviences and fears of potential adopters relate to technical aspects that have been tackled by the following:
        \begin{itemize}
            \item significant improvements in the battery technology performance, driving range, and charging power  
            \item that most traveled routes are of lengths below the driving range, making the battery capacities sufficient
            \item convinient charging opportunities that align with downtime of vehicles, and may even provide 
            \item \textit{now making the point of other fast charging is needed --- } sufficient charger network is needed to reduce the fear of a potential lack, i.e. range anxiety;  and the provision of charging at higher power levels to enable long-distance travel is vital
        \end{itemize}

        

        
        
\textbf{Lead questions}:
\begin{itemize}

    \item \textit{What are the driving dimensions of the decarbonization of the private passenger car sector? shift---improve---avoid; How does the electrification fit here? TCO, LCA, }
    \item \textit{What have been most effective incentives?} 
    \item \textit{What are the main drivers of the electrification of passenger car fleet?}
    \item \textit{Who is adopting the vehicles?}
    \item \textit{What are the barriers to overcome --- for large-scale adoption? on social level; range anxiety, charging duration, availability of BE passenger car models}
    
    \item \textbf{Main outcomes should be:}
    \begin{itemize}
        \item unclear development of the demand and electrification (paper 1 contribution point)
        \item Income is plays an important role here (paper 2 contribution point) 
        \item at end: transition to the role of charging infrastructure availability
    \end{itemize}
\end{itemize}

\begin{comment}
    
    The battery-electric drive-train has long been technically undebatably the most cost-efficient solution for decarbonizing the private passenger car sector [REF]. This statement is supported in multiple aspects: 
    \begin{itemize}
        \item environmental impact and energy demand [REF LCA]
        \item the total costs of ownership statistics
        \item trajectories in improvements and costs long-term \dots 
    \end{itemize}
    
    In its application (aside from the charging versus the refueling process), substituting the battery-electric vehicle has a limited impact on the mobility pattern, compared to other measures for reducing the carbon emissions of the transport sector. These other measures include shifting to other transport modes (

\end{comment}

\color{lightgray}
There are four main potential drivers to counteract increasing greenhouse gas emissions in the face of rising travel demand: avoidance of journeys, shift to low-carbon modes, fuel substitution, and improvements in energy efficiency \citep{Josefina, Edelenbosch2017DecomposingTransport, Milovanoff2021}. These drivers are deeply intertwined, and their effectiveness is rooted in technological progress, social learning, and infrastructure adoptions \citep{Edelenbosch2017DecomposingTransport, Muller2021}. \citet{PRIEMUS2001167} describes the close relationship between people's mobility behavior and spatial infrastructure planning, indicating that modal shift can only happen effectively as a response to changes in transport network infrastructure. Similarly, \citet{briggs2015automotive} finds that infrastructure planning decisions affect the modal split long-term, for over multiple decades, and can therefore also form a macro-level barrier to climate mitigation \citep{UNRUH2000817}. Infrastructure changes supporting climate-friendly mobility include the expansion of the railway system and construction of sidewalks and cycling routes, essentially aimed toward a reduction in motorized transport \citep{SONG2017320, Muller2021}. Other motivators in changing mobility behavior include monetary incentives, such as the adjustment of travel costs, costs of vehicle ownership or parking charges \citep{Hammadou2015, Silva2022}.

To decarbonize the passenger transport sector, the introduction of electromobility plays an important role \citep{HEIDRICH201717, Das2020}. Life cycle assessments revealed that electric vehicles emit less greenhouse gas emissions over their lifetime than internal-combustion-engine vehicles \citep{VanVliet2011, Bauer2015}. These assessments also highlight that the positive effect of electric vehicles is highly dependent on the shares of renewable energy sources in the energy mix. \citet{Costa2021} found that measures related to behavioral changes have overall similar potential to technological improvements in climate change mitigation. \citet{DILLMAN2021102614} argued similarly, finding that measures combining changes in travel behavior and technological improvements are the most effective in the decarbonization of passenger transport.

The level of income has been proven to be a significant predictor for the purchase of a battery-electric vehicle in many studies \citep{AUSTMANN2021101846}. The amount of disposable income influences the frequency of the acquisition of a new passenger car and whether it is purchased on the first or second-hand market \citep{STAJIC2023100658, VALENZUELALEVI2021161}. There exists a wide range of choice models that relate the decision of BEV purchase to socio-economic variables and reveal the influence of different factors, such as the income level, home ownership, household size, a person's attitude towards renewable technologies, and availability of charging infrastructure \citep{AUSTMANN2021101846, Liu2020}. \citet{GAZMEH2024104222} identify large disparities in charging infrastructure roll-out in different neighborhoods and find that higher-income communities have significantly better access to public charging infrastructure than lower-income communities. \citet{STEFANIEC2025101495} analyse such disparities for Ireland and conclude that, to effectively increase the adoption of BEVs in lower-income classes, prices of BEVs need to decrease and the deployment of public charging infrastructure in lower-income areas needs to be subsidized.
\color{black}

        \subsection{Role of charging infrastructure roll-out for passenger cars}
            Charging infrastructure is categorized by the power level, the use case, and access type [REF]. Home charging is typically at low voltage levels, providing charging directly to the parked car at the parking spot at home. This is currently the most popular way to charge as current users of BEVs are home owners [REF] and this provides the most convenient and cost-efficient way to charge [REF]. Often times, the BEVs at home chargers are integrated directly in the home energy managment system, co-optimized with the home PV electricity production, battery storage and/or other home appliances. Similar to home charging, \textit{work charging} offers charging at a daily regular parking place. This is highly limited though, and the provision of work charging depends on the 
            At public charging stations, consumers are currently faced with higher charging prices that are due to the provision of charging at higher power levels, but also the markup stems from uncertainties of future demand at the installed charging stations [REF].  
            
            Public charging plays here to role of \dots \hl{\textit{include some deeper meaning of this statement.}} TODO: find a study that addresses the significance of the charging infrastructure. Public charging infrastructure mirrors current fueling stations but is mostly allocated at parking places due to the longer refueling times for the vehicle. \hl{TODO: \textit{find convenient way to bridge the words fueling and charging, same principle but not same words}}.\\ 
            Public charging infrastructure is coarsely divided into \textit{slow} and \textit{fast} charging. 
            Public charging infrastructure is mostly allocated to provide (a) coverage, but also convenience (b). The convenience often corresponds to the overlap of charger timing with other activities, such as running errands, like shopping. This led to the allocation at \textit{points of interest}. \textit{Slow} charging provides here an opportunity to rechare during longer stays but also is an opportunity to charge during daily downtimes, during the night or day, providing a similar level of comfort  as home or work charging. \textit{Fast} charging provides the opportunity to recharge quickly, f.e, during errands, or for charging en-route. The efforts in developing fast chargers is to provide a similar refueling experience as the fueling at conventional fueling stations. \textit{Fast charging} \dots \hl{\textbf{TODO}: include here some references on the proven significance of fast charging to the bev usage! (not adoption)} 
            
            From the business case perspective, the challenge lies here to balance the investments and the competitiveness in charging prices with the uncertainty surrounding expected charging demand and growth of the demand at a charging site [REF]. This leads to increased pricing of charging at charging stations. 

            Simultaneously, there is an access gap in the home charging that correlates also with the income level of passenger car owners [REF]. Therefore, this group is espessially dependent on the access to the public access to charging infrastructure, and, with this, also facing higher charging prices than car owner using the home charger regularly. Simulteanously, charging points operators are more likely to allocate charging stations in communitites of high income, due to higher certainty in the adoption of BEVs and, therefore, charging demand \cite{STEFANIEC2025101495}. Therefore, the equity in charging stations allocation is an important factor, along with subsidization and other monetary-based incentives, to support the rapid adoption across all income classes \cite{HOPKINS2023113398}. 

            In scientific literature, the relationship effect between the local charging infrastructure development and the uptake of BEVs remains unclear. Range anxiety is directly related to the absence of charging infrastructure, and poses a psychological barrier to the adoption of BEVs \cite{RAINIERI202352}. \citet{Illmann2020} find an evident causality between the deployment of public charging infrastructure and the uptake of BEVs in German communities. Another finding of their study is that the uptake is more affected by the charging speed provided at the charging stations, rather than the number of chargers themselves. Similar is found in the work of \citet{WHITE2022102663} who conduct an analysis for three U.S. metropolitan areas. In their work, no direct impact of range anxiety on the BEV adoption is found. Their key finding is that higher spatial densities of public charging infrastructure increase the social perceived social pressure to adopt BEVs. 

            Beyond the local impact of charging infrastructure, empirical studies have also shown that the local expansion of public charging infrastructure also has an impact beyond the respective region \cite{QIAN2024104400, SELENASHENG2022103256}. The spatial spillover effect of transport-related infrastructures is common in transport infrastructure investment as it directly results from the network nature and interregional connectivity of good transport and travel \cite{Shevtsova31122025, SPORKMANN2023103851, MONTOLIO200999}. In the work of \citet{QIAN2024104400}, a retrospective analysis of the relationship between the charging infrastructure capacity within a region and the BEV adoption in neighboring regions was conducted. They find that, particularly in centrally allocated regions, fast chargers have a positive impact on the BEV adoption in neighboring regions. \citet{SELENASHENG2022103256} analyzes cross-regional impacts in New Zealand. A critical finding is that empirical analysis reveals that public charging infrastructure in neighboring areas has a more significant impact on local BEV adoption than locally installed charging infrastructure. Both these papers, by \citet{SELENASHENG2022103256} and \citet{QIAN2024104400}, emphasize in their conclusion that charging infrastructure planning needs to be under the consideration of a spatially holistic perspective, i.e., across borders, to effectively support battery-electric vehicle adoption. 

            

            

            
        \textbf{Lead questions:}
        \begin{itemize}
            \item \textit{What types of charging infrastructure exist?}
            \item \textit{What role does each of these do play? What do they provide (f.e. convenience, en-route, public charging etc.)}
            \item \textit{Home charging access gap}
            \item \textit{Grid connection as a constraint!} --- what role does this play here? does it play a vital role? 
            \item \textit{why is it now so difficult to allocate charging infrastructure?}
            \item \textit{first: where is the demand? (This is already part of the paragraph on the role of each of the charging stations)}
            \item \textit{Bridge to allocation of the infrastructure}
            \item \textit{spatial layouts of charging infrastructure: WHERE? pois, housing areas, parking spaces,} 
            \item  \textit{what about cross-border impacts?}
            \item \textit{and spatial density of charging station availability?}
            
            \item \textbf{Main outcomes should be:}
            \begin{itemize}
                \item fast charging is reducing range anxiety (paper 1)
                \item equitable distribution of public charging is lacking (paper 2)
            \end{itemize}
        \end{itemize}
\color{lightgray}
The term \textit{fast-charging} generally encompasses AC charging at capacities higher than 22~kW and DC charging at high power levels \citep{Falvo2014}. These chargers are found within the public charging infrastructure \citep{Kotzur2021}. The expansion and densification of the public charging infrastructure have been used as important incentives to increase the adoption of electric vehicles in many countries \citep{Parliament2018, Schulz2022}. However, it should be noted that requirements for public charging infrastructure significantly vary between different countries and regions \citep{Arpad2019, Coffman2017, Parliament2018, Illmann2020}. In an extensive literature review on motivators and barriers for electro-mobility adoption in Europe, \citet{Efe2018} found that the lack of public infrastructure poses a barrier to widespread battery-electric vehicle usage. Moreover, the availability of public charging allows for longer trip lengths and reduces the phenomenon of range anxiety \citep{NEUBAUER201412, Neaimeh2017}. Studies suggest that with increasing adoption, the importance of sufficient public charging infrastructure will increase as more vehicles will be owned by people without private parking \citep{Arpad2019, Engel2018}.

\citet{Illmann2020} found that charging speed plays a more important role than the density of public charging infrastructure in adoption decisions, with consumers preferring fewer fast chargers over many slow-charging stations. Similar results were obtained by \citet{Globisch2019}, who found that lower charging duration is valued over higher spatial density and lower charging fees. \citet{Gebauer2016} demonstrated that potential users perceive the future of electromobility more positively in the presence of more widespread DC fast-charging.

Among the most important locations of fast chargers are service areas along highways \citep{Philipsen2020}. Densification of fast-charging infrastructure along highways is aimed for by means of tendering processes that motivate parking place owners to install charging stations \citep{deutschlandnetz}. Such measures are imposed to tackle the \enquote{chicken-egg} conundrum between potential buyers needing charging infrastructure availability and private businesses reluctant to build infrastructure due to lack of demand \citep{6915043}. \citet{Jochem2019} argued that running fast-charging infrastructure along main highway corridors will become profitable due to high demand and willingness to pay. A core challenge in deploying fast-charging stations is the need for sufficient connection to the power grid, which can heavily increase capital expenditures \citep{Fernandez2019, Serradilla2017}. Studies have suggested how limitations imposed by grid connection can be addressed through battery storage systems, on-site electricity generation, price signals and scheduled charging \citep{Gusrialdi2017, Chen2016, Fernandez2019, Dong2018}.

In scientific literature, the particular impact of public charging infrastructure on the decision to purchase a BEV has not been fully clear \citep{WHITE2022102663}. There exists evidence of range anxiety, which relates directly to the fear of the potential lack of available charging infrastructure \citep{RAINIERI202352}. Moreover, there exists an evident correlation between the available public charging infrastructure capacities and the share of BEVs in the passenger car fleet within a given area \citep{HAUSTEIN2021112096}. The analysis presented in \citep{ILLMANN2020102413} reveals positive causality between charging infrastructure and BEV adoption that persists in the long run. \citet{WHITE2022102663} conduct a similar analysis for three metropolitan regions in the U.S. and find that the most significant impact that public charging infrastructure has is on social norms. The choice of mode for passengers is evidently influenced by the travel time, which can be monetized by measuring how time is subjectively perceived by a consumer group \citep{TATTINI2018265}. The charging time is often reported to be a barrier to BEV adoption and therefore levers for BEV adoption include technical improvements in charging speed and convenient allocation of charging infrastructure \citep{BERKELEY2017320, PAMIDIMUKKALA2024100153}.

Concerning equity in infrastructure roll-out, \citet{HOPKINS2023113398} give an overview of policy packages for enabling the equitable roll-out of public charging infrastructure. Key levers for increasing BEV adoption among lower-income classes include the availability of second-hand BEVs together with supporting grants, and the allocation of public charging infrastructure with affordable tariffs close to homes. \citet{Bhatt_2024} design a framework for assessing equity in public charging infrastructure based on the proximity of charging stations to homes of BEV owners. One significant reason for disparities in infrastructure availability is the lack of charging demand stemming from the lack of BEV ownership in lower-income areas, which makes the deployment of charging infrastructure there not profitable for charging point operators \citep{HOPKINS2023113398}.

There is evidence of the spatial spillover effect of transport infrastructure investments on economic activity beyond the regions where these investments are made \citep{Shevtsova31122025}. These effects occur mainly due to the network nature and inter-regional connectivity. \citet{SPORKMANN2023103851} analyze the relation between national transport infrastructure capacities in European countries and carbon emissions in neighboring countries. In terms of long-term BEV adoption, studies have found significant interaction between the roll-out of public charging infrastructure and adoption in neighboring regions. \citet{QIAN2024104400} analyzes this relationship between neighboring US States and finds that the accessibility of slow chargers has a significant impact on adoption in neighboring States, particularly due to fast charger deployment. \citet{SELENASHENG2022103256} conduct a similar empirical analysis in New Zealand, finding that public charging infrastructure in neighboring areas has a more significant impact on local BEV adoption than locally installed infrastructure. 

\color{black}
    \section{Battery-electric vehicle adoption in commercial fleets}
        \subsection{Electrification across commercial fleet segments}
        \textit{Somewhere I should include the definition of commercial fleets -- mabye in the beginning of the text? s}
        The core difference to the passenger sector is cost-efficiency as the core driving force (compared to social aspects and personal freedom aspect in the private passenger car ownership).
        The three aforementioned decarbonization levers in the passenger transport sector also apply to the commercial transport sector --- reducing demand, shifting demand towards low-carbon modes and replacing conventional driver-trains and substituting fuels [REF]. In regard to the commercial road transport sector, this translates to two concrete changes that drive the carbon emission reduction. The option of alternative modes is mostly limited to rail. Though the shift is substantially limited by (1) the cost-competitiveness starting at longer ranges, (2) the bottleneck of the rail infrastructure transport capacity with simultaneous high investment needed to add capacities, and (3) the flexibility in the vehicle is very restricted, due to the limiting track-based application and the last-mile problem remains. This is mainly discussed and applicable for \textit{long-haul} applications. 

        Therefore, similar as in the private passenger fleet, the direct electrification of drive-trains is the primary way to decarbonize-
        \textit{maybe restructure here so that main lever = electrification/technology substitution; then other two }
        
        \footnote{An important lever to mention is here also the reduction of transport demand through increased rationalization of production and industry. Though, in general, the trend concerning this is rather negative. Therefore, it is not frequently discussed as a forefront solution.}

        Though other than in the private passenger car fleet, the vehicle segments are significantly more diverse, specifically in the daily ranges and the transported weights in the application. This has also led to the significance of other fuels and drive-train technologies other than the battery-electric drivetrain, in the technology substitution.
        While, for light-duty applications, the battery-electric vehicles have been, similarly as in the passenger car sector, the clear solution for reducing fossil fuel consumption, for heavy-duty applications, the picture has been significantly less clear. Zero-emission options narrow down to the battery-electric vehicle and hydrogen fuel-cell electric vehicles. While, in the general perspective, the battery-electric vehicle technology significantly exceeds in the cost-effectiveness, these costs increase substantially for long-haul heavy-duty applications, primarily due to payload losses cause by heavy batteries required to overcome the long-haul ranges and transport heavy goods \cite{MARTIN2023100156}. In the work of \citet{NOLL2022118079}, the total costs of ownership for different drive-trains of trucks of different segments are analysed for the different European countries. The authors find that for light- and medium-duty applications, the cost-benefits of the battery-electric drive-trains are clear. For long-haul heavy-duty applications, the cost competitiveness varies substantially by country. The driver for these lie primarily in the operational expenditures (OPEX), the country-dependent tolls and the relative difference in fuel prices, i.e., between diesel and electricity prices. 
        Hydrogen fuel-cell electric trucks have also been pushed in the policy making [\hl{where exactly?} AFIR?] but the recent developments in battery-electric drive-train technology have increased the advantages in the application of battery-electric vehicles [REF], essentially making this option substantially more cost-efficient. The main reasons for this are discussed in the publication by \citet{Plötz2022}: \hl{\dots}. 
        The core message of this paper is that the core challenges of implementing battery-electric trucks for long-haul applications, specifically the long charging times and limited range, have been overcome. Against this background, the article challenges the need for using hydrogen fuel cell electric trucks, and concludes that \dots \textit{not sure about this statement yet, if this is needed}

        

        
        \textbf{Lead questions}
        \begin{itemize}
            \item \textit{What is the commercial sector? }
            \item \textit{What are the driving forces here? (parallel to the social component --- here its more cost-efficiency-driven --- what are these costs driven by? fuel costs.., drivers, --\dots)}
            \item \textit{Acknowledgment of fuel cell electric vehicles?}
            \item \textit{What are the technical difficulties and how are these different by vehicle segment?}
            \item \textit{What changes or impacts the logistic process through the electrification? tonnage, charging activity, safety}
            \item \textit{What is the risk through electrification? (how does it impact the cost-efficient logistic process? }
            \item \textbf{Main outcomes}
            \begin{itemize}
                \item The fleet development is cost-driven?
                \item range and weight are difficult 
                \item charging infrastructure placement is important here to align with logistic processes and provide cost-efficient charging 
            \end{itemize}
            \item \textit{other fuels only play a role under assumed constraints on the availability or possibility of installing charging stations, so, f.e., no possibility to install charging station (or temperature related concerns)} but these concerns are quickly invalidated, f.e. bev application in nordic country; and are not relevant at large scale; mostly concerned with that the electric grid expansion and elec system will not be ready for it 
        \end{itemize}
\color{lightgray}
For the decarbonization in the transport sector, new technologies as well as the shift to different modes are important levers discussed in the Smart Mobility Strategy by the EU \citep{eu_mobility_strategy}. In the road transport sector, the direct electrification of vehicle drive-trains is supported by literature to be the most cost-effective and efficient measure to defossilize the sector on a large scale \citep{Auer2020, Plötz2022}. The introduction of this new technology is connected to requirements in structural changes, such as the deployment of charging infrastructure, as well as changes in logistic processes and travel behavior \citep{LEBROUHI2021103273}. This is particularly a challenge for commercial vehicles, such as trucks and buses \citep{KONSTANTINOU2023100746}. This electrification comes with a significant additional electricity demand but also provides demand-side flexibility \citep{ag2022flexibility}.

Currently, the electrification of the commercial fleet is in significantly earlier stages than the private passenger vehicle sector. One early and significant adopter in the commercial fleet sector is the light-duty truck segment \citep{statistik_at_kfz_bestand}. Other commercial vehicle types that are used for heavy duty require bigger battery capacities due to higher specific energy consumption and longer trip ranges. To overcome this, overhead lines have been used to power electric buses; trolley buses have been in operation in urban applications in some European cities since around the mid of last century \citep{ietmeishner}. Due to difficulties in the social acceptance of installing new overhead lines, the adoption of electric buses that are exclusively battery-powered is becoming the most common in the electrification of city bus fleets \citep{orf2023}. Moreover, short-haul applications of heavy-duty trucks are projected to be adopted soon \citep{NOLL2022118079}.
\color{black}
        \subsection{Charging infrastructure for battery-electric trucks}
        In the switch to battery-electric vehicles in the commercial sector, the integration of the time-consuming charging activities into operational schedules, business practices, and logistics processes is a particular challenge. A further factor in integrating the charging activities is the cost-optimal charging. The most cost-effective way is at low power levels (up until 100kW) overnight at depots and home bases of fleets. For this, fleet operators need to invest in on-site reinforcement of grid connection capacity and investment in charging infrastructure \cite{Borlaug2021} [REF some paper that plans charging infrastructure for a depot]. With additional on-site renewable electricity production and smart charging, costs for charging can be even more decreased [REF].  

        Opportunity charging at depots during shift changes of (un-)loading activities offer complementary charging opportunities [REF] \textit{specify here which vehicle semgnets\dots } \textit{\hl{busses should be also somewhere included}}\\

        Public charging infrastructure play a supporting role in \dots \textit{Write here on the role of public charging infrastructure }\\

        In the electrification of long-haul trucking, charging opportunities along highway networks play an important role. Both, slow charging and fast charging support the integration to the transport route by providing charging during the driver's obligatory break times en-route [REF]. The fast charging opportunities are for charging during the obligatory breaks of 45 min. It is typically at high power levels. Currently charging at around 350kW is offered, soon trucks will be able to charge at 1 MW - which is also called megawatt charging [REF]. Slow charging is for longer, overnight breaks. Due to the sufficient range of BEVs [REF Plötz], this charging opportunities can be sufficiently integrated into logistics [REF].

        

        % Several studies have shown that at high shares of battery-electric vehicles in the commercial fleet lead to increased investments needs to the electricity grid, and generation and storage capacities [REF see paper vof here]. 
        \textit{\hl{still need to think here of how to integrate here the large-scale perspective of system impacts}}
        
        

        
        
        \textbf{Lead questions:}
        \begin{itemize}
            \item \textit{What categories of charging infrastructure do we have? which commercial fleet segments do this cater to?}
            \item \textit{What power levels do we have here?}
            \item \textit{How quickly can these charges?}
            \item \textit{How far do these also may impact logistic processes?}
            \item \textit{What is megawatt charging?}
            \item \textit{What are the dangers or opportunities here for the electricity system? (difference to passenger cars?) }
            \item \textbf{Main outcome:}
            \begin{itemize}
                \item slow charging the preferred way which is convient for mobility patterns that have daily range small than the technically possible range 
                \item en-route charging at depots or along highway stations required; 
                \item particular challenge here is: the provision of sufficient infrastructure, to avoid congestion and prolong transport time; 
                \item therefore it needs to be enroute!
                \item en-route: using EU regulation hours with \textit{charging windows}
            \end{itemize}
        \end{itemize}
        \color{lightgray}
\citet{Scherrer2024Requirements} describe existing barriers to the technology switch to battery-powered medium- and heavy-duty trucks for fleet operators in Germany. The major barriers refer to the challenges in the deployment of high-power-level charging infrastructure and the difficulty of adopting operational patterns to the need for being flexible in the timing and location of the charging process. Given this, extensive research is dedicated to the optimal integration of the charging process in operational schedules. This encompasses the geographic allocation of charging stations and the determination of optimal timing and position of the charging process, as well as the optimization of the charging load curve during the plug-in time \citep{Shoman2023Public, 7947231, KARMALI2024101397}. The overall preferred charging strategy with regards to avoiding major alterations of current operational schedules is charging during downtime, i.e., outside of operation times, during night time or during weekend days, or during obligatory rest times of the driver \citep{teoh2022, Shoman2023Public}. In this context, the build-up of charging capacity at depots and megawatt charging points along the route, particularly for long-haul applications, is vital \citep{SPETH2024104078}.

Particularly for slow-charging processes, the power level during the charging process is of interest and is aimed to be coordinated cost-efficiently. On the one hand, continuously charging at low power levels is a lower risk for battery degradation, and investments for the local grid reinforcements and charging capacity can be saved \citep{Borlaug2021}. On the other hand, a variable power level throughout the charging process can be utilized to coordinate with volatile day-ahead electricity prices and to participate in balancing and congestion markets \citep{ag2022flexibility, MANZOLLI2022124252}.
        \color{black}
        % \subsection{Charging patterns and the utilization of different charging infrastructure types}

        % \subsection{Impact of large-scale charging loads on the electricity system}
        
        
            
        
    \section{Modeling of charging demand, charging infrastructure capacities, and roll-out}
        \subsection{Graph-based charging infrastructure allocation}
        \cite{Metais2022}: review on the different models; \cite{Pagany2019}: also review; \cite{li2024investigating}: is also a review; going into different stake-holder perspectives 
        \cite{Wang2025SpatiotemporalPO}: Spatiotemporal planning of electric vehicle charging infrastructure: Demand estimation and grid-aware optimization under uncertainty
        \cite{GHANBARIMOTLAGH2026119325}: another review (2025) 
\color{lightgray}
A graph representation of a street network is frequently used for the allocation of charging stations as it allows to reflect high spatial resolution in modeling while simplifying complex networks without suffering from information loss \citep{Metais2022, Pagany2019}. Edges typically represent connections between nodes, and nodes represent junctions or ends of edges. \citet{Metais2022} differentiated between node-based and flow-based approaches. In node-based approaches, charging demand is assigned to nodes based on population density and the goal is to allocate charging infrastructure such that as much demand as possible is covered at nodes. Contrary, flow-based approaches aim to allocate charging stations along edges through which the highest vehicle flows occur. This approach requires origin-destination data describing the traffic load between all nodes in the network, which is not always available \citep{Ghamami2016}. \citet{Metais2022} concluded that both approaches offer benefits: node-based approaches offer input data simplicity and the ability to regard charging capacity, while flow-based approaches succeed in reflecting the nature of traffic flow. Furthermore, the authors stated that the combination of both approaches may yield the most robust charging station allocation model.

Overall, the proposed methodologies encompass mostly extensions of the basic formulations of node- and flow-based approaches while using various optimization techniques, such as genetic algorithm, particle swarm optimization and integer programming, as well as iterative algorithms \citep{SHAREEF2016403}. Most commonly, objective functions are formulated as minimization of charging infrastructure installation costs and driver expenses, maximization of charged vehicles, and minimization of failed trips. \citet{6183279} proposed an allocation model where nodes represent service area positions, and optimized allocation using integer programming to minimize total costs of installation, recharging, and transportation. \citet{Wang2018} optimized highway charging infrastructure allocation with regard to minimizing charging time, queuing, driver costs and deployment costs using a multistage equilibrium-model with origin-destination data. \citet{Jochem2019} used the flow-based approach, minimized the number of charging points in the network, and conducted calculations on sizing based on flow coverage. \citet{7575934} also followed a flow-based approach and proposed a two-stage genetic algorithm with multi-objective formulation to minimize construction costs and charging costs. \citet{Napoli2020} developed an iterative allocation algorithm correlating the distance between charging stations with the driving range of vehicles. \citet{Csiszar2020} designed a sequential selection of positions based on traffic volume and nearby settlement population.

In these studies, driving range was commonly used as an indicator for maximum distance between charging stations. \citet{Csiszar2020} found that with increasing range, the number of serviced vehicles by a given charging network increases. \citet{Kavianipour2021} analyzed the effect of increasing driving range and charging power, indicating that with technological developments, required investment costs sink as fewer charging points and less charging capacity need to be installed. \citet{WANG20181255} analyzed optimal deployment under a fixed budget and observed a saturation of flows at a certain driving range, indicating that under a given budget, higher ranges would not affect infrastructure requirements.
\color{black}
        \subsection{Long-term optimal charging infrastructure deployment for different consumer segments for passenger car}
\color{lightgray}
In charging infrastructure roll-out planning, charging stations are placed with the objective of cost-optimality and meeting the charging demand \citep{METAIS2022111719}. Research analyzing the optimal expansion together with its impact on BEV adoption can be categorized into three strands: first, system dynamics models which capture a high level of complexity of the interaction of different components \citep{9706486, GAO2024103969}; second, agent-based models with high spatial resolution in the allocation of charging stations and mobility behavior \citep{LIAO2023103645, WOLBERTUS2021262}; third, large-scale optimization approaches that aim to map cost-optimal decarbonization pathways for the transport sector \citep{TATTINI2018265}. \citet{WOLBERTUS2021262} find that single distributed roll-out strategy has a positive effect on BEV adoption, while hub roll-out scenarios show significantly smaller impact.

Concerning the third category, there are many studies dedicated to compensating for the shortcomings of the total-cost-minimization approach \citep{TATTINI2018265, PATIL2023104265}. \citet{Venturini2019} describe different tools to introduce behavioral aspects, concluding that effective tools are segmentation of consumers and inclusion of intangible costs. \citet{luh2022behavior} introduce segmentation of consumer groups by housing type and include different types of charging infrastructure, finding that the option to charge at home has a substantial impact on BEV adoption, while public slow and fast charging play a vital role when private charging is limited. A follow-up publication \citep{LUH2024122412} includes the value of travel time, which also includes time to recharge and detour time to reach the closest charging station.

In the modeling of optimal public charging infrastructure roll-out, there exist only a few studies that explicitly consider income class disparities: \citet{BATIC2025135834} develop a graph neural network to plan public charging infrastructure deployment in urban areas on street grid level, while aiming to reduce spatial inequity in charging infrastructure availability. \citet{LONI2023128796} introduce the maximization of social equity access to a multi-objective charging infrastructure planning optimization model at the neighborhood level.
\color{black}
        \subsection{Charging infrastructure planning for international long-haul road freight transport}

        
\color{lightgray}
The reduction of fossil fuel dependency in the road freight transport sector has been challenged by its significant growth in service demand and the current lack of uptake of zero-emission technologies \citep{EEA_RoadTransport}. The \textit{Sustainable \& Smart Mobility Strategy} by the European Commission describes the goal to shift road freight transport demand towards zero-emission drivetrain technologies by 2050 \citep{EU_MobilityStrategy}. In recent years, many empirical and theoretical studies have showcased the applicability of battery-electric trucks (BET) for this purpose \citep{Plötz2022}. The large-scale adoption of BETs is supported by continuously declining costs of ownership \citep{MAGNINO2024113215, NOLL2022118079} and technological improvements in battery technology leading to the overcoming of often-disputed limited range and charging speed \citep{AGUILAR2022102624}.

One central challenge to enable the widespread adoption of BET is the build-up of a dense network of charging infrastructure \citep{MULLER2025104455}. The \textit{Alternative Fuels Infrastructure Regulation (AFIR)} explicitly mandates the build-up of mega-watt charging stations every 120~km along the Trans-European Transport network (TEN-T) \citep{EU_AFIR2023}. The high power levels connected to charging infrastructure for trucks pose a particular challenge due to the high volumes of simultaneous electricity loads that require substantial investments in grid infrastructure at distribution grid level \citep{ZHAO2025309}. A wide range of studies has been dedicated to exploring grid-friendly charging and using the flexibility potential of trucks for power grid congestion on large and small scales \citep{GOLAB2025134385, KARMALI2024101397}.

\citet{hildermeier2024power} have compared different border regions in the EU, which vary significantly in the costs of installing charging infrastructure and the charging activity due to different electricity market zones, network connection fees, and grid fees. The cost-optimal fueling of trucks has been observed to be driven mainly by different taxation laws between countries \citep{bmik2021mobility}. The country-dependent charging costs \citep{Lanz2022} are likely to shift the cost-optimal allocation of the fueling/charging activity, potentially allocating charging capacities to unfavorable locations in the electricity system.
\color{black}
        \subsection{Sub-annual modeling of charging demand for commercial fleets at large scales}
 \cite{ANADONMARTINEZ2025104547}: spatio-temporal demand estimation at highway using traffic flow 
    \section{Contribution to the progress beyond state-of-the-art}

    Against this background, the novelties are comprised in the following chapter by addressing the research question.

    \textbf{Research question 1:}
    \begin{itemize}
        \item A hybrid charging station allocation problem is formulated for highway networks that combines the node-based optimization structure with path-based demand representation. The charging demand is defined at discrete nodes in a graph network, while we, simultaneously, consider the flow of vehicles as in the path-based approach. Charging demand is assigned to descrete nodes. Demand shifting between nodes based on traffic direction and driving range capture vehicle flow and flexibility in the demand allocation. This approach is developed in order to use granular data on service area positions and traffic flow counts along the highway network.
        \item Fast-charging infrastructure capacities are planned for the Austrian highway network for 2030 under different scenarios of BEV shares in the passenger car fleet, road traffic load, BEV driving range, and charging power levels. Extensive sensitivity analyses identify impactful parameters to charging capacity requirements. The scenarios address uncertainties in demand resulting from the different transport sector decarbonization pathways.
    \end{itemize}
    % TODO: Modeling long-term charging infrastructure roll-out
    \textbf{Research question 2:}
    \begin{itemize}
        \item Public charging capacities are planned under the consideration of the impact on BEV adoption across consumer groups of different income levels at the federal and state levels. This is in contrast to existing literature, in which the focus has lain on a lower geographic scope, i.e., municipality and neighborhood level. With this, we bridge the gap of addressing the impact of charging infrastructure pathways on federal states level on various income classes. 
        \item The detour fueling time, i.e, the extra walking or driving time to reach the nearest charging station, is used to link the impact of infrastructure roll-out to the BEV adoption. Thereby, exhaustive analysis based on conventional high-resolution geographic data is avoided, while the densification of public charging infrastructure is still represented and connected to the BEV adoption.
        \item We analyze cross-border impacts between the local charging infrastructure expansion and the BEV adoption in neigbouring countries within a techno-economic optimization. Until now, the potential spillover of charging infrastructure between regions has been exclusively investigated based on empirical data, neglecting the future techno-economic potential of spillover by ambitious local charging infrastructure expansion. \textit{not 100\% happy with this formulation} 
    \end{itemize}


    \textbf{Research question 3:}

    \begin{itemize}
        \item Quantifying impact of electricity zone-differing electricy prices on the adoption of battery-electric trucks 
        
        \item Flexible charging load allocation with BEV adoption
        \item Modal shift the high resolution planning following the european goals of considering this for decarbonization\dots ?  
    \end{itemize}


    \textbf{Research question 4:}

    \begin{itemize}
        \item quantifying charging demand based on origin-destination data
        \item quantifying the flexibility demand for Austria 2050 for utilization in the redispatch
    \end{itemize}