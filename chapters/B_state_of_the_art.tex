\chapter{Review of State-of-the-Art \& Progress Beyond}
    \section{Electrification of passenger cars in the context of energy system decarbonization}
        \subsection{Towards large-scale electrification of passenger car fleets}

There are four main potential drivers to counteract increasing greenhouse gas emissions in the face of rising travel demand: avoidance of journeys, shift to low-carbon modes, fuel substitution, and improvements in energy efficiency \citep{Josefina, Edelenbosch2017DecomposingTransport, Milovanoff2021}. These drivers are deeply intertwined, and their effectiveness is rooted in technological progress, social learning, and infrastructure adoptions \citep{Edelenbosch2017DecomposingTransport, Muller2021}. \citet{PRIEMUS2001167} describes the close relationship between people's mobility behavior and spatial infrastructure planning, indicating that modal shift can only happen effectively as a response to changes in transport network infrastructure. Similarly, \citet{briggs2015automotive} finds that infrastructure planning decisions affect the modal split long-term, for over multiple decades, and can therefore also form a macro-level barrier to climate mitigation \citep{UNRUH2000817}. Infrastructure changes supporting climate-friendly mobility include the expansion of the railway system and construction of sidewalks and cycling routes, essentially aimed toward a reduction in motorized transport \citep{SONG2017320, Muller2021}. Other motivators in changing mobility behavior include monetary incentives, such as the adjustment of travel costs, costs of vehicle ownership or parking charges \citep{Hammadou2015, Silva2022}.

To decarbonize the passenger transport sector, the introduction of electromobility plays an important role \citep{HEIDRICH201717, Das2020}. Life cycle assessments revealed that electric vehicles emit less greenhouse gas emissions over their lifetime than internal-combustion-engine vehicles \citep{VanVliet2011, Bauer2015}. These assessments also highlight that the positive effect of electric vehicles is highly dependent on the shares of renewable energy sources in the energy mix. \citet{Costa2021} found that measures related to behavioral changes have overall similar potential to technological improvements in climate change mitigation. \citet{DILLMAN2021102614} argued similarly, finding that measures combining changes in travel behavior and technological improvements are the most effective in the decarbonization of passenger transport.

The level of income has been proven to be a significant predictor for the purchase of a battery-electric vehicle in many studies \citep{AUSTMANN2021101846}. The amount of disposable income influences the frequency of the acquisition of a new passenger car and whether it is purchased on the first or second-hand market \citep{STAJIC2023100658, VALENZUELALEVI2021161}. There exists a wide range of choice models that relate the decision of BEV purchase to socio-economic variables and reveal the influence of different factors, such as the income level, home ownership, household size, a person's attitude towards renewable technologies, and availability of charging infrastructure \citep{AUSTMANN2021101846, Liu2020}. \citet{GAZMEH2024104222} identify large disparities in charging infrastructure roll-out in different neighborhoods and find that higher-income communities have significantly better access to public charging infrastructure than lower-income communities. \citet{STEFANIEC2025101495} analyse such disparities for Ireland and conclude that, to effectively increase the adoption of BEVs in lower-income classes, prices of BEVs need to decrease and the deployment of public charging infrastructure in lower-income areas needs to be subsidized.

        \subsection{Role of charging infrastructure roll-out for passenger cars}

The term \textit{fast-charging} generally encompasses AC charging at capacities higher than 22~kW and DC charging at high power levels \citep{Falvo2014}. These chargers are found within the public charging infrastructure \citep{Kotzur2021}. The expansion and densification of the public charging infrastructure have been used as important incentives to increase the adoption of electric vehicles in many countries \citep{Parliament2018, Schulz2022}. However, it should be noted that requirements for public charging infrastructure significantly vary between different countries and regions \citep{Arpad2019, Coffman2017, Parliament2018, Illmann2020}. In an extensive literature review on motivators and barriers for electro-mobility adoption in Europe, \citet{Efe2018} found that the lack of public infrastructure poses a barrier to widespread battery-electric vehicle usage. Moreover, the availability of public charging allows for longer trip lengths and reduces the phenomenon of range anxiety \citep{NEUBAUER201412, Neaimeh2017}. Studies suggest that with increasing adoption, the importance of sufficient public charging infrastructure will increase as more vehicles will be owned by people without private parking \citep{Arpad2019, Engel2018}.

\citet{Illmann2020} found that charging speed plays a more important role than the density of public charging infrastructure in adoption decisions, with consumers preferring fewer fast chargers over many slow-charging stations. Similar results were obtained by \citet{Globisch2019}, who found that lower charging duration is valued over higher spatial density and lower charging fees. \citet{Gebauer2016} demonstrated that potential users perceive the future of electromobility more positively in the presence of more widespread DC fast-charging.

Among the most important locations of fast chargers are service areas along highways \citep{Philipsen2020}. Densification of fast-charging infrastructure along highways is aimed for by means of tendering processes that motivate parking place owners to install charging stations \citep{deutschlandnetz}. Such measures are imposed to tackle the \enquote{chicken-egg} conundrum between potential buyers needing charging infrastructure availability and private businesses reluctant to build infrastructure due to lack of demand \citep{6915043}. \citet{Jochem2019} argued that running fast-charging infrastructure along main highway corridors will become profitable due to high demand and willingness to pay. A core challenge in deploying fast-charging stations is the need for sufficient connection to the power grid, which can heavily increase capital expenditures \citep{Fernandez2019, Serradilla2017}. Studies have suggested how limitations imposed by grid connection can be addressed through battery storage systems, on-site electricity generation, price signals and scheduled charging \citep{Gusrialdi2017, Chen2016, Fernandez2019, Dong2018}.

In scientific literature, the particular impact of public charging infrastructure on the decision to purchase a BEV has not been fully clear \citep{WHITE2022102663}. There exists evidence of range anxiety, which relates directly to the fear of the potential lack of available charging infrastructure \citep{RAINIERI202352}. Moreover, there exists an evident correlation between the available public charging infrastructure capacities and the share of BEVs in the passenger car fleet within a given area \citep{HAUSTEIN2021112096}. The analysis presented in \citep{ILLMANN2020102413} reveals positive causality between charging infrastructure and BEV adoption that persists in the long run. \citet{WHITE2022102663} conduct a similar analysis for three metropolitan regions in the U.S. and find that the most significant impact that public charging infrastructure has is on social norms. The choice of mode for passengers is evidently influenced by the travel time, which can be monetized by measuring how time is subjectively perceived by a consumer group \citep{TATTINI2018265}. The charging time is often reported to be a barrier to BEV adoption and therefore levers for BEV adoption include technical improvements in charging speed and convenient allocation of charging infrastructure \citep{BERKELEY2017320, PAMIDIMUKKALA2024100153}.

Concerning equity in infrastructure roll-out, \citet{HOPKINS2023113398} give an overview of policy packages for enabling the equitable roll-out of public charging infrastructure. Key levers for increasing BEV adoption among lower-income classes include the availability of second-hand BEVs together with supporting grants, and the allocation of public charging infrastructure with affordable tariffs close to homes. \citet{Bhatt_2024} design a framework for assessing equity in public charging infrastructure based on the proximity of charging stations to homes of BEV owners. One significant reason for disparities in infrastructure availability is the lack of charging demand stemming from the lack of BEV ownership in lower-income areas, which makes the deployment of charging infrastructure there not profitable for charging point operators \citep{HOPKINS2023113398}.

There is evidence of the spatial spillover effect of transport infrastructure investments on economic activity beyond the regions where these investments are made \citep{Shevtsova31122025}. These effects occur mainly due to the network nature and interregional connectivity. \citet{SPORKMANN2023103851} analyze the relation between national transport infrastructure capacities in European countries and carbon emissions in neighboring countries. In terms of long-term BEV adoption, studies have found significant interaction between the rollout of public charging infrastructure and adoption in neighboring regions. \citet{QIAN2024104400} analyze this relationship between neighboring US States and find that accessibility of slow chargers has a significant impact on adoption in neighboring states, particularly attributed to fast charger deployment. \citet{SELENASHENG2022103256} conduct a similar empirical analysis for New Zealand, finding that public charging infrastructure in neighboring areas has a more significant impact on local BEV adoption than infrastructure installed locally.

    \section{Battery-electric vehicle adoption in commercial fleets}
        \subsection{Electrification across commercial fleet segments}

For the decarbonization in the transport sector, new technologies as well as the shift to different modes are important levers discussed in the Smart Mobility Strategy by the EU \citep{eu_mobility_strategy}. In the road transport sector, the direct electrification of vehicle drive-trains is supported by literature to be the most cost-effective and efficient measure to defossilize the sector on a large scale \citep{Auer2020, Plötz2022}. The introduction of this new technology is connected to requirements in structural changes, such as the deployment of charging infrastructure, as well as changes in logistic processes and travel behavior \citep{LEBROUHI2021103273}. This is particularly a challenge for commercial vehicles, such as trucks and buses \citep{KONSTANTINOU2023100746}. This electrification comes with a significant additional electricity demand but also provides demand-side flexibility \citep{ag2022flexibility}.

Currently, the electrification of the commercial fleet is in significantly earlier stages than the private passenger vehicle sector. One early and significant adopter in the commercial fleet sector is the light-duty truck segment \citep{statistik_at_kfz_bestand}. Other commercial vehicle types that are used for heavy duty require bigger battery capacities due to higher specific energy consumption and longer trip ranges. To overcome this, overhead lines have been used to power electric buses; trolley buses have been in operation in urban applications in some European cities since around the mid of last century \citep{ietmeishner}. Due to difficulties in the social acceptance of installing new overhead lines, the adoption of electric buses that are exclusively battery-powered is becoming the most common in the electrification of city bus fleets \citep{orf2023}. Moreover, short-haul applications of heavy-duty trucks are projected to be adopted soon \citep{NOLL2022118079}.

        \subsection{Charging infrastructure for battery-electric trucks}

\citet{Scherrer2024Requirements} describe existing barriers to the technology switch to battery-powered medium- and heavy-duty trucks for fleet operators in Germany. The major barriers refer to the challenges in the deployment of high-power-level charging infrastructure and the difficulty of adopting operational patterns to the need for being flexible in the timing and location of the charging process. Given this, extensive research is dedicated to the optimal integration of the charging process in operational schedules. This encompasses the geographic allocation of charging stations and the determination of optimal timing and position of the charging process, as well as the optimization of the charging load curve during the plug-in time \citep{Shoman2023Public, 7947231, KARMALI2024101397}. The overall preferred charging strategy with regards to avoiding major alterations of current operational schedules is charging during downtime, i.e., outside of operation times, during night time or during weekend days, or during obligatory rest times of the driver \citep{teoh2022, Shoman2023Public}. In this context, the build-up of charging capacity at depots and megawatt charging places along the route, in particular for long-haul applications, is vital \citep{SPETH2024104078}.

Particularly for slow-charging processes, the power level during the charging process is of interest and is aimed to be coordinated cost-efficiently. On the one hand, continuously charging at low power levels is a lower risk for battery degradation, and investments for the local grid reinforcements and charging capacity can be saved \citep{Borlaug2021}. On the other hand, a variable power level throughout the charging process can be utilized to coordinate with volatile day-ahead electricity prices and to participate in balancing and congestion markets \citep{ag2022flexibility, MANZOLLI2022124252}.
        
        % \subsection{Charging patterns and the utilization of different charging infrastructure types}

        % \subsection{Impact of large-scale charging loads on the electricity system}
        
        
            
        
    \section{Modeling of charging demand, charging infrastructure capacities, and roll-out}
        \subsection{Graph-based charging infrastructure allocation}

A graph representation of a street network is frequently used for the allocation of charging stations as it allows to reflect high spatial resolution in modeling while simplifying complex networks without suffering from information loss \citep{Metais2022, Pagany2019}. Edges typically represent connections between nodes, and nodes represent junctions or ends of edges. \citet{Metais2022} differentiated between node-based and flow-based approaches. In node-based approaches, charging demand is assigned to nodes based on population density and the goal is to allocate charging infrastructure such that as much demand as possible is covered at nodes. Contrary, flow-based approaches aim to allocate charging stations along edges through which the highest vehicle flows occur. This approach requires origin-destination data describing the traffic load between all nodes in the network, which is not always available \citep{Ghamami2016}. \citet{Metais2022} concluded that both approaches offer benefits: node-based approaches offer input data simplicity and the ability to regard charging capacity, while flow-based approaches succeed in reflecting the nature of traffic flow. Furthermore, the authors stated that the combination of both approaches may yield the most robust charging station allocation model.

Overall, the proposed methodologies encompass mostly extensions of the basic formulations of node- and flow-based approaches while using various optimization techniques, such as genetic algorithm, particle swarm optimization and integer programming, as well as iterative algorithms \citep{SHAREEF2016403}. Most commonly, objective functions are formulated as minimization of charging infrastructure installation costs and driver expenses, maximization of charged vehicles, and minimization of failed trips. \citet{6183279} proposed an allocation model where nodes represent service area positions, and optimized allocation using integer programming to minimize total costs of installation, recharging, and transportation. \citet{Wang2018} optimized highway charging infrastructure allocation with regard to minimizing charging time, queuing, driver costs and deployment costs using a multistage equilibrium-model with origin-destination data. \citet{Jochem2019} used the flow-based approach, minimized the number of charging points in the network, and conducted calculations on sizing based on flow coverage. \citet{7575934} also followed a flow-based approach and proposed a two-stage genetic algorithm with multi-objective formulation to minimize construction costs and charging costs. \citet{Napoli2020} developed an iterative allocation algorithm correlating the distance between charging stations with the driving range of vehicles. \citet{Csiszar2020} designed a sequential selection of positions based on traffic volume and nearby settlement population.

In these studies, driving range was commonly used as an indicator for maximum distance between charging stations. \citet{Csiszar2020} found that with increasing range, the number of serviced vehicles by a given charging network increases. \citet{Kavianipour2021} analyzed the effect of increasing driving range and charging power, indicating that with technological developments, required investment costs sink as fewer charging points and less charging capacity need to be installed. \citet{WANG20181255} analyzed optimal deployment under a fixed budget and observed a saturation of flows at a certain driving range, indicating that under a given budget, higher ranges would not affect infrastructure requirements.

        \subsection{Long-term optimal charging infrastructure deployment for different consumer segments for passenger car}

In charging infrastructure roll-out planning, charging stations are placed with the objective of cost-optimality and meeting the charging demand \citep{METAIS2022111719}. Research analyzing the optimal expansion together with its impact on BEV adoption can be categorized into three strands: first, system dynamics models which capture a high level of complexity of the interaction of different components \citep{9706486, GAO2024103969}; second, agent-based models with high spatial resolution in the allocation of charging stations and mobility behavior \citep{LIAO2023103645, WOLBERTUS2021262}; third, large-scale optimization approaches that aim to map cost-optimal decarbonization pathways for the transport sector \citep{TATTINI2018265}. \citet{WOLBERTUS2021262} find that single distributed roll-out strategy has a positive effect on BEV adoption, while hub roll-out scenarios show significantly smaller impact.

Concerning the third category, there are many studies dedicated to compensating for the shortcomings of the total-cost-minimization approach \citep{TATTINI2018265, PATIL2023104265}. \citet{Venturini2019} describe different tools to introduce behavioral aspects, concluding that effective tools are segmentation of consumers and inclusion of intangible costs. \citet{luh2022behavior} introduce segmentation of consumer groups by housing type and include different types of charging infrastructure, finding that the option to charge at home has a substantial impact on BEV adoption, while public slow and fast charging play a vital role when private charging is limited. A follow-up publication \citep{LUH2024122412} includes the value of travel time, which also includes time to recharge and detour time to reach the closest charging station.

In the modeling of optimal public charging infrastructure roll-out, there exist only a few studies that explicitly consider income class disparities: \citet{BATIC2025135834} develop a graph neural network to plan public charging infrastructure deployment in urban areas on street grid level, while aiming to reduce spatial inequity in charging infrastructure availability. \citet{LONI2023128796} introduce the maximization of social equity access to a multi-objective charging infrastructure planning optimization model at the neighborhood level.

        \subsection{Charging infrastructure planning for international long-haul road freight transport}

The reduction of fossil fuel dependency in the road freight transport sector has been challenged by its significant growth in service demand and the current lack of uptake of zero-emission technologies \citep{EEA_RoadTransport}. The \textit{Sustainable \& Smart Mobility Strategy} by the European Commission describes the goal to shift road freight transport demand towards zero-emission drivetrain technologies by 2050 \citep{EU_MobilityStrategy}. In recent years, many empirical and theoretical studies have showcased the applicability of battery-electric trucks (BET) for this purpose \citep{Plötz2022}. The large-scale adoption of BETs is supported by continuously declining costs of ownership \citep{MAGNINO2024113215, NOLL2022118079} and technological improvements in battery technology leading to the overcoming of often-disputed limited range and charging speed \citep{AGUILAR2022102624}.

One central challenge to enable the widespread adoption of BET is the build-up of a dense network of charging infrastructure \citep{MULLER2025104455}. The \textit{Alternative Fuels Infrastructure Regulation (AFIR)} explicitly mandates the build-up of mega-watt charging stations every 120~km along the Trans-European Transport network (TEN-T) \citep{EU_AFIR2023}. The high power levels connected to charging infrastructure for trucks pose a particular challenge due to the high volumes of simultaneous electricity loads that require substantial investments in grid infrastructure at distribution grid level \citep{ZHAO2025309}. A wide range of studies has been dedicated to exploring grid-friendly charging and using the flexibility potential of trucks for power grid congestion on large and small scales \citep{GOLAB2025134385, KARMALI2024101397}.

\citet{hildermeier2024power} have compared different border regions in the EU, which vary significantly in the costs of installing charging infrastructure and the charging activity due to different electricity market zones, network connection fees, and grid fees. The cost-optimal fueling of trucks has been observed to be driven mainly by different taxation laws between countries \citep{bmik2021mobility}. The country-dependent charging costs \citep{Lanz2022} are likely to shift the cost-optimal allocation of the fueling/charging activity, potentially allocating charging capacities to unfavorable locations in the electricity system.

        \subsection{Sub-annual modeling of charging demand for commercial fleets at large scales}

    \section{Contribution to the progress beyond state-of-the-art}
    % TODO: Modeling long-term charging infrastructure roll-out
    \hl{Kurz und knackig hinschreiben}
    