\section{Motivation}\label{sec:motivation}
Decarbonising the entire energy system is a primary objective within the European \textit{Fit for 55 Package} \cite{Fitfor55}. The expansion of \ac{RESE} emerges as critical, leading to a transformative shift in electricity market design. This transition of the electricity market in which flexible generation meets inflexible demand to one in which flexible demand meets inflexible generation results in the need for several flexibility options \cite{Gea2021, Plaum2022}. The need for flexibility arises not only to balance generation and demand \cite{Saffari2023, Gade2022} but also to address uneven \ac{RESE} distribution and the subsequent need to balance residual loads \cite{Allard2020}. Only with this flexibility the use of fossil fuel-fired power plants can be prevented, resulting in reduced CO\textsubscript{2} emissions and decreased \ac{RESE} curtailment. This leads to higher revenues, more investments, and an accelerated electricity market transformation \cite{Saffari2023}. Acknowledging these needs, the European Commission emphasises integrating flexibility into European electricity markets as one primary objective of the electricity market reform. This aims to diminish dependence on fossil fuel-fired power plants, particularly \ac{CCGT} \cite{EU_electricitymarketdesign}.

 
However, the need for decarbonisation extends beyond the electricity market to sectors that are hard to electrify, such as heavy transport, aviation, shipping, steel, and chemical industries \cite{Fortes2019}. Hydrogen is one promising option for overcoming this challenge. It can be used for the indirect electrification of these applications \cite{Nurdiawati2021}. Despite the potential of hydrogen, the current European hydrogen production results in considerable CO\textsubscript{2} emissions. Decarbonised hydrogen production by electrolysis can eliminate this disadvantage. Hence, the European hydrogen strategy sets a 40 GW electrolyser capacity target by 2030 \cite{european2020hydrogen, su132313464}. Several countries published national hydrogen strategies, setting individual goals by 2030 to meet these European expansion targets on a national level. However, using hydrogen for indirect electrification helps the relevant industry to meet its emissions targets and provides essential flexibility for the electricity market. Hydrogen integration introduces three types of flexibility: 
 
\begin{itemize}
 \setlength\itemsep{-0.5em}
    \item Firstly, Hydrogen can be stored and used when needed, providing end-user flexibility. This can be done seasonally through relatively inexpensive storage options. Finally, fuel cells can convert hydrogen into electricity. This combination can help provide electricity when \ac{RESE} are in short supply \cite{Lewandowska2018}.

    \item Secondly, Hydrogen can be stored to change the timing of hydrogen production, providing market time flexibility. Hence, the production can be decoupled from the actual demand because immediate coverage is unnecessary. These storage units allow for the optimisation of hydrogen production based on available electricity or the current electricity price.
    
    \item Thirdly, Hydrogen can transport renewable energy to areas where the transmission grid is insufficient and causes congestion, providing spatial flexibility. However, the required pipeline infrastructure is expensive, thus increasing end-user hydrogen prices \cite{Nastasi2019}.
\end{itemize}

Nevertheless, it is not only fossil fuel-fired power plants and hydrogen sector coupling that can provide flexibility \cite{Kara2022}. On the supply side, in addition, controllable \ac{RESE} and storage are used. From the infrastructure perspective, regional and local grids can add flexibility to the system. On the demand side, end-users can adjust their actual demand and provide demand-side flexibility. This requires flexible loads such as heat pumps or \ac{EV}. Furthermore, industry can use their flexible industrial processes to provide demand-side flexibility \cite{Golmohamadi2022}. 


While substantial demand-side flexibility already exists, integrating this into the electricity market requires the provision of price signals without high investment costs and entry barriers. A uniform smart meter roll-out is necessary to perform and monitor the corresponding interactions \cite{Newbery2018}. However, the profitability of flexibility decreases in electricity markets when it is installed on a large scale due to its price-smoothing effect \cite{Chyong2022}. Competition between flexibility options, such as battery storage units, sector coupling, grid expansion and demand-side flexibility, leads to smaller individual revenues \cite{Scheller2020, Quintero2022}. 

Moreover, demand-side flexibility lead to increased simultaneous electricity consumption, while it is beneficial for decarbonising the electricity market \cite{Gilleran2021, Tehrani2015}. This can be potentially problematic, as the integration of non dispatchable \ac{RESE}, combined with the electrification of several sectors, is already leading to increased power flows over the transmission grid. Consequently, the number of congestion events and their magnitude and frequency has risen in recent years \cite{ACER2021}. This occurs, because the demand and supply curves are intersected during dispatch to determine optimal generation schedules, while the transmission grid within a bidding zone is neglected. The market clearing involves thermal power plants, \ac{RESE}, and storage, maximising \ac{SEW} while allocating cross-border capacities between bidding zones. Hence, congestion between bidding zones can be prevented mainly during the dispatch through the explicit allocation of cross-border capacities. These capacities are calculated simultaneously with the market clearing through the principles of \ac{FBMC} \cite{Schonheit2021}. This makes more cross-border transmission capacity available through an improved physical representation of the grid \cite{Bergh2016}. However, transmission line limitations within a bidding zone are neglected. Therefore, congestion, a temporal overload of a transmission line, could occur. 


In addition to obvious long-term strategies such as transmission line expansion, redispatch measures remain crucial in the central European electricity market design to overcome these congestion. These measures regulate the power generation of thermal power plants based on \ac{TSO} instructions. A power plant will increase its power generation on one side of the congestion and reduce it on the other side to overcome this congestion. However, these redispatch measures and associated costs have increased and may continue to rise in the future \cite{AustrianPowerGrid2019, Wirtschaft2018}.