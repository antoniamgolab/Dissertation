\subsection{Limitations of the proposed methods}\label{sec:limitations}
To evaluate competitive and synergistic effects between different flexibility options, the ex-post calculated \ac{SEW} is evaluated. An essential component of \ac{SEW} is \ac{CS}, which determines the effects on consumers. This term is only associated with the \ac{CS} on the electricity market in this thesis, but positive effects in other sectors are neglected. For example, using \ac{EV}s instead of conventional cars increases demand, raising electricity prices and reducing \ac{CS}. However, end-consumers benefit more than conventional cars because electricity is still cheaper than gasoline.


However, it is not only consumers but also producers who are evaluated as a component of \ac{SEW}. \ac{PS} acts as an indicator in this context. A change in \ac{PS} is due to a change in electricity prices or generation schedules. These changed generation schedules imply lower or higher operation hours per year. The electricity market model assumes electricity generation costs of thermal power plants according to their \ac{SRMC}. These costs do not include thermal power plant investment costs. If these are included in the bidding process to guarantee a sufficient business model for the power plant operator, the electricity generation costs per unit will decrease with increased operation hours and vice versa. This work does not capture these effects of the \ac{PS}. Consequently, the extent of \ac{PS} reductions and increases can be significantly higher as postulated in this study. 


The implemented hydrogen sector coupling approach couples the European electricity market with national hydrogen markets. Although these are modelled with the same hydrogen price, no hydrogen trade between the individual countries can be evaluated. Thus, the effect of hydrogen transport versus electricity trade cannot be analysed. 


Fuel cells, as one sector coupling component, are implemented as additional generation capacities to evaluate the welfare effects of hydrogen sector coupling, including hydrogen production, storage, and reconversion to electricity. As the results show that these are used to reduce price peaks any additional generation capacity on the market can generate similar results.


Sensitivity analysis was performed in this work to evaluate the effects of different hydrogen prices on consumers and producers. While different hydrogen prices are used, the natural gas price, as an exogenous model input, is the same for all scenarios. A cross-price elasticity between natural gas and hydrogen prices will affect the use of \ac{CCGT} and hydrogen production. Furthermore, hydrogen generation technologies other than electrolysis can provide additional generation capacity. Blue hydrogen, produced by steam methane reformation, and turquoise hydrogen, produced by methane pyrolysis, can be an alternative to electrolysis. Both hydrogen types need natural gas in production but emit CO\textsubscript{2}, which could be stored and captured. In the model, the electricity generation fleet and electrolysers are installed without determining their physical connections. Direct coupling between a specific generation unit and a specific electrolyser, besides the market, will change the hydrogen \ac{PS} as there is no option to use low electricity market prices to generate hydrogen anymore. This can significantly affect the amount of produced pink hydrogen and nuclear power plants \ac{PS}. In addition, this approach would allow the integration of policy and regulatory schemes regarding the definition of renewable hydrogen. 


The used modelling approach does not limit hydrogen storage capacity. As a result, the required capacity is calculated ex-post. The results will change if the available hydrogen storage capacities are smaller than those needed for the reconversion of hydrogen to electricity as calculated ex-post. Hence, the available hydrogen storage capacity drastically impacts the results, especially in 2040, with high electricity price peaks.  


The implementation of \ac{EV}s as demand-side flexibility uses \ac{CT}s to aggregate individual \ac{EV}s to fleets with similar timeframes in which it is possible to provide demand-side flexibility. These \ac{EV} fleets use the same \ac{CT}s independent of their actual location and population density. A more precise distinction between urban and rural areas will affect the available flexibility potential. This could be done by coupling a traffic flow model to the used two-stage electricity market model. Furthermore, this thesis implements private \ac{EV}s into the model. Consideration of the electrification of the commercial fleet will add additional flexibility with a different spatial and temporal distribution. 


To evaluate the effects of demand-side flexibility on congestion, \ac{RESE} curtailment, thermal power plant regulation and \ac{EV} demand-side flexibility are available as redispatch measures. The additional implementation of storage regulation will add additional redispatch capacity. 