\subsection{Discussion}\label{sec:discussion_RQ3}
\textbf{Research question 3:} \textit{What are the influences of customer integration as aggregated \ac{EV} fleet demand-side flexibility on congestion and needed redispatch measures? On the one hand, as flexible demand using smart-charging to minimise electricity costs, and on the other hand to balance redispatch needs.}


The results show that the use of temporal flexible charging events significantly reduces redispatch costs in comparison to the scenario without \ac{EV} demand-side flexibility. These costs are reduced from 3.3\% to 13.9\% in the three scenarios. Hence, the cost reduction increases significantly with the number of available \ac{EV}s. If demand-side flexibility is used in the dispatch for market-based charging, the costs of the subsequent redispatch increase. Especially if a large \ac{EV} fleet is considered, these costs rise by 186\%. 
The demand of \ac{EV}s and the cost savings from their participation in the congestion management market result in potential profits for the participants of less than one cent per kWh. Hence, from the customer`s perspective, participation in the congestion management market is only reasonable if additional attractive conditions are provided. This could be done by different pricing models of charging tariffs at public charging points and grid tariff designs. 


The use of demand-side flexibility to balance redispach needs influences \ac{RESE} curtailment as redispatch measure. On the one hand, it reduces \ac{PV} and wind curtailment (up to 25\%) and decreases the generation increase of thermal power plants as a redispatch measure. This, moreover, leads to less CO\textsubscript{2} emissions. On the other hand, using this flexibility in the dispatch increases \ac{RESE} curtailment. Especially if a large \ac{EV} fleet is considered. The influence of demand-side flexibility on curtailment can be seen when considering curtailment in redispatch only and when considering both dispatch and redispatch as a whole. This demonstrates that the demand increase in dispatch reduces curtailment. The additional curtailment in the redispatch compensates for this positive effect. Avoided \ac{RESE} curtailment leads to lower CO\textsubscript{2} emissions. As a redispatch measure, flexible demand significantly reduces redispatch-caused emissions (up to 85\%). Flexible demand during the dispatch leads to frequent congestion due to market-based charging. Hence, emissions rise. 


Using demand-side flexibility to reduce balancing needs leads to large demand shifts. 
The total amount of electricity to be regulated during the redispatch increases due to the rebound effect of demand shifts. The associated regulation effort of the aggregators and the \ac{TSO} requires a suitable IT infrastructure and standardised interfaces of the charging infrastructure. This establishes a congestion management market with low barriers for customers.


In summary, using demand-side flexibility as a market-based charging strategy significantly increases redispatch costs and CO\textsubscript{2} emissions. The use of demand-side flexibility in the redispatch, on the other hand, significantly reduces the need for redispatch measures, whilst the associated costs, CO\textsubscript{2} emissions and \ac{RESE} curtailment are reduced. This finding shows that integrating demand-side flexibility, combined with \ac{RESE} expansion, can be essential to achieving climate goals. It can reduce the need for redispatch measures and transmission grid expansion. However, the total amount of electricity to be regulated during the redispatch increases due to the rebound effect of demand shifts. 