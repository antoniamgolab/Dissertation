\section{Contribution to the progress beyond state-of-the-art}\label{sec:SOTA_progress_beyond}
The two-stage fundamental European electricity market model developed to answer the research questions gives several signs of progress beyond the current state of the art. 
While some studies focus on the revenue generated by different flexibility options, this study investigates the influence on \ac{SEW}. Even if the combination of different flexibility options reduces revenue, the extent of benefits of this combination to the overall system must be considered. One possibility is to partially shift the value of the \ac{SEW} increase to flexibility providers to establish sufficient investment incentives despite potentially decreased revenue. Therefore, this thesis investigates the effects on \ac{SEW} and its components. This allows for a detailed evaluation of the effects on various stakeholders on the supply and demand side. 


The evaluation of a detailed hydrogen sector coupling allows the assessment of the impact of sector coupling between the European electricity market and national hydrogen markets. This includes hydrogen production by electrolysers, seasonal storage and reconversion of hydrogen to electricity. Hydrogen storage units allow long-term energy storage by shifting energy over time. Optimising for the entire year provides valuable information about seasonal effects. 


To analyse the potential of demand-side flexibility provided by \ac{EV}s to reduce redispatch needs, these are implemented as market participants in the dispatch and the redispatch. A detailed power plant fleet and a market clearing before the redispatch make it possible to consider the individual costs for regulating generation units and their availability. 


For this purpose, this thesis introduces several additional flexibility options into the European electricity market model. On the demand side, \ac{EV} fleets provide flexibility. On the producer and demand sides, storage provides additional flexibility. On the infrastructure side, the expansion of a partly congested transmission line between two bidding zones is introduced. In addition,  sector coupling to the hydrogen market is implemented. This includes electrolysers as hydrogen producers, national hydrogen markets, hydrogen storage, and fuel cells that perform the reconversion of hydrogen to electricity. 
Following the dispatch, the cost-optimal redispatch is calculated using, among other things, \ac{EV} fleets to balance redispatch needs. \\
The contributions of this thesis are as follows:
\begin{enumerate}
\item [(i)] Implementation of sector coupling between the European electricity market and national hydrogen markets. Conversion technologies allow these markets to interact. A joint \ac{SEW} optimum is obtained. Because the hydrogen market is not modelled as a price taker, increased demand or production can influence the electricity price.
\item [(ii)] Implementation of aggregated \ac{EV} fleets as demand-side flexibility into a generic \ac{EV} model. Many of the necessary calculations are conducted before the start of the optimisation, considering the technical limitations as an \ac{LP} model. Consequently, integrating this model into computationally intensive electricity market models is possible.
\item [(iii)] The qualitative and quantitative effects of integrating various flexibility options on electricity producers and consumers are analysed in detail. Possibilities of mutual substitution between these flexibility options are identified and analysed. 
\item [(iv)] Synergistic and competitive effects between flexibility providers regarding the overall \ac{SEW} are evaluated.
\item [(v)] Examination of sector coupling solely through hydrogen production and its functionality as seasonal storage and hydrogen reconversion to electricity. Comprehensive analysis of the changes in \ac{CS}, \ac{PS} and \ac{SEW} and welfare trade-offs between the two markets.
\item [(vi)] Evaluation of the influence of distributed \ac{EV} demand-side flexibility on redispatch costs, CO\textsubscript{2} emissions and \ac{RESE} curtailment.
\end{enumerate}