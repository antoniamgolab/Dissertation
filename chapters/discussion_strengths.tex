\subsection{Strengths of the proposed methods}\label{sec:strengths} 
A two-stage fundamental European electricity market model is developed to answer the research question of this thesis. The model includes a dispatch model that calculates the market clearing while maximising \ac{SEW} and a subsequent redispatch model. Hydrogen sector coupling is integrated into the model by maximising the electricity and the hydrogen market's joint \ac{SEW}. Hydrogen production increases electricity demand, influencing electricity generation and, consequently, the electricity price. The latter influences electricity generation via hydrogen use by fuel cells and, thus, the price of electricity. Hence, the electricity and hydrogen markets affect each other. As a result, they are optimised jointly to achieve a holistic \ac{SEW} optimum. Therefore, the methodology does not model hydrogen technologies as price takers and the electricity market as price setters but achieves a joint optimum without requiring several model iterations. 


In addition to hydrogen sector coupling, the implementation of \ac{EV}s and their time-dependent and spatially distributed demand-side flexibility is essential to evaluate the effects on the electricity market. The described methodology allows the aggregation of charging events without losing the necessary parameters for their application as demand-side flexibility. Furthermore, compliance with the technical limitations of the vehicles and the charging infrastructure and the avoidance of restrictions for the \ac{EV} users is ensured. However, the formulation as \ac{LP} is possible because many of the necessary calculations can already be performed before the start of the optimisation. Therefore, integration is possible for large-scale and computationally intensive models. The dispatch model with subsequent redispatch is solved as a rolling optimisation with short time steps due to the model's level of detail and size. Consequently, the theoretically possible optimum is approximated but corresponds to the realistic behaviour of \ac{DA} electricity markets.  


The implemented flexibility models are not limited to the flexibility options evaluated in this thesis. The methodology used to couple markets is more comprehensive than the coupling between the electricity and national hydrogen markets. Instead, it can be easily adapted to integrate other sectors with appropriate parameter selection and conversion technologies. However, the methodology for implementing demand-side flexibility is not limited to \ac{EV}s. However, it can also be easily adapted to integrate other coupled sectors through appropriate parameter selection. Hence, in addition to the electrification of individual transport, integration as demand-side flexibility is possible in other sectors.


In addition, the presented results to evaluate competitive and synergistic effects between flexibility options were determined based on a case study between the bidding zones Austria and Germany. However, the different generation fleets in the two areas serve as representative examples for many European regions. Hence, the results can be decoupled from the case study and analyse in detail the qualitative and quantitative effects of different flexibility options. 


Furthermore, hydrogen sector coupling has currently implemented no policy or regulatory limitations. This approach makes it possible to determine the best possible solution for hydrogen production and the share of different hydrogen types while achieving a holistic optimum. 


Another strength of this thesis is that modelling the redispatch in a two-stage model makes it possible to include the dispatch generation schedules in detail. Thus, the use of power plants can only be regulated depending on the previously planned generation schedules. In particular, in an electricity market dominated by \ac{RESE}, it cannot be assumed that thermal power plants are always dispatched and can consequently be regulated for redispatch measures. Furthermore, this modelling approach, in combination with the spatial distribution of power plants, considers the distance in the meshed transmission grid between controllable units and the congestion.