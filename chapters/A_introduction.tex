\chapter{Introduction}
\section{Background: Electrification of road transport sector in the context of the European energy system decarbonization}

The road transport sector is a prominent contributor to Europe's\footnote{EU-27} greenhouse gas emissions, accounting for 73\% of total transport sector emissions in 2023 \cite{EEA2025GHGTransport} and the transport sector to 29\% of total European GHG emissions. The European Green Deal indicates a minimum requirement for a 90\% reduction in GHG emissions in the transport sector by 2050 [REF].

In the context of the energy system decarbonization, the electrification of the transport sector has long been established as the cost-optimal pathway towards effectively reducing fossil fuel demand in the transport sector [REF many ]. This is mainly driven by the energy efficiency in the direct electrification as well as cheap fuel costs, i.e. electricity, provided through renewable energy sources. 

In particular, the battery-electric drivetrain has gained significant technological and market maturity during the last decade through significant advances in charging speed, driving range, and dropping costs [REF]. A clear picture has been emerging on this by the increased implementation of battery-electric vehicles in the private passenger and commercial sectors [REF]. The latest data on this was published for year 2023, indicating that 1.8\% or the EU passenger car fleet, 0.1\% of trucks and 2.5\% of buses are electric \cite{ACEA2025VehiclesEuropeanRoads}.

The transition towards electrified fleets is also supported by European Union (EU) level policies. The Fit for 55 Package specifies a reduction of at least 55\% by 2030 [REF]. To achieve this, the Fit for 55 policy package specifies regulations for CO\textsubscript{2} emission standards for road vehicles, regulations on the introduction of alternative fuels in the maritime and aviation and for charging and alternative fueling infrastructure. A significant regulation in this framework is the 100\% CO\textsubscript{2} emission reduction target for new passenger cars from 2035. 


A central element in this frame work is the Alternative Fuels Infrastructure Regulation (AFIR; Regulation (EU 2023/1804), which sets binding targets on expanding charging infrastructure for battery-electric vehicles, including passenger and heavy-duty vehicles, in terms of distance-based targets for the core European road Network, the Trans-European Transport network (TEN-T), and compulsory national targets. The distance-based targets indicate the distances along the network at which charging or alternative fueling stations should be installed. On the national level, the regulation addresses the minimum availability of charging capacity, which is dependent on the number of registered battery-electric vehicles in each country. 


The build-up of charging infrastructure is vital to enable the road transport electrification and, moreover, to support the cost-effective transport and travel [REF]. Particularly, the demand-oriented siting of charger capacities has been identified to be crucial to overcome range-anxiety of BEV drivers[REF], provide convenience and more time-efficient, and therefore more cost-efficient in the commercial sector, charging to the user [REF], and support the cost-effective investments in charging infrastructure by siting charging stations at locations of high utilization [REF]. However, the quantification and geographic allocation of the required charging capacities remain a non-trivial problem, given the balancing capacity sizing with the infrastructure utilization, the added layer of complexity through the integration to the electricity system, and the limited empirical experience because of the currently still low adoption numbers.  
% Policies at the country level support electrification primarily by subsidizing the purchase of battery-electric vehicles, the installation of charging stations, and providing benefits such as free parking or toll exemptions [REF].
\begin{comment}
    
last paragraph should be: \hl{\textit{it should}}
\begin{itemize}
    \item Decarbonization goals -- main policy packages: Green Deal, Fit for 55, Alternative Fuel directive; ... (EU ) --- how refer these also to the road transport electrification? REPowerEU plan [1] and the "Fit for 55
    \item On country level in Europe?
    
    \item why electrification used for decarbonization? efficiency, battery costs (energy system perspective)
    \item also co2 standards regulations for cars,vans , \dots  that imply 
    
    \item Global perspective: Where are we with the electrificiation? which segments and degree?
    \item what are the enablers and hinderers
    \item introducing the role of charging infrasturcture
\end{itemize}

\end{comment}


\section{Problem statement: Geographic allocation of future charging loads}
% \textit{maybe distinctly separate from the detailed operational research looking at singular fleets; or at a small geographic scope, f.e, a distribution grid}

The overarching problem statement addressed by this work is the geographic allocation required for charging infrastructure capacities. Understanding long-term requirements for large-scale take-ups of battery-electric vehicles at \textit{large} geographic scopes, i.e., beyond regions and countries, is particularly crucial to two decision-making perspectives: First, transport policy makers and infrastructure planners who have to translate the regulations, such as the AFIR, to the concrete investments in charging sites. These can include national or local authorities, or road network operators that are responsible for enabling for example, by tendering or investing in local grid connection capacity, future charging infrastructure investments. Second, operators of the transmission and electricity grid --- of, both the transmission and distribution system --- require long-term planning in the grid inforcement to support increased peak loads through the charging of battery-electric vehicles and to integrate potential demand-side flexibilities for, for example, congestion management.

The question of \textit{where} charging capacities are  based on answering the following questions: 
\begin{itemize}
    \item Where is the charging infrastructure \textit{needed}?
    \item How do the local charging capacity and the relative setup of the charging capacities impact \textit{charging behaviors and demands}, by impacting BEV adoption and utilization? 
    \item How can charging capacities be utilized in an electricity-system-friendly way?  
\end{itemize}

Depending on the exact specification of the question surrounding the \textit{where}, different analytical methodologies are to be used. These mainly different in the following modeling dimensions:
\begin{itemize}
    \item The \textbf{geographic scope}: With this definition, a geographic system boundary is drawn based on assuming the boundaries within which interactional effects in charging capacity planning may appear and the geographic extent within which vehicles would be moving
    \item The \textbf{spatial scale}: This translates to how specific the geographic allocation is required. For example, from the perspective of a highway network operator, the charging capacities by service area are of interest. In contrast to this, transmission grid operators provide electricity for large areas at each substation. Therefore, this problem requires less spatial granularity in the resolution.
    \item The \textbf{temporal scope}: The extent of the analysed period is essentially dictated by two core elements in the research question, which are the planning horizon of the charging infrastructure development, and, moreover, the scope of modeled developments in the system that impact the required charging infrastructure. This could, for example, include the development of the vehicle fleet for which the realistic modeling of the aging process required the temporal scope of 10+ years.
    \item \textbf{Temporal scale}: The granularity in time steps is similarly determined as the temporal scope. In the case of electricity system operation, the minimal temporal resolution is hourly, while assumptions on the average hourly distribution of charger utilization and mobility patterns allow for modeling at higher temporal granularity.
    \item \textbf{The vehicle segment}: Road transport vehicles can be coarsely divided into two groups: The \textit{passenger} and \textit{commercial} vehicles. For both passenger and commercial transport, the distinction between local/short-haul and long-distance/long-haul use is critical: the former can largely rely on home, workplace, or depot charging, while the latter depends on publicly accessible en-route infrastructure.
\end{itemize}

These model design choices are interdependent and are chosen in trade-off with each other. 



\begin{comment}
Why is it relevant? 
\begin{itemize}
    \item who is interested?
    \begin{itemize}
         \item electricity system planning perspective
    \item policy makers/planning in tender processes/ highway network infrastructure planners
    \end{itemize}
   
    \item understanding the following dimensions
    \begin{itemize}
        \item where charging infrastructure is needed
        \item how this impacts the demands (via which parameters? i.e. costs, non-monetary/subjective parameters)
        \item how the capacities can be utilized in a electricity-system-friendly way
    \end{itemize}
\end{itemize}
These dimensions are driven by:
\begin{itemize}
    \item where: on different geographic levels (i.e. different geographic resolutions)  
    \item timing of the demand: long-term: when fleets are electrified; + short-term: when demand/flexibility is available (hour of the day)
    \item different by consumer segment characterized by different mobility patterns (temporal and locational extend)/ charging infrastructure utilization
\end{itemize}

Traffic flow \dots 
\end{comment}
Compared to other consuming sectors, such as, f.e., building or industrial facilities, the transport sector has high flexiblity in the allocation of its demand in geographic dimension as multiple potential fueling/charging allocations are visited; because of different mobility patterns, dictating routing and timing of position.

A common representation at large geographic scopes for transport demand is origin-destination flows. These indicate how much passengers or specific freight goods are transported between two regions. Moreover, these also may indicate this by transport mode. Coupling origin-destination data with network topology results in information on specific traffic flow by location.

The use of origin-destination flow representation allows for consideration of flexibility in demand allocation for large fleets.

With the relevance and decade-long awareness concerning the challenges of charging infrastructure capacity allocation, there exists an extensive body of literature dedicated to analysing these different aspects. Along with, this state-of-the-art methodologies have been developed and used throughout. The traffic-flow-based representation of the demand has been leveraged in two ways (in the context of geographic allocaiton of charging capacity):
\begin{itemize}
    \item facility allocation problem adapted for charging capacity allocation
    \item national freight transport models 
\end{itemize}


The vision of this thesis is to extend the scientific literature on geographic allocation of charging capacities for different vehicle segments to extend on understanding transport-demand-driven charging capacity allocation while leveraging the origin-destination data. 
We thereby go beyond state-of-the-art methodologies to integrate this origin-destination-flow-based representation of charging demand, while addressing important interrelations around consumer-segment-dependent vehicle fleet turnover, the cost-optimal charging and the spatial modeling of charging capacities.


% \textit{old}: 
% \begin{itemize}
%     \item facility allocation problems; i.e., addressing the geographic allocation at high granularity in geographic dimension; assuming endogeneously given demand/ objective: covering demand cost-optimally;capacity planning; 
%     \item operational models, coming from energy management and focusing on optimizing charging activity/ objective: covering demand cost-optimally in regard to costs of grid operator and fleet operator; geographic scope:small; granularity: high; (f.e. distribution grid)
%     \item decarbonization pathways considering the fleet development, charging infrastructure investements; 
%     \item 
% \end{itemize}

% \textbf{traffic flow?? }


% Beyond

% There is extensive literature dedicated to studying these different aspects. 
% Where is the lack of? 
% \begin{itemize}
%     \item 
% \end{itemize}

\section{Relevant literature in geographic charging load allocation}
Relevant literature can we categorized in:
\begin{itemize}
    \item Capacity allocation problems (Metais et al, -review; ); assuming exogenous charging demand (characterized by high degree of granularity in geographic dimension)
    \item 
\end{itemize}


Most relevant literature that we build on:
\begin{itemize}
    \item capacity-allocation problem;
    \item large-scope (international modeling): Shoman, Germany (Plötz)
    \item what the former two do not represent? any flexibility in the demand allocation
    \item national freight transport models aim to capture the cost-optimal pathway, considering also 
    \item including non-monetary aspects: MOCHOTimes
    \item integrating non-monetary: The work by Luh et al,
    \item the work of the swiss electricity market model: home vs. work charging 
    \item this one work on the operational perspective (long-distance, large-scope; with leveraging time)  
    \item traffic simulations based on simulations for hourly profiles on small scale: \textit{Forecasting the spatial and temporal charging demand of fully electrified urban private car transportation based on large-scale traffic simulation}; (look here more to the large-scale literature maybe I have referenced something interesting) 
\end{itemize}


\section{Thesis' research questions}

    The thesis addressed four research questions that extend the state-of-the-art literature on geographic allocation of future charging capacity requirements. Each of the research questions is addressed within a scientific journal paper. Figure \dots displays an overview of the articles characterized by the addressed problem statement, research question, and specifications on the corresponding publication. 
    

    Figure \dots illustrates the contextualization of paper in the state-of-the-art literature. The papers' insights extend the state of the art in various directions, reflecting on the diversity of BEV application segments that are characterized by different requirements for the charging infrastructure allocation. The first and second contribution addresses the private passenger car segment, while the third and fourth, focus on commercial vehicles fleets. Moreover, the focus of contributions one and three lie on the charging infrastructure planning for long-distance applications. Contributions two and four address the charging of vehicle used for shorter distances. The key difference here lies in the charging activity allocated en route.
    
    \subsection{Research question 1 (RQ1)}
    The first contribution addresses the spatial allocation of fast-charging infrastructure for battery-electric passenger cars along highway networks. Long-distance travel by car depends on the availability of en-route charging, yet planning such infrastructure involves considerable uncertainty regarding future demand. This demand is shaped by the technical characteristics of the electrified fleet (notably battery capacity and energy consumption), the share of battery-electric vehicles in the total fleet, and potential shifts in travel behaviour — including modal shift and overall demand reduction. The challenge lies in ensuring sufficient geographic coverage to enable long-distance electric mobility, while avoiding overcapacities that result from overly optimistic or misaligned assumptions. We apply a traffic flow-based optimization approach to the Austrian highway network to quantify fast-charging capacity requirements under a range of decarbonization scenarios for 2030, addressing the following research question: 

    \textbf{Research question 1 (RQ1):} What are the required fast-charging capacities along Austria's highway network for battery-electric passenger cars under different decarbonization scenarios, and where are these allocated? 

    In particular, this paper contributes to the understanding of the impact of technological advances on battery performance parameters, i.e., the charging efficiency and the driving range, of the traveling fleet along the highway network on the cost-optimal geographic allocation of charging stations. To analyze this, a mixed-integer linear program is proposed that considers the flexible allocation of charging demand under the consideration of local magnitudes of traffic flow, the network topology and parameters of the battery and charging stations. Fast-charging capacities are determined at high geographic resolution for explicit service areas.
    Based on energy system decarbonization scenarios and sensitivity analysis, understanding of these parameters on the fast-charger requirements is gained.
    \subsection{Research question 2 (RQ2)}

    
    The second contribution (REF contribution) focuses on charging infrastructure for the support of vehicles adoption throughout different consumer segments. We hereby focus on the role of public charging infrastructure in two aspects: the density of charging infrastructure stations within a region and the speed of charging infrastructure build up and its impact in the BEV adoption in neighbouring regions. 

    \textit{Problem statement:} Public charging infrastructure plays a key role in enabling the adoption of passenger BEVs at a large scale. Early adopters of BEVs are characterized by high income and homeownership [REF], adopters of lower income are dependent on affordable and accessible public charging infrastructure. Empirical case studies have found that there is a disparity in the availability for potential adopters of different income classes \citep{GAZMEH2024104222}. \textit{old text:} While in the long-term perspective, BEVs need to be adopted by passenger car owners of all socio-economic backgrounds \citep{HOPKINS2023113398}, this will also shift the current central role of home chargers to public chargers. Simultaneously, lower-income classes are more likely to travel longer distances than higher-income consumers for charging [REF]. 

    Along with this large-scale adoption, the flexibility in the geographic allocation of the charging activity increases through [REF]. Empirical studies have shown cross-border impacts of local public charging infrastructure to neighboring regions and countries, which is likely to increase with the spatial flexibility. 

    Bringing these drivers of these aspects together in the context of public charging infrastructure expansion at the regional level, the research question is as follows: 
    
    \textbf{Research question 2:} To what extent do the speed and spatial distribution of public charging infrastructure expansion influence BEV adoption rates among various income groups and across neighboring regions? 

    We introduce here intangible costs of consumers that very by income class levels to a techno-economic optimization to reflect on tolerances of time consumption on detouring to re-charge as well as tolerance for the time consumption of charging. Impacts across regions are analysed with the consideration of commuter traffic and the potential to flexibly allocate the charging activity between origin and destination charging. This contribution particularly highlights income-class-dependent preferences for charger utilization and points out a marginal impact of public charging infrastructure build-up speed beyond a region's border.

    \subsection{Research question 3 (RQ3)}

    In the third contribution, the focus lies on long-haul international trucking. The variable costs of fueling or charging are an essential driver of transportation costs. For fossil-fueled trucks, the country-dependent regulations of diesel taxation have determined allocations of the fueling activity[REF]. As there are country-dependent charging costs that are driven, by, next to taxation, local network fees and electricity-zone-dependent electricity prices [REF hildemeier, REF Lanz], this phenomenon will shift, with the adoption of battery-electric trucks, also to allocation of charging loads. This can lead to high concentrations of charging load in high-unfavorable locations in the electricity system. This leads to the following research question:
    
    % The third contribution (REF to contribution) addresses the charging capacity planning for long-distance trucks. The article's focus lies on the cost-optimal decarbonization of long-distance road freight transport, considering charging infrastructure placement for international freight transport.  

    \textbf{Research question 3:} What magnitude of charging infrastructure capacities is affected by geographically varying network fees and electricity prices in the cost-optimal planning of charging infrastructure allocation for long-haul battery-electric trucks? 

    Addressing this question, we analyze the impact of geographically varying charging costs on the cost-optimal allocation and the route-dependent adoption of charging 
    International origin-destination truck flows are hereby used as an input to the cost-optimization of fleet turnover, charging infrastructure build-up, and charging activities. In different electricity and technology pricing scenarios, we observe how local small charging costs along a corridor can lead to the high concentration of charging load, and, further, how the local intervention to charging prices can prevent these. 

    \subsection{Research question 4 (RQ 4)}
    With the electrification of the commercial fleet, large peak charging power levels are expected. The large-scale requirements in shifting charging loads in time to minimize required investments into grid capacity and balancing energy have been demonstrated by many studies [REF]. For this, the spatio-temporal distribution of the future charging loads and flexibility potentials of different vehicle segments within an electricity market zone are required. The fourth contribution [REF] addresses this for the case of Austria in 2040: 
    
    % The fourth contribution (REF to contribution) extends to multiple transport segments of the electrified commercial fleet and aims to provide spatio-temporally varying charging patterns and flexibility potentials for the integration of the commercial BEV fleet to the electricity system. 

    \textbf{Research question 3 (RQ3):} What is the expected charging demand of the commercial BEV fleet spatio-temporally distributed in Austria in 2040?

    The methodology integrates both local and interregional transport patterns to estimate accumulated charging demands and flexibility potentials. The consideration of the interregional transport is hereby based on, again, traffic flow patterns that include national and transit transport going through Austria. This analysis showcases a bottom-up modeling of charging patterns implying the signficant differences of charging flexibility by geographically varying transport demands by segments that are, moreover, dictated by transport flows along the highway network. 

    \dots \textit{\hl{TODO} insert here some stuff that the four research questions are aiming for}
    
   %  \begin{itemize}
   %  \hl{introduce this aspect to the introductions: }
   %     \item electrification of road transport (mention here most important concrete goals in EU for electrification; with concrete numbers and facts!) 
   %      \item planning of charging infrastructure/ charging locations + charging capacities (needed for grid reinforcement) + charging points and charging stations (for covering demand)
   %      \begin{itemize}
   %          \item for meeting charging demand of fleet | providing charging opportunity,  
   %          \item to plan for the electricity infrastructure accordingly (long-term planning of charging capacity to direct investments in power grid)
   %          \item policy design for integration of battery-electric vehicles to electricity system --- via flexibility; loads that are flexible in time AND SPACE; moveable loads; grid2vehicle
   %      \end{itemize}
   %      \item relevance of spatial allocation in regard to demand and electricity system; spatial flexiblity + allocation; moveable batteries with state of charge as function 
   %  \end{itemize}
   %  \section{Relevant Literature}
    
    
   %  %\section{Problem Statement}
    
   %  \begin{itemize}
   %      \item [RQ1] Review paper on Charging infrastructure planning methods [REF]
   %      \item [RQ1] charging infrastructure - different transprot sector decarbonization scenarios (techn, societal, \dots)[include here also what has happened research-wise since 2022]
   %      \item [RQ2] MOCHO Times paper;
   %      \item [RQ2] some paper showing the impact of income levels to BEV adoption
   %      \item [RQ2] empirical study on adoption of BEV 
   %      \item [RQ3] European-wide adoption of charging infrastructure for trucks (Shoman)
   %      \item [RQ3] tank tourism for BEVs (?)
   %      \item [RQ4] some paper on the large-scale impact of the electrification of commercial fleet 
   %      \item [RQ4] this paper on the flexibility estimation of different segments for Germany --- showcasing the potential of different segments 
   %  \end{itemize}

   % \begin{itemize}
   %      \item literature: focusing on initial setups of charging infrastructure for early adopters;
   %      \item expansion of state-of-the-art towards two dimensions:
   %      \begin{itemize}
   %          \item consumer-side, i.e., future demand of charging
   %          \item understanding impact of electricity system to charging infrastructure allocation and 
   %      \end{itemize}
        
   %  \end{itemize}
   %  \begin{figure}[h]
   %  \centering
   %  \includegraphics[width=1.0\textwidth]{graphics/RQs/rqs.drawio.pdf}
   %  \caption{Research questions tackling different interaction effects between the allocation of charging infrastructure, developments in the transport sector, and aspects of the electricity system. \hl{include references to publication + a bit more information on the components and characteristrics}}
   %  \label{fig:priceduration_basline}
   %  \end{figure}


   %  \begin{figure}[h]
   %  \centering
   %  \includegraphics[width=1.0\textwidth]{graphics/RQs/method_overview_rq.pdf}
   %  \caption{Details on the applied methods and analysis for answering \textbf{RQ1---4}.}
   %  \label{fig:priceduration_basline}
   %  \end{figure}
   %  \section{Research questions}
   %   this is tackled on multiple levels 
   %      \begin{itemize}
   %          \item understanding impact of longer range on infrastructure needs [RQ1]
   %          \item \textit{nicht mehr statisch;..} charging infrastructure expansion adoption of BEVs in differnt income levels + on neighboring regions [RQ2]
   %          \item \textit{stromnetz als kupferplatte; shift von bedarfsseite zu stromnetz} how charging infrastructure built-up will be affected by geographically varying connection + electricity costs [RQ3]
   %          \item \textit{hier zur operation vom stromnetz; mit repräsentation nicht mehr als kupferplatte, sondern höherauflösend; spatio-temporal} commercial fleet charging demand and flexibility estimation [RQ4]
   %      \end{itemize}
   %  Methodological contributions:
   %      \begin{itemize}
   %          \item traffic-flow-based methods (all)
   %          \item spatial flexibility together with technology adoption/modal shift
   %          \item traffic-flow-based flexibility estimation
   %      \end{itemize}

   %  \begin{sidewaysfigure}
   %  \centering
   %  \includegraphics[width=1.0\textwidth]{graphics/RQs/problemstatement_rqs_contribution.pdf}
   %  \caption{Overview of most important building blocks of thesis.}
   %  \label{fig:priceduration_basline}
   %  \end{sidewaysfigure}
   %  \hl{To include: intertemporal changes}
\section{Outline of the thesis}




