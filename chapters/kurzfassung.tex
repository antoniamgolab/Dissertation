\chapter*{Kurzfassung}
Das Hauptziel dieser Arbeit ist die umfassende Analyse der wirtschaftlichen und technischen Auswirkungen der Flexibilitätsintegration in den europäischen Strommarkt. Zu diesem Zweck wurden drei Forschungsfragen formuliert, die jeweils in einer wissenschaftlichen Publikation beantwortet wurden. Hierfür wurde ein zweistufiges europäisches Strommarktmodell entwickelt. Dieses umfasst ein Dispatch Modell, das den kostenminimalen Kraftwerkseinsatz berechnet und ein anschließendes Redispatch Modell, das Engpässe analysiert und kostenminimal beseitigt. Besonderer Augenmerk wird auf die Modellierung der Sektorenkopplung des Wasserstoffsektors und die Integration von Elektrofahrzeugen als laststeitige Flexibilität gelegt. Qualitative und quantitative Synergie- und Wettbewerbseffekte zwischen verschiedenen Flexibilitätsoptionen werden ermittelt. Anschließend wird die Sektorkopplung zwischen dem europäischen Strommarkt und den nationalen Wasserstoffmärkten im Detail analysiert. Dies umfasst die Wasserstoffproduktion, saisonale Speicherung und die Rückverstromung. Um diese zu bewerten, werden die wirtschaftlichen Auswirkungen für alle relevanten Erzeuger und Verbraucher im Strommarkt analysiert. Abschließend werden die Auswirkungen der nachfrageseitigen Flexibilität auf Engpässe und notwendige Redispatchmaßnahmen bewertet. Die Ergebnisse zeigen, dass lastseitige Flexibilität als Redispatchmaßnahme die Redispatchkosten erheblich senken kann, während marktpreisgesteuertes Laden das Gegenteil bewirken kann. Abschließend werden Ideen für zukünftige Arbeiten gegeben, darunter die Modellierung der Kreuzpreiselastizität zwischen Erdgas- und Wasserstoffpreisen, eine Implementierung der Regularien sowie eine Analyse der Auswirkungen der Ergebnisse auf Investitionsentscheidungen. Darüber hinaus kann die Elektrifizierung des Nutzverkehrs als Flexibilität in einem erweiterten grenzüberschreitenden Redispatchmodell implementiert werden. 
\cleardoublepage