\chapter{Methods}\label{methods}
In this chapter, the methodology and the mathematical formulation of the model are described. Furthermore, the use cases and scenarios examined to answer the research questions are outlined. Section \ref{sec:methods_methodology} comprehensively examines the methodology to implement the two-stage electricity market model. This includes the integration of sector coupling to the hydrogen market, \ac{EV}s, the redispatch model and the \ac{SEW} calculation. Section \ref{sec:methods_mathemical} includes the mathematical formulation of this methodology. The section is divided into the formulation of the dispatch model with the main constraints, the subsequent redispatch model and the ex-post calculation of \ac{SEW} and its components. The use cases and scenarios to answer the research questions are described in Section \ref{sec:methods_scenarios}.
\section{Methodology and model functionalities}\label{sec:methods_methodology}
\subsection{Overview}
This chapter describes the basic methodology used to implement the flexibility options into the model. In particular, the components that extend the existing European electricity market model \textit{EDisOn} \cite{Dallinger2018} are described. These are sector coupling between the electricity market and hydrogen markets and the integration of \ac{EV}s. Furthermore, the second stage of the model framework that performs the redispatch calculations is described. Finally, the methodology required for the ex-post calculation of \ac{SEW} and its components is described. An overview of these model stages and components is given in Figure \ref{fig:two_stage_model}. 
\begin{figure}[h]
    \centering
    \includegraphics[width=1.0\textwidth]{graphics/two_stage_model.pdf}
    \caption{Graphic illustration of the two-stage optimisation model}
    \label{fig:two_stage_model}
\end{figure} 
\FloatBarrier
\subsection{Sector coupling integration}
The sector coupling between the European electricity market and national hydrogen markets is done to integrate electrolysers, hydrogen storage, fuel cells and the resulting flexibility. The methodology is based on \textit{Hydrogen as Short-Term Flexibility and Seasonal Storage in a Sector-Coupled Electricity Market} \cite{LoschanH2}. Figure \ref{fig:Model_setup} gives an overview of this sector-coupling approach. 


The blocks on the left (\textit{Electricity Market}, \textit{Electricity Generators} and \textit{Electricity Demand}) were already part of the existing \textit{EDisOn} model. The blocks on the right (\textit{Hydrogen Market}, \textit{Hydrogen Storage}, \textit{Hydrogen Generators} and \textit{Hydrogen Consumers}) are model extensions. 
Several signals connect these two markets. The clearing price of the electricity market impacts the hydrogen market. It alters the production of hydrogen and the use of hydrogen to generate electricity. The first causes increased electricity demand, which influences electricity generation and price. The latter can have an impact on electricity generation and thus on electricity prices. As a result, the electricity and hydrogen markets impact one another. Thus, they are optimised together to achieve a holistic \ac{SEW} optimum for the electricity- and hydrogen markets. It is preferable to avoid an iterative solution in which the hydrogen market acts as a price taker and the electricity market acts as a price setter.
\begin{figure}[h]
\centering
      \includegraphics[width=1.0\textwidth]{graphics/RQ2/Model_setup.pdf}
      \caption{Graphic illustration of the sector coupling approach}
      \label{fig:Model_setup}
\end{figure}
\FloatBarrier
\subsection{Electric vehicle integration}\label{sec:EV_model}
Another flexibility option integrated as demand-side flexibility into the model are \ac{EV}s. The methodology is based on \textit{Flexibility potential of aggregated electric vehicle fleets to reduce transmission congestions and redispatch needs: A case study in Austria} \cite{LoschanEV}. The \ac{EV} model implementation focuses on a description of \ac{EV}s as flexible demand into an \ac{LP} unit commitment model. These macroeconomic models, formulated as \ac{LP}, aim to maximise \ac{SEW} of the overall system. Hence, business models and profit maximisation of individual market participants should not be considered. Otherwise, the \ac{SEW} optimum could not be found. To achieve these goals, this type of electricity market models are typically implemented as a unit commitment model \cite{Dallinger2018}. 


Individual \ac{CP} are aggregated to \ac{CT} such that the model can reasonably calculate the optimal dispatch or redispatch. Additionally, this strategy ensures that the technical limitations of the \ac{EV}s and their charging infrastructure will be considered, and no restrictions emerge for the user. 


It reduces the number of decision variables significantly. This is done based on a statistical evaluation of real-world charging events \cite{Flammini2019}. \ac{CP}s with similar plug-in and plug-out times are summed up. These similarities can be described as a specific type of car user. Each \ac{CT} describes a specific time slot within which a certain number of vehicles are connected to a charging station. For example, some \ac{EV}s  are used for commuting to work. These vehicles are charged at the workplace. The rectangular function of \ac{CT} \textit{day} in Figure \ref{fig:plug-in statistic} describes this. The share of plugged-in \ac{EV}s is larger than zero from 7 am to 5 pm (10 hours). 2\% of the considered vehicles are plugged in at each of these hours within the whole timeframe of 24 hours. Consequently, this \ac{CT} includes a total share of 20\% ($10 h \cdot 2\%/h = 20 \%$) of all \acp{EV}. This finding does not indicate that each vehicle connects to the charging station for the full timeframe. However, only the sum of the vehicles corresponds to this behaviour due to the aggregation.

\begin{figure}[ht]
\centering
   \begin{minipage}[b]{.49\linewidth} 
    \includegraphics[width=1.0\textwidth]{graphics/RQ3/plug-in statistic.pdf}
    \caption{Temporal distribution function of plugged-in vehicles}
    \label{fig:plug-in statistic}
   \end{minipage}
   \begin{minipage}[b]{.49\linewidth} 
    \includegraphics[width=1.0\textwidth]{graphics/RQ3/EV_demand.pdf}
    \caption{Charging demand of 30 000 vehicles with all charging strategies}
    \label{fig:EV_demand_200k}
   \end{minipage}
\end{figure}
Thus, the sum of all \ac{CT}s defines the share of the considered \ac{EV}s that are connected with the charging infrastructure at each hour. Hence, the sum of all \ac{CT}s is 100\%. The joint consideration of numerous vehicles within a \ac{CT} reveals a similarity considering demand and flexibility potential. Consequently whether seven vehicles charge daily or one vehicle charges weekly leads to the same \ac{CT}. 


The \ac{CT}s are the temporal distribution of the plugged-in vehicles that must be converted into an electricity demand. Charging strategies are used for this purpose. Three different charging strategies are used to calculate an electrical demand from the distribution function of the \ac{CT}s described above. 
\begin{enumerate}
\item [(i)] Strategy \textit{immediately}: Charging with the maximum available power allowed by the vehicle and the charging infrastructure. The corresponding demand of one \ac{EV} per \ac{CT} charging with a nominal charging power of 22 kW can be observed in Figure \ref{fig:uncontrolled_charging}. 
\item [(ii)] Strategy \textit{peak-shaving}: Charging with a perfect smoothed demand over the entire plug-in time to prevent demand peaks (Figure \ref{fig:smoothed_charging}).  
\item [(iii)] Strategy \textit{partly peak-shaving}: Charging the same way as with the strategy \textit{peak-shaving}, while the plug-in time is reduced by 60\%. 
\end{enumerate}
\begin{figure}[hb]
\centering
   \begin{minipage}[b]{.49\linewidth} 
    \includegraphics[width=1\textwidth]{graphics/RQ3/uncontrolled_charging.pdf}
    \caption{Strategy \textit{immediately}}
    \label{fig:uncontrolled_charging}
   \end{minipage}
   \begin{minipage}[b]{.49\linewidth}
    \includegraphics[width=1\textwidth]{graphics/RQ3/smoothed_charging.pdf}
    \caption{Strategy \textit{peak-shaving}}
    \label{fig:smoothed_charging}
   \end{minipage}
\end{figure}
The multiplication of the \ac{CT} distribution with the number of \ac{EV}s, their demand, based on statistical data and the charging strategies results in the overall \ac{CD}. Each of these \ac{CD}s corresponds to one \ac{CT} block and is the weighted sum of the three charging strategies (E.g. 70\% \textit{immediately}, 20\% \textit{partly peak-shaving} and 10\% \textit{peak-shaving}). This \ac{CD}s serve as model input and can be regulated as demand-side flexibility. 


Figure \ref{fig:EV_demand_200k} shows the resulting \ac{CD} for 30 000 \ac{EV}s over 24 h. The electricity demand of a specific \ac{CD} is always larger than zero, because of the partial use of the charging strategy \textit{peak-shaving}, whilst the car is plugged in. This implies the flexibility potential for the model. The comparison of Figures \ref{fig:plug-in statistic} and \ref{fig:EV_demand_200k} shows that a \ac{CD} only exists if \ac{EV}s of the respective \ac{CT} are connected to the charging infrastructure and that the demand is unevenly distributed due to a high proportion of the uncontrolled charging strategy \textit{immediately}.

A graphic illustration of one \ac{CD} and the resulting flexibility potential is shown in Figure \ref{fig:flexibility_potential_detail}. The initial charging demand can be decreased or increased but must be balanced till the \ac{EV} is plugged out. In the figure, the initial electricity demand is the direct connection between the plug-in and the plug-out time. Hence, this represents the charging strategy \textit{peak-shaving} because the charging power is the lowest possible constant demand without the need for demand peaks. The demand-side flexibility is limited by technical constraints that correspond to the fastest possible charging and the latest possible start of the charging.

\begin{figure}[h]
\centering
\includegraphics[width=0.8\textwidth]{graphics/Flexibility_potential_detail.pdf}
\caption{Graphic illustration of the demand-side flexibility implementation}
\label{fig:flexibility_potential_detail}
\end{figure}
\FloatBarrier
\subsection{Redispatch framework}
A two-stage optimisation model is used to analyse the influence of \ac{EV} demand-side flexibility on congestion and redispatch needs. To couple the dispatch and the redispatch model two additional data preparation steps are needed resulting in a four step model (Figure \ref{fig:model_overview}). The methodology is based on \textit{Flexibility potential of aggregated electric vehicle fleets to reduce transmission congestions and redispatch needs: A case study in Austria} \cite{LoschanEV}. 

\begin{figure}[h]
    \centering
    \includegraphics[width=1.0\textwidth]{graphics/RQ3/model_overview.pdf}
    \caption{Graphic illustration of the two-stage model}
    \label{fig:model_overview}
\end{figure} 
The time sequence of the model and the corresponding spatial resolution in use is illustrated in Figure \ref{fig:Redispatch_timing}.


The first step of the model is aggregating the necessary input data. \ac{EV}s are aggregated to several \ac{CD}s for each node. These \ac{CD}s are characterised by a specific timeframe and electricity demand (Section \ref{sec:EV_model}). The nodes within Austria are aggregated based on the research question to be answered, This could be an aggregation to one node per country to model the market without physical limitations. Another option is to use the nodes that are connected by transmission lines with a high congestion risk and \ac{FBMC} in the dispatch \cite{Bergh2016}. 


However, fewer nodes and transmission lines but all generation units exist in the dispatch. These generation units can no longer be precisely assigned to their geographical position (nodes) in the transmission grid. The correct cross border capacities are still included. Within Austria, the transmission grid is no longer modelled in sufficient detail. This may lead to congested transmission lines in Austria. 


In the following dispatch step, the generation schedule of all power plants within each node is calculated based on their \ac{SRMC} to maximise \ac{SEW}.


The previously aggregated nodes are disaggregated after the market clearing to consider transmission lines' physical limitations. Hence, a detailed power plant fleet and the related transmission grid are modelled for Austria. The power plant use remains unchanged compared with the one determined in the dispatch as the optimum of the market-clearing perspective. The last step of the optimisation involves preventing congestion by redispatch measures within Austria. As derived from the market-clearing perspective, local \ac{DA} prices are used to calculate the redispatch costs of different generation units. Thermal power plant regulation and \ac{RESE} curtailment are associated with redispatch costs to ensure revenue compensation. 

\begin{figure}[h]
    \centering
    \includegraphics[width=1.0\textwidth]{graphics/Redispatch_timing.pdf}
    \caption{Time sequence and spatial resolution of the optimisation approach}
    \label{fig:Redispatch_timing}
\end{figure} 
An overview of the two-stage electricity market model and the implementation of \ac{EV} demand-side flexibility can be seen in Figure \ref{fig:model_dependencies}. The “Dispatch model” block calculates the generation schedules and the \ac{DA} prices. These results influence the available capacity for redispatch measures. The block \textit{Redispatch model} performs the redispatch after the market clearing is done in the \textit{Dispatch model} block. The model uses the \ac{DA} price calculated in the dispatch. Available generation capacities for redispatch measures result from the previously calculated generation schedules. The flexibility potential provided by \ac{EV} fleets (block \textit{Electric Vehicle Model}) can be included in both model steps. It is used to balance redispatch needs or as a market-based charging strategy to minimise costs. The dashed arrows show the possible integration of the model stages, whilst the \ac{CD} and their constraints are always considered.

\begin{figure}[h]
    \centering
    \includegraphics[width=1.0\textwidth]{graphics/RQ3/Data-Flow.pdf}
    \caption{Electric vehicle model dependencies}
    \label{fig:model_dependencies}
\end{figure}
\FloatBarrier
\subsection{Socio-economic welfare}\label{sec:method_SEW}
This section describes the principle of the used approach to maximise \ac{SEW}. Furthermore, it includes a methodological explanation of hydrogen sector coupling and is essential for evaluating the results after the optimisation. The methodology is based on \textit{Hydrogen as Short-Term Flexibility and Seasonal Storage in a Sector-Coupled Electricity Market} \cite{LoschanH2}. 
\subsection*{Without Sector Coupling}
In economics theory, the electricity demand is elastic. A relationship exists between the actual demand and the price (Figure \ref{fig:SEW_theory}). This implementation cannot be solved by large-scale electricity market models in a reasonable time. As a consequence, electricity demand is assumed to be inelastic, but demand-side flexibility is implemented (Figure \ref{fig:SEW_model}). The demand remains constant until the electricity price equals the \ac{WTP}, an exogenous predefined maximum electricity price.

\begin{figure}[h]
\centering
   \begin{minipage}[b]{.49\linewidth} % [b] => Ausrichtung an \caption
      \includegraphics[width=1\textwidth]{graphics/RQ2/Social_Welfare_theory.pdf}
      \caption{Economic welfare in theory}
      \label{fig:SEW_theory}
   \end{minipage}
   \hspace{0.00\linewidth}% Abstand zwischen Bilder
   \begin{minipage}[b]{0.49\linewidth} % [b] => Ausrichtung an \caption
        \includegraphics[width=1\textwidth]{graphics/RQ2/Social_Welfare_model.pdf}
        \caption{Economic welfare in the model}
        \label{fig:SEW_model}
   \end{minipage}
\end{figure}
The \ac{PS} describes the area between the cost of electricity generation (supply curve) and the resulting clearing price (p*). The \ac{CS} is calculated by multiplying the difference between the clearing price and the \ac{WTP} by the hourly electricity demand (Figure \ref{fig:SEW_model}). Hence, the \ac{WTP} directly determines this value. Because this analysis considers the difference from a reference scenario, the chosen value does not affect the results.
\subsection*{With Sector Coupling}
Figure \ref{fig:SEW_theory} depicts the electricity market clearing without sector coupling to the hydrogen market. The intersection of the supply curve and the demand curve results in a market clearing price ($p^*$). The corresponding \ac{PS} is highlighted in a light orange area and the \ac{CS} in a light blue area. Figure \ref{fig:MO_wH2} depicts the impact of increased electricity demand due to hydrogen production. 
\begin{figure}[h]
    \centering
        \includegraphics[width=0.8\textwidth]{graphics/RQ2/Merit_Order_el_wH2.pdf}
        \caption{Electricity market merit order with sector coupling}
        \label{fig:MO_wH2}
\end{figure} 
\FloatBarrier
The electricity demand without sector coupling (blue line) raises (red line) due to the electrolyser's demand ($\Delta el.D$). This additional electricity demand will exist until a predetermined exogenous market clearing price is realised (P\textsubscript{bid el.}). If possible, the additional demand will be so high that the new clearing price ($p^{*}_{new}$) as the intersection between the supply curve and the new demand curve (red line) corresponds to the maximum bid price of the hydrogen producers on the electricity market (P\textsubscript{bid el.}). This is the maximum bid made by hydrogen producers on the electricity market. It depends on the associated hydrogen price until production becomes economically viable. The amount of this additional electricity demand is determined by the amount of hydrogen produced and the efficiency of the electrolysers. 


If the clearing price without the additional demand is already higher than P\textsubscript{bid el.}, there will be no additional demand. Their installed capacity limits the additional demand of hydrogen producers. Hence, the demand will not always rise until $p^*_{new}$ corresponds to P\textsubscript{bid el.}. This case is shown in Figure \ref{fig:MO_wH2}. The raised market clearing price increases the \ac{PS} (green and dark orange areas). Part of this increased \ac{PS} is due to a reduction in \ac{CS} (green area). Additionally, the \ac{CS} rises due to the increased electricity demand of hydrogen producers (dark blue area).

\begin{figure}[h]
    \centering
      \includegraphics[width=0.8\textwidth]{graphics/RQ2/Merit_Order_H2.pdf}
      \caption{Hydrogen market merit order}
      \label{fig:MO_H2}
\end{figure} 
Figure \ref{fig:MO_H2} depicts the additional \ac{PS} obtained on the hydrogen market. The amount of hydrogen produced (hy. Q; red line) is proportional to the electricity demand increase ($\Delta el.D$) on the electricity market via the electrolyser efficiency ($\eta$). Moreover, the hydrogen market clearing price ($p^{*}_{H_2}$) is proportional the electricity clearing price ($p^{*}_{new}$). A \ac{PS} (turquoise area) occurs if the hydrogen market clearing price is lower than the maximum hydrogen price (P\textsubscript{bid H2}). In this case, the hydrogen \ac{PS} is calculated by multiplying the difference between the clearing price and the maximum bid price by the hourly hydrogen generation. If these two prices are the same, the benefits of the market coupling are fully captured by the electricity market \ac{CS} increase. This occurs if there is more electrolyser capacity installed than needed.


Fuel cells that generate electricity using hydrogen are another available generation capacity. These are used if the electricity prices exceed their \ac{SRMC}. The required hydrogen must be produced ahead of time. This can result in hydrogen production even if the market clearing price (p*\textsubscript{H2}) is higher than the maximum hydrogen price (P\textsubscript{bid H2}), which corresponds to a negative electrolyser \ac{PS}.
\FloatBarrier
\section{Mathematical formulation}\label{sec:methods_mathemical}
This section describes the mathematical formulation of the two-stage optimisation model. To calculate the optimal dispatch, maximising \ac{SEW}, while neglecting physical restrictions of the transmission grid within bidding zones a reduced set of nodes ($N$) is used. In the subsequent redispatch the full set of nodes ($N^\mathrm{All}$) and all transmission lines are used to evaluate congestion. 


In both optimisation models the decision variables are written in lowercase, and exogenously defined constants are written in uppercase. If a variable is a constant or a decision variable may differ between the dispatch model and the subsequent redispatch model.
\subsection{Dispatch model}\label{sec:model_dispatch}
The European electricity market model \textit{EDisOn} \cite{Dallinger2018,LoschanEV,LoschanH2} minimises the overall electricity generation costs of the entire system, which corresponds to \ac{SEW} maximisation. This includes the European electricity market and national hydrogen markets in an \ac{LP} unit commitment model that is implemented in MATLAB. The electricity generation fleets consist of thermal power plants, \ac{RESE}, storage and fuel cells. Whereby thermal power plants include both CO\textsubscript{2} emitting power plants (coal, \ac{CCGT}, oil and lignite) and power plants without direct CO\textsubscript{2} emissions (nuclear, biomass, biogas). Interconnectors link the countries in the market area with a limited capacity. Further \ac{EV}s are implemented with a temporally available demand-side flexibility \cite{LoschanEV}. The model calculates the cost minimal dispatch of all thermal power plants, \ac{RESE} curtailment, and storage use. 
The \ac{RESE} generation technologies, namely \ac{PV}, \ac{RoR} and wind, are not dispatchable, and their generation profiles are exogenously defined. Figure \ref{fig:market_area} illustrates an example of nodes and interconnectors used for optimisations that use one node per country.
\begin{figure}[ht]
\centering
      \includegraphics[width=0.48\textwidth]{graphics/nodes_methodik.jpg}
      \caption{Nodes and grid of the electricity market model}
      \label{fig:market_area}
\end{figure}
\FloatBarrier
\subsubsection{Objective function}\label{sec:objective_function}
The objective function (Equation \ref{equ:objective_function}) minimises the overall electricity generation costs due to generation ($q_\mathrm{th}$) proportional to the \ac{SRMC}, and due to start ups ($\mathrm{str}$) proportional to the start-up costs ($C^\mathrm{start}$) \cite{Farahmand2012} of thermal power plants, 
\ac{RESE} \ac{OM} ($C^\mathrm{RESE}$),  
\ac{PHS} turbine \ac{OM} costs ($C^\mathrm{ps}$),  
\ac{NSE} fee ($\mathrm{VoLL}$), 
costs for insufficient \ac{PHS} \ac{SOC} ($C^{\mathrm{missingInflow}}$),
and costs ($C^\mathrm{DSM}$) for temporal demand reduction ($d^\mathrm{down}$).
Furthermore, hydrogen demand generates associated costs, and hydrogen generation adds revenue proportional to the hydrogen market price ($P^\mathrm{bid}_{H_2}$).
\newpage
\begin{flalign} 
\label{equ:objective_function}
&\underset{q, \mathrm{str}, q^\mathrm{tu}, \mathrm{spill}^{\mathrm{RESE}}, \mathrm{nse}, q^\mathrm{missingInflow}, d^\mathrm{down}, q^{H_2}_\mathrm{el}, d^{H_2}_\mathrm{el}}{\mathrm{min}} 
\\ \nonumber
& \sum_{t \in T} \sum_{\mathrm{th} \in \mathrm{TH}} \left( q_{t,\mathrm{th}} \cdot \mathrm{SRMC}_\mathrm{th} + \mathrm{str}_{t,\mathrm{th}} \cdot C^{\mathrm{start}}_\mathrm{th} \right)  
+ \sum_{t \in T} \sum_{\mathrm{ps} \in \mathrm{PS}} q^\mathrm{tu}_{t,\mathrm{ps}} \cdot C^\mathrm{ps} 
\\ \nonumber
+ &\sum_{t \in T} \sum_{r \in \mathrm{RESE}}
\left( Q^{\mathrm{RESE}}_{t} - \mathrm{spill}^{\mathrm{RESE}}_{t,r} \right) \cdot C^{\mathrm{RESE}} \\ \nonumber
+& \sum_{t \in T} \sum_{n \in N} \mathrm{nse}_{t,n} \cdot \mathrm{VoLL} 
+  \sum_{t \in T} \sum_{\mathrm{ps} \in \mathrm{PS}} q^\mathrm{missingInflow}_{t,\mathrm{ps}} \cdot C^{\mathrm{MissingInflow}}\\ \nonumber
+&\sum_{t \in T} \sum_{n \in N} d^\mathrm{down}_{n} \cdot C^\mathrm{DSM}_{n} - \sum_{t \in T} \sum_{e \in E} P^\mathrm{bid}_{H_2} \cdot q^{H_2}_{\mathrm{el}_{t,e}} + 
\sum_{t \in T} \sum_{\mathrm{fc} \in \mathrm{FC}} P^\mathrm{bid}_{H_2} \cdot d^{H_2}_{\mathrm{el}_{t,\mathrm{fc}}} \\
\nonumber
\end{flalign}
The \ac{RESE} \ac{OM} costs are proportional to \ac{RESE} generation ($Q^\mathrm{RESE}$) minus curtailment ($\mathrm{spill}^\mathrm{RESE}$). \ac{PHS} turbine \ac{OM} costs are proportional to the turbined electricity ($q^\mathrm{tu}$). The coupling of the electricity- and the hydrogen market is done by the amount of hydrogen produced ($q^{H_2}_\mathrm{el}$) and the amount of hydrogen used to generate electricity ($d^{H_2}_\mathrm{el}$). These hydrogen quantities influence the objective function in proportion to the hydrogen price ($P^\mathrm{bid}_{H_2}$).
\subsubsection{Constraints}
\paragraph*{Demand equilibrium}\mbox{}\\
Determining the demand that must be met in every node for each timestamp (Equation \ref{equ:dispatch_kirchhoff}) employs several parts. The exogenous electricity demand ($D_{n}$) and flexible, temporal shift-able, \ac{EV} demand ($d^\mathrm{EV}_{n}$). In addition, the demand can increase ($d^\mathrm{up}$) or decrease ($d^\mathrm{down}$), wherein the use of this power- and energy constrained flexibility is associated with costs.
\newpage
\begin{flalign} \label{equ:dispatch_kirchhoff} 
&D_{t,n} + d^\mathrm{up}_{t,n} - d^\mathrm{down}_{t,n} + d^\mathrm{EV}_{t,n} + \displaystyle\sum\limits_{e \in E_{n}} d^\mathrm{el}_{H_{2_{t,e}}} = \\ \nonumber
& \displaystyle\sum\limits_{\mathrm{th} \in \mathrm{TH}_{n}} q_{t,\mathrm{th}} + \displaystyle\sum\limits_{\mathrm{fc} \in \mathrm{FC}_{n}} q^\mathrm{el}_{H_{2_{t,\mathrm{fc}}}} + 
\displaystyle\sum\limits_{\mathrm{ps} \in \mathrm{PS}_{n}} (q^\mathrm{tu}_{t,\mathrm{ps}} - q^\mathrm{pu}_{t,\mathrm{ps}}) 
+ \displaystyle\sum\limits_{\mathrm{st} \in \mathrm{ST}_{n}} (q^\mathrm{out}_{t,\mathrm{st}} - q^\mathrm{in}_{t,\mathrm{st}}) \\ \nonumber
&+ Q^\mathrm{RESE}_{t,n} - \mathrm{spill}^\mathrm{RESE}_{t,n} - \mathrm{exch}_{t,n} + \mathrm{nse}_{t,n} : p^\mathrm{CP}_{t,n}\\ \nonumber
&\forall t \in T, \forall n \in N 
\end{flalign}
Demand compensation is done through thermal power plant generation ($q_\mathrm{th}$), \ac{PHS} turbine generation ($q^\mathrm{tu}$), storage discharging ($q^\mathrm{out}$), \ac{RESE} generation ($Q^\mathrm{RESE}$) minus curtailment ($\mathrm{spill}^\mathrm{RESE}$), and \ac{NSE} ($\mathrm{nse}$). Demand increases through \ac{PHS} pump use ($q^\mathrm{pu}$) and storage charging ($q^\mathrm{in}$). A power exchange ($\mathrm{exch}$) between nodes connects various nodes and bidding zones. The dual variable of this constraint ($p^\mathrm{CP}$) provides the local electricity market clearing price as an hourly profile, which is used to calculate \ac{CS}, \ac{PS}, \ac{CR}, and \ac{SEW}.

\paragraph*{Chargeable demand-side flexibility}\mbox{}\\
Chargeable demand-side flexibility and demand response are modeled as storage with 100\% efficiency (Equation \ref{equ:DSM_storage} - \ref{equ:DSM_storage_CAP}), wherein power of demand increase (Equation \ref{equ:DSM_increase}) and decrease (Equation \ref{equ:DSM_decrease}) are constrained.
\begin{flalign}
    &0 \leq d^\mathrm{up}_{t,n} \leq \mathrm{CAP}^\mathrm{DSR}_{n} \label{equ:DSM_increase}\\ 
    &0 \leq d^\mathrm{down}_{t,n} \leq \mathrm{CAP}^\mathrm{DSR}_{n} \label{equ:DSM_decrease}\\
    &\mathrm{soc}^\mathrm{DSR}_{t,n} = \mathrm{soc}^\mathrm{DSR}_{t-1,n} - d^\mathrm{down}_{t,n} + d^\mathrm{up}_{t,n} \label{equ:DSM_storage}\\ 
    &0 \leq \mathrm{soc}^\mathrm{DSR}_{t,n} \leq \mathrm{soc}^\mathrm{max. DSR}_{n} \label{equ:DSM_storage_CAP}\\ \nonumber
    &\forall t \in T, \forall n \in N
\end{flalign}
\paragraph*{Electric vehicle}\mbox{}\\
The summation of individual \ac{CD}s ($D^\mathrm{EV}$) and their positive ($d^\mathrm{EV +}$) and negative demand regulation ($d^\mathrm{EV -}$) to an aggregated demand per node ($d^\mathrm{EV}$) is described in Equation \ref{equ:EV_aggregation}.
\begin{flalign} 
    &d^\mathrm{EV}_{t,n} = \displaystyle\sum\limits_{\gamma \in \Gamma_{n}} D^\mathrm{EV}_{t,\gamma} + d^\mathrm{EV +}_{t,\gamma} - d^\mathrm{EV -}_{t,\gamma} \label{equ:EV_aggregation} \\ \nonumber
    &\forall t \in T, \forall n \in N
\end{flalign}
A demand increase (Equation \ref{equ:EV_curtailment_up}) and decrease (Equation \ref{equ:EV_curtailment_down}) is only possible when the \ac{CD} is larger than zero, thereby implying that the \ac{EV}s are connected to a charging station. The used \ac{EV}s and the charging infrastructure limit the charging power. On the one hand, the upper power limit ($\mathrm{CAP}^\mathrm{EV max.}$) must not be exceeded. On the other hand, the charging power must not be lower than the minimum power ($\mathrm{CAP}^\mathrm{EV min.}$). These calculations are performed before the optimisation and therefore do not influence the model optimisation time; their implementation as \ac{LP} models is possible.
\begin{flalign} 
    &d^\mathrm{EV +}_{t,\gamma} = 0 ;\quad \forall D^\mathrm{EV}_{t,\gamma} \leq 0 
    \label{equ:EV_curtailment_up}\\ \nonumber
    &0 \leq d^\mathrm{EV +}_{t,\gamma} \leq \mathrm{CAP}^\mathrm{EV max.}_{\gamma} - D^\mathrm{EV}_{t,\gamma} ;\quad \forall D^\mathrm{EV}_{t,\gamma} > 0 \\
    &d^\mathrm{EV -}_{t,\gamma} = 0 ;\quad \forall D^\mathrm{EV}_{t,\gamma} \leq 0
    \label{equ:EV_curtailment_down} \\ \nonumber
    &0 \leq d^\mathrm{EV -}_{t,\gamma} \leq D^\mathrm{EV}_{t,\gamma} - \mathrm{CAP}^\mathrm{EV min.}_{\gamma} ;\quad \forall D^\mathrm{EV}_{t,\gamma} > 0 \\ \nonumber
    &\forall t \in T, \forall \gamma \in \Gamma 
\end{flalign}
Moreover, these equations limit each \ac{EV}'s minimal- and maximal charging power whilst the demand flexibility is in use. The electricity demand of each \ac{CD} is always compensated whilst the vehicle is plugged in, which corresponds to the time frame during which a demand regulation is possible to avoid negative effects on the car owner. Equation \ref{equ:EV_charging_guarantee} ensures that the sum of up and down regulations over the entire 24 hour \ac{DA} optimisation step equals zero. Thus, demand-side flexibility influences the temporal charging power but not charged electricity in total.
\begin{flalign}
    &\displaystyle\sum\limits_{t=j\cdot24+1}^{t=j\cdot24+24} d^\mathrm{EV +}_{t,\gamma} - d^\mathrm{EV -}_{t,\gamma} = 0 \label{equ:EV_charging_guarantee} \\ \nonumber
    &\forall \gamma \in \Gamma, \forall j \in \left\{0,1,...,\frac{T}{24}-1 \right\}
\end{flalign}
\paragraph*{Sector coupling}\mbox{}\\
Conversion technologies are used to connect the electricity- and hydrogen markets. Production ($q$) in one sector causes demand ($d$) in another. Superscripts indicate the target sector, whereas subscripts indicate the source sector.
Electrolysers couple the electricity and hydrogen markets (Equation \ref{equ:electro_conversion}). Hydrogen generation ($q^{H_2}_\mathrm{el}$) causes additional electricity demand ($d^\mathrm{el}_{H_2}$) that must be met (Equation \ref{equ:dispatch_kirchhoff}). 
\begin{flalign}
    &q^{H_2}_{\mathrm{el}_{t,e}} = d^\mathrm{el}_{H_{2_{t,e}}} \cdot \eta_{e} \label{equ:electro_conversion} \\ \nonumber
    &\forall t \in T, \forall e \in E 
\end{flalign}
The coupling from the hydrogen market to the electricity market is accomplished using fuel cells (Equation \ref{equ:fuelcell_conversion}). Electricity generation ($q^\mathrm{el}_{H_2}$) causes hydrogen demand ($d^{H_2}_\mathrm{el}$) that must be met. 
\begin{flalign}
    &q^\mathrm{el}_{H_{2_{t,\mathrm{fc}}}} = d^{H_2}_{\mathrm{el}_{t,\mathrm{fc}}} \cdot \eta_{fc}
    \label{equ:fuelcell_conversion} \\ \nonumber
    &\forall t \in T, \forall \mathrm{fc} \in \mathrm{FC} 
\end{flalign}
Therefore, hydrogen must be produced prior to use. Electrolysers (Equation \ref{equ:CAP_electrolyzer}) and fuel cells (Equation \ref{equ:CAP_fuelcell}) are limited by a specified minimum and maximum electrical power. 
\begin{flalign}
    &0 \leq d^\mathrm{el}_{H_{2_{t,e}}} \leq \mathrm{CAP}_{e} \label{equ:CAP_electrolyzer}\\ \nonumber
    &\forall t \in T, \forall e \in E  \\
    &0 \leq q^\mathrm{el}_{H_{2_{t,\mathrm{fc}}}} \leq \mathrm{CAP}_{fc} \label{equ:CAP_fuelcell}\\ \nonumber
    &\forall t \in T, \mathrm{fc} \in \mathrm{FC} 
\end{flalign}
Hydrogen production and consumption are coupled with hydrogen storage (Equation \ref{equ:H2_storage} - \ref{equ:H2_storage_CAP}). The storage capacity is not limited but evaluated ex-post optimisation.
\begin{flalign}
    &\mathrm{soc}^{H_2}_{t,n} = \mathrm{soc}^{H_2}_{t-1,n} - \sum\limits_{\mathrm{fc} \in \mathrm{FC}_{n}} d^{H_2}_{\mathrm{el}_{t,\mathrm{fc}}} + \sum\limits_{e \in E_{n}} q^{H_2}_{\mathrm{el}_{t,e}} \label{equ:H2_storage}\\
    &0 \leq \mathrm{soc}^{H_2}_{t,n} \leq \infty \label{equ:H2_storage_CAP}\\ \nonumber
    &\forall t \in T, \forall n \in N 
\end{flalign}
\paragraph*{Power exchange}\mbox{}\\
Two different methods for modeling the power exchange between the nodes are implemented. These are used depending on the research question to be answered. 


The first method calculates the power exchange between the nodes (exch) using the transmission line power flow ($\mathrm{flow}$) and an incidence matrix ($A$) (Equation \ref{equ:flow}). This matrix describes which nodes are connected by a specific transmission line. The transmission lines power flows are limited by total \ac{NTC} (Equation \ref{equ:flow1_limit}) \cite{NTCDEF}.
\begin{flalign}
    &\mathrm{exch}_{t,n} = \displaystyle\sum\limits_{l \in L_n} A_{l,n} \cdot \mathrm{flow}_{t,l} \label{equ:flow} \\
    & \forall t \in T, \forall n \in N \nonumber \\
    &-\mathrm{CAP}_l \leq \mathrm{flow}_{t,l} \leq \mathrm{CAP}_l \label{equ:flow1_limit} \\
    &\forall t \in T, \forall l \in L \nonumber
\end{flalign}
The second method calculates the power exchange between the nodes (exch) connected by \ac{AC} transmission lines using a \ac{PTDF} matrix, while \ac{DC} transmission lines are always implemented using the \ac{NTC} approach. This method presupposes that the voltage angle between neighbour nodes is small; thus, a \ac{DC} approximation of the power flow can be used \cite{VanDenBergh2014}. Based on this methodology, the susceptance ($B_{l_\mathrm{AC}}$) of the transmission lines is calculated (Equation \ref{flow1}). 
\begin{flalign}
    &B_{l_\mathrm{AC}} = \frac{-X_{l_\mathrm{AC}}}{R^{2}_{l_\mathrm{AC}} + X^{2}_{l_\mathrm{AC}}} \approx - \frac{1}{X_{l_\mathrm{AC}}} \label{flow1} \\ \nonumber
    &\forall l_\mathrm{AC} \in L_\mathrm{AC} 
\end{flalign}
The resistance ($R_{l_\mathrm{AC}}$) is negligible and the reactance ($X_{l_\mathrm{AC}}$) is considered. Subsequently, the corresponding diagonal matrix ($B_d$) is formed (Equation \ref{flow2}) to calculate the \ac{PTDF} matrix (Equation \ref{flow3}). 
\begin{flalign}
    &B_d =: \mathrm{diag}(B_{L_\mathrm{AC}}) \label{flow2} \\
    &\mathrm{PTDF} = (B_d*A)*(A^T*B_d*A)^{-1} \label{flow3}
\end{flalign}
This describes the relationship between the energy exchange between the nodes (exch), and the load flows over the transmission lines (flow) (Equation \ref{flow4} - \ref{flow5}). 
\begin{flalign}
    &flow_{t,l_\mathrm{AC}} = \displaystyle\sum\limits_{n \in N} \mathrm{PTDF}_{l_\mathrm{AC},n} \cdot 
    \mathrm{exch}_{t,n} \label{flow4} \\ \nonumber
    &\forall t \in T, \forall l_\mathrm{AC} \in L_\mathrm{AC} \\
    &\mathrm{exch}_{t,n} = \displaystyle\sum\limits_{l_\mathrm{AC} \in L_{\mathrm{AC}_n}} A_{l_\mathrm{AC},n} \cdot \mathrm{flow}_{t,l_\mathrm{AC}} \label{flow5} \\ \nonumber
    &\forall t \in T, \forall n \in N 
\end{flalign}
These transmission lines are constrained by their technical capacity (Equation \ref{equ:flow6}). Because the distribution of the power flow on the individual lines is known, the total tradeable capacity between the two bidding zones can increase compared to the \ac{NTC} approach \cite{Jegleim2015}. 
\begin{flalign}
    &-\mathrm{CAP}_l \leq \mathrm{flow}_{t,l_\mathrm{AC}} \leq \mathrm{CAP}_l \label{equ:flow6}\\
    &\forall t \in T, \forall n \in N, \forall l_\mathrm{AC} \in L_\mathrm{AC} \nonumber
\end{flalign}
If a subset of nodes is used the transmission lines and their transfer capacities are aggregated. Hence, the information regarding their spatial distribution is lost but the transmission lines are limited by their total capacity \cite{NTCDEF}.
\paragraph*{Other constraints}\mbox{}\\
Thermal power plants are limited by a ramp rate and technical minimum and maximum capacities, which are implemented as a linear function to consider start-up costs \cite{Farahmand2012}.
The power generation of \ac{RESE} technologies is represented by a yearly profile using hourly temporal resolution and installed capacity per node. 
Storages are power- and capacity limited. Furthermore, \ac{PHS} must follow an annual pattern to map realistic operating behaviour as a long-term storage unit. In addition, \ac{PHS} have a natural inflow that increases its \ac{SOC} representing rainfall and meltwater.
\subsection{Redispatch model}\label{sec:model_redispatch}

After the dispatch model has minimised the overall electricity generation costs with an aggregated set of nodes ($N$) and transmissions lines (Figure \ref{fig:market_area}), these nodes are disaggregated (Figure \ref{fig:Nodes_redispatch}) ($N^\mathrm{All}$) before performing the cost minimal redispatch in Austria. Thus, the differences between the market-based optimum and the actual physical power flows that may lead to congested transmission lines and must be compensated are analysed.
The previously dispatched power plants are assigned to the respective nodes using their existing generation schedules. The decision variables of the dispatch remain unchanged, as parameters, during the redispatch optimisation. 
The redispatch needs are balanced cost-minimal by new decision variables that adjust the generation of thermal power plants, additional curtail \ac{RESE} and use the \ac{EV} demand-side flexibility. 

\begin{figure}[ht]
    \centering
    \includegraphics[width=0.48\textwidth]{graphics/RQ3/Nodes_redispatch.jpg}
    \caption{Disaggregated nodes and transmission lines within the redispatch}
    \label{fig:Nodes_redispatch}
\end{figure}

\subsubsection{Objective function}
The modification of generation schedules to balance redispatch needs is associated with costs. Costs that arise due to the increase in thermal power plant generation must be compensated. The costs of a thermal generation increase ($C^\mathrm{th Redis}$) as redispatch measure is the same as the local \ac{DA} market price ($P^\mathrm{CP}$). If the \ac{SRMC} of the generation unit is higher than this price, then these costs must be compensated (Equation \ref{equ:redis_thermal_costs}). Thermal power plant generation decrease leads to refund proportional to the respective \ac{SRMC}. 
The dispatch model is based on the merit-order function. Consequently, the \ac{DA} market price will be zero or even negative during periods with a remarkably high share of \ac{RESE}. Redispatch costs are also calculated for the curtailment of \ac{PV}, wind and \ac{RoR} ($C^\mathrm{RES Redis}$) to ensure the economic participation in the redispatch market for \ac{RESE} from the operator viewpoint. These costs are either the fixed market premium ($P^\mathrm{Premium}$) for the specific technology or the local \ac{DA} price ($P^\mathrm{CP}$) if this is higher (\ref{equ:redis_RESE_costs}).
\begin{flalign} 
    &C^\mathrm{th Redis}_{t,\mathrm{th}} = \mathrm{max}\left(P^\mathrm{CP}_{t,n_{th}}, \mathrm{SRMC}_\mathrm{th}\right)
    \label{equ:redis_thermal_costs} \\ \nonumber
    &\forall t \in T, \forall \mathrm{th} \in \mathrm{TH} \\
    &C^\mathrm{RES Redis}_{t,n} = \mathrm{max}\left( P^\mathrm{CP}_{n}, P^\mathrm{Premium}_{n}\right) \label{equ:redis_RESE_costs} \\ \nonumber  
    &\forall t \in T, \forall n \in N^\mathrm{All}
\end{flalign} 

The redispatch costs, that are minimised for Austria (Equation \ref{equ:redispatch_objective_function}), arise from costs of thermal power plant generation schedule modification (generation increase: $q^\mathrm{+ Redis}_{t,\mathrm{th}}$, generation decrease: $q^\mathrm{- Redis}_{t,th}$) and \ac{RESE} curtailment ($\mathrm{spill}^\mathrm{RESE}_\mathrm{Redis}$). \ac{EV} demand regulation does not lead to additional costs.
\begin{flalign}
    &\underset{q^\mathrm{+ Redis}, q^\mathrm{- Redis}, \mathrm{spill}^\mathrm{RESE}_\mathrm{Redis}}{\mathrm{min}}
    \label{equ:redispatch_objective_function} \\ \nonumber
    &\sum_{t \in T} \sum_{\mathrm{th} \in \mathrm{TH}_\mathrm{AT}} \left( q^\mathrm{+ Redis}_{t,\mathrm{th}}  \cdot C^\mathrm{th Redis}_{t,\mathrm{th}} 
        - q^\mathrm{- Redis}_{t,\mathrm{th}}  \cdot \mathrm{SRMC}_\mathrm{th} \right)\\  \nonumber
    &+ \sum_{t \in T}\sum_{n \in N^\mathrm{All}_\mathrm{AT}}
        \mathrm{spill}^\mathrm{RESE}_\mathrm{Redis_{t,n}} \cdot C^\mathrm{RES Redis}_{t,n}
\end{flalign}

\subsubsection{Constraints}
\paragraph*{Demand equilibrium - Austria}\mbox{}\\
The demand compensation constraint (Equation \ref{equ:dispatch_kirchhoff}) is expanded by the additional redispatch decision variables (Equation \ref{equ:redispatch_kirchhoff}). The generation schedules are taken over from the dispatch; thus, the redispatch regulation must be used if necessary. This constraint is only used within Austria because the redispatch evaluation is conducted in this country. In contrast to the market-based dispatch, all transmission lines and nodes in Austria are considered ($N^\mathrm{All}_\mathrm{AT}$). The nodes are still aggregated in all other countries; therefore, no local congestion management is necessary. However, redispatch measures within Austria will marginally change the power flow over the whole market area and all transmission lines.
\begin{flalign} \label{equ:redispatch_kirchhoff}
&D_{t,n} + D^\mathrm{up}_{t,n} - D^\mathrm{down}_{t,n} + d^\mathrm{EV}_{t,n} + \displaystyle\sum\limits_{e \in E_{n}} D^\mathrm{el}_{H_{2_{t,e}}} = \\ \nonumber
&\displaystyle\sum\limits_{\mathrm{th} \in \mathrm{TH}_{n}} Q_{t,\mathrm{th}} + q^\mathrm{+ Redis}_{t,\mathrm{th}} - q^\mathrm{- Redis}_{t,\mathrm{th}} + \displaystyle\sum\limits_{\mathrm{fc} \in \mathrm{FC}_{n}} Q^\mathrm{el}_{H_{2_{t,\mathrm{fc}}}} \\ \nonumber
&+ \displaystyle\sum\limits_{\mathrm{ps} \in \mathrm{PS}_{n}} (Q^\mathrm{tu}_{t,\mathrm{ps}} - Q^\mathrm{pu}_{t,\mathrm{ps}}) + \displaystyle\sum\limits_{\mathrm{st} \in \mathrm{ST}_{n}} (Q^\mathrm{out}_{t,\mathrm{st}} - Q^\mathrm{in}_{t,\mathrm{st}}) \\ \nonumber 
&+ Q^\mathrm{RESE}_{t,n} - \mathrm{Spill}^\mathrm{RESE}_{t,n} - \mathrm{spill}^\mathrm{RESE}_\mathrm{Redis_{t,n}} - \mathrm{exch}_{t,n} + \mathrm{NSE}_{t,n} \\ \nonumber
&\forall t \in T, \forall n \in N^\mathrm{All}_\mathrm{AT} 
\end{flalign}
\paragraph*{Electricity supply - Austria}\mbox{}\\
The sum of generated electricity and \ac{EV} demand regulation within Austria is constant before and after the congestion management (Equation \ref{equ:redis_energy_equilibrium}).
\begin{flalign} \label{equ:redis_energy_equilibrium}
    &\displaystyle\sum\limits_{n \in N^\mathrm{All}_\mathrm{AT}} d^\mathrm{EV +}_{t,n} + \displaystyle\sum\limits_{\mathrm{th} \in \mathrm{TH}_{N^\mathrm{All}_\mathrm{AT}}} q^\mathrm{- Redis}_{t,\mathrm{th}} 
     + \displaystyle\sum\limits_{n \in N^\mathrm{All}_\mathrm{AT}} \mathrm{spill}^\mathrm{RESE}_{\mathrm{Redis}_{t,n}} \\ \nonumber
     &= \displaystyle\sum\limits_{n \in N^\mathrm{All}_\mathrm{AT}} d^\mathrm{EV -}_{t,n} + \displaystyle\sum\limits_{\mathrm{th} \in \mathrm{TH}_{N^\mathrm{All}_\mathrm{AT}}} q^\mathrm{+ Redis}_\mathrm{t,th} \\ \nonumber 
     &\forall t \in T
\end{flalign}
\paragraph*{Electricity supply - other control areas}\mbox{}\\
All other control areas expect Austria are restrictively aggregated as a sum for their thermal power plant generation (Equation \ref{equ:redis_energy_equilibrium_other}). 
Consequently, these areas can compensate for the differing exchanges over the interconnectors created by congestion management in Austria. The exchanges with Austria are in general consistent with the ones determined in the dispatch, but other transmission lines are possibly used.
\begin{flalign} \label{equ:redis_energy_equilibrium_other}
&\sum_{\mathrm{th} \in \mathrm{TH}_{n}} q^\mathrm{- Redis}_{t,\mathrm{th}}
= \sum_{\mathrm{th} \in \mathrm{TH}_{n}} q^\mathrm{+ Redis}_{t,\mathrm{th}} \\ \nonumber
&\forall t \in T, \forall n \in N^\mathrm{All}_\mathrm{\setminus AT} 
\end{flalign}
\paragraph*{Generation regulation - thermal power plants}\mbox{}\\
The generation of the thermal power plants can be increased (Equation \ref{equ:redis_thermal_1}) or decreased (Equation \ref{equ:redis_thermal_2}) based on the generation schedules in the dispatch. However, neither the limits of their technical capacity nor the power gradient may be exceeded.
\begin{flalign} \label{equ:redis_thermal_1}
    &q^\mathrm{+ Redis}_{t,\mathrm{th}} \geq 0 \\ \label{equ:redis_thermal_2}
    &q^\mathrm{- Redis}_{t,\mathrm{th}} \geq 0 \\ \nonumber
    &\forall t \in T, \forall \mathrm{th} \in \mathrm{TH} 
\end{flalign}
\paragraph*{Generation regulation - renewable energies}\mbox{}\\
The curtailment of \ac{RESE} as a redispatch measure (Equation \ref{equ:redis_RESE_curtail}) must consider the curtailment that was already performed in the dispatch.
\begin{flalign} \label{equ:redis_RESE_curtail}
    &0 \leq \mathrm{spill}^\mathrm{RESE}_{\mathrm{Redis}_{t,n}} \leq Q^\mathrm{RESE}_{t,n} - \mathrm{Spill}^\mathrm{RESE}_{t,n} \\
    &\forall t \in T, \forall n \in N^\mathrm{All} \nonumber
\end{flalign}
The use of the storage units, sector coupling and \ac{NSE}, remain unchanged as planned in the dispatch. Readjusting them within the redispatch is not implemented as part of this work. The nodes' exchanges are calculated again considering the new disaggregated transmission grid.
\subsection{Socio-economic welfare calculation}\label{sec:welfare_calculation}

This section introduces a calculation scheme into the model to evaluate the effects of flexibility integration on system-wide \ac{SEW}. This is evaluated ex-post optimisation. The overall \ac{SEW} change is calculated to investigate the influence of added flexibility. A reference scenario is required for this calculation method. \ac{SEW} corresponds to the sum of the \ac{PS} and the \ac{CS} of electricity and hydrogen markets, the \ac{CR}, and other costs (Equation \ref{equ:delta_SEW}). Hydrogen market \ac{CS} is zero because the price bid for hydrogen is inelastic and unaffected by the amount of hydrogen produced. This implies that the \ac{WTP} for hydrogen mirrors the bid price for hydrogen.
\begin{flalign} \label{equ:delta_SEW}
    \Delta \mathrm{SEW} &= \Delta \mathrm{SEW}^\mathrm{el} + \Delta \mathrm{SEW}^{H_2} - \Delta \mathrm{costs}^\mathrm{el} + \Delta \mathrm{CR}^\mathrm{el}\\
    &= \Delta \mathrm{PS}^\mathrm{el}  + \Delta \mathrm{CS}^\mathrm{el}  + \Delta \mathrm{PS}^{H_2} + \Delta \mathrm{CS}^{H_2} - \Delta \mathrm{costs}^\mathrm{el} + \Delta \mathrm{CR}^\mathrm{el}\nonumber \\
    &= \Delta \mathrm{PS}^\mathrm{el}  + \Delta \mathrm{CS}^\mathrm{el}  + \Delta \mathrm{PS}^{H_2} - \Delta \mathrm{costs}^\mathrm{el} + \Delta \mathrm{CR}^\mathrm{el}\nonumber 
\end{flalign}
The electricity market \ac{CS} ($\mathrm{CS}^\mathrm{el}$) is determined by the inelastic and flexible electricity demand and the corresponding prices. The actual electricity demand is multiplied by the difference between \ac{WTP} and the resulting market clearing price (Equation \ref{equ:CS_calc}). The electricity demand includes constant and variable elements that are affected by demand-side flexibility. Additional electricity demand from the electrolyser multiplied by the difference between the price bid on the electricity market ($P^\mathrm{bid}_\mathrm{el}$) and the resulting market clearing price produces a surplus.
\begin{flalign}
\mathrm{CS}^\mathrm{el} = &\left(D^\mathrm{el} + d^\mathrm{up} - d^\mathrm{down} + d^\mathrm{EV} \right) \cdot \left(\mathrm{WTP} - p^\mathrm{CP}\right) \label{equ:CS_calc}\\ \nonumber
&+ d^\mathrm{el}_{H_2} \cdot \left(P^\mathrm{bid}_\mathrm{el} - p^\mathrm{CP}\right) 
\end{flalign}
The price bid on the electricity market results from the price bid on the hydrogen market and the electrolyser efficiency (Equation \ref{equ:CS_pricebid}). 
\begin{flalign}
& P^\mathrm{bid}_\mathrm{el} = P^\mathrm{bid}_{H_2} \cdot \eta_{e} \label{equ:CS_pricebid}
\end{flalign}
The \ac{PS} on the electricity market is calculated by the electricity generated through thermal- and renewable generation units multiplied by the market clearing price minus generation costs plus \ac{PHS} and battery storage \ac{PS} (Equation \ref{equ:PS_el}). The latter is calculated by multiplying the charged and discharged electricity with the market clearing price.
\newpage
\begin{flalign}
\mathrm{PS}^\mathrm{el} = 
&\sum_{\mathrm{th} \in \mathrm{TH}} q_\mathrm{th} \cdot \left(p^\mathrm{CP} - \mathrm{SRMC}_\mathrm{th} \right) \label{equ:PS_el} \\
+ &\sum_{r \in \mathrm{RESE}} \left( Q^{\mathrm{RESE}}_{r} - \mathrm{spill}^{\mathrm{RESE}}_{r} \right) \cdot \left(p^\mathrm{CP} - C^{\mathrm{RESE}} \right)\nonumber \\
+ & \sum_{\mathrm{ps} \in \mathrm{PS}} q^\mathrm{tu}_\mathrm{ps} \cdot \left(p^\mathrm{CP} - C^\mathrm{ps}\right) - q^\mathrm{pu}_\mathrm{ps} \cdot p^\mathrm{CP}
+ \sum_{\mathrm{st} \in \mathrm{ST}} q^\mathrm{out}_\mathrm{st} \cdot p^\mathrm{CP} - q^\mathrm{in}_\mathrm{st} \cdot p^\mathrm{CP} \nonumber
\end{flalign}
On the hydrogen market a \ac{PS} is generated by electrolysers and fuel cells (Equation \ref{equ:PS_H2_electrolyser}). The electrolyser \ac{PS} is given by the amount of hydrogen produced multiplied by the difference between the hydrogen price bid and the clearing price on the hydrogen market. The fuel cell \ac{PS} is equivalent to that of thermal power plants, whereby their \ac{SRMC} are defined by the hydrogen price bid and the fuel cell efficiency.
\begin{flalign}
\mathrm{PS}^{H_2} = &q^{H_2}_\mathrm{el} \cdot \left( P^\mathrm{bid}_{H_2} - p^\mathrm{CP}_{H_2}\right)\label{equ:PS_H2_electrolyser}  \\ \nonumber
+ &q^\mathrm{el}_{H_2} \cdot \left( p^\mathrm{CP} - \frac{P^\mathrm{bid}_{H_2}}{\eta_\mathrm{fc}}\right)   
\end{flalign}
The clearing price on the hydrogen market ($p^\mathrm{CP}_{H_2}$) results from the clearing price on the electricity market and the electrolyser efficiency (Equation \ref{equ:Clearingprice_H2}). 
\begin{flalign}
p^\mathrm{CP}_{H_2} = \frac{p^\mathrm{CP}}{\eta_{e}} \label{equ:Clearingprice_H2}
\end{flalign}
Transmission line \ac{CR} is calculated by multiplying the power flow over a transmission line by the corresponding absolute value of the price difference between two nodes (Equation \ref{equ:CR}). 
\begin{flalign}
\mathrm{CR}^\mathrm{el} = 
\sum_{l \in L} |\mathrm{flow}_{l} \cdot \left(p^{\mathrm{CP}_{n_\mathrm{one}}} - p^{CP_{n_\mathrm{two}}} \right)| \label{equ:CR}
\end{flalign}
Other electricity costs (Equation \ref{equ:other_costs}) include all costs that are part of the objective function but are not part of the \ac{PS}, \ac{CS}, and \ac{CR}. These are start-up costs of thermal power plants ($C^{\mathrm{start}}$), cost for demand-side flexibility / demand response ($C^\mathrm{DSM}$), costs for \ac{NSE} ($\mathrm{VoLL}$), and \ac{PHS} missing inflow costs ($C^{\mathrm{MissingInflow}}$).
\begin{flalign} \label{equ:other_costs}
\mathrm{costs}^\mathrm{el} &= 
\mathrm{str} \cdot C^{\mathrm{start}} + 
d^\mathrm{down} \cdot C^\mathrm{DSM} \\ \nonumber
&+ \mathrm{nse} \cdot \mathrm{VoLL} +
p^\mathrm{missingInflow} \cdot C^{\mathrm{MissingInflow}}
\end{flalign}
\subsection{Nomenclature}
An overview of all sets, parameters and decision variables is given in Table \ref{table:nomenclature}.
\begin{longtable}{l|l|l} 
\multicolumn{3}{l}{\textbf{Sets}}     \endfirsthead
$E$ & Electrolysers & index: $e$ \\
$\mathrm{FC}$ & Fuel cells & index: $\mathrm{fc}$ \\
$L$ & All transmission lines & index: $l_\mathrm{AC}$ \\
$L_\mathrm{AC}$ & AC transmission lines & index: $l_\mathrm{AC}$ \\
$N$     & Reduced set of nodes   & index: $n$        \\ 
$N^\mathrm{All}$     & Complete set of nodes   & index: $n$        \\ 
$\mathrm{PS}$    & Hydro storage units          & index: $\mathrm{ps}$        \\ 
$\mathrm{RESE}$ & Renewable energy sources & index: $r$ \\
$\mathrm{ST}$  & Battery storage units & index: $\mathrm{st}$   \\ 
$T$      & Time steps  & index: $t$         \\ 
$\mathrm{TH}$  & Thermal power plants  & index: $\mathrm{th}$        \\ 
$\Gamma$ & Charging demands & index: $\gamma$ \\

\multicolumn{2}{l}{}       &       \\
\multicolumn{2}{l}{\textbf{Parameters}}   &       \\
$A_{l,n}$ & Transmission line and node incidence matrix & (0;1) \\
$B_l$ & Transmission line susceptance & $1/\Omega$  \\
$B_d$ & Susceptance diagonal matrix & 1/$\Omega$  \\
$C^\mathrm{DSM}$ & Demand-side response costs & \euro{}/MWh \\
$C^\mathrm{ps}$               & Pumped hydro-power storage turbine costs  & \euro{}/MWh            \\ 
$C^{\mathrm{MissingInflow}}$ & Missing inflow costs  & \euro{}/MWh            \\ 
$C^\mathrm{RESE}$             & Renewable energy maintenance costs & \euro{}/MWh            \\ 
$C^\mathrm{start}$                & Start-up costs    & \euro{}                \\ 
$\mathrm{CAP}^\mathrm{EV max.}$  & Electric vehicle's maximum charging power & MW    \\ 
$\mathrm{CAP}^\mathrm{EV min.}$  & Electric vehicle's minimum charging power & MW   \\
$\mathrm{CAP}_l$ & Transmission line's capacity & MW \\
$\mathrm{CAP}_{e}$  & Maximum electrical electrolyser power & MW   \\
$\mathrm{CAP}_\mathrm{fc}$  & Maximum electrical fuel cell power & MW    \\ 
$D^\mathrm{EV}_{\gamma}$  & Electricity demand of charging profile $\gamma$  & MWh   \\ 
$D_{n}$       & Electricity demand  & MWh   \\
$P^\mathrm{bid}_\mathrm{el}$ & Electricity price bid to produce hydrogen & \euro{}/MWh \\
$P^\mathrm{bid}_{H_2}$ & Hydrogen market price & \euro{}/MWh\textsubscript{H\textsubscript{2}} \\
$Q^\mathrm{RESE}$            & Renewable energy generation  & MWh              \\ 
$R_l$ & Transmission line resistance & $\Omega$  \\
$\mathrm{SRMC}$                     & Short run marginal costs          & \euro{}/MWh            \\
$\mathrm{VoLL}$                            & Value of lost load    & \euro{}/MWh            \\
$\mathrm{WTP}$                            & Willingness to pay for electricity & \euro{}/MWh            \\
$X_l$ & Transmission line reactance & $\Omega$  \\
$\eta_{e}$ & Electrolyser efficiency & \% \\
$\eta_\mathrm{fc}$ & Fuel cell efficiency & \% \\

\multicolumn{2}{l}{}       &       \\
\multicolumn{3}{l}{\textbf{Parameters - Redispatch model}} \\
$C^\mathrm{RES Redis}$ & Renewable energy curtailment costs & \euro{}/MWh \\ 
$C^\mathrm{th Redis}$ & Generation increase costs & \euro{}/MWh \\ 
$P^\mathrm{Premium}$ & Market premium & \euro{}/MWh \\

\multicolumn{2}{l}{}       &       \\
\multicolumn{3}{l}{\textbf{Decision Variables}} \\
$d^\mathrm{down}$ & Demand-side response - demand reduction & MWh \\
$d^\mathrm{el}_{H_2}$ & Electrolysers' electricity demand & MWh  \\
$d^\mathrm{EV}$       & Demand of all electric vehicles & MWh   \\ 
$d^\mathrm{EV}_{\gamma}$  & Demand of charging profile $\gamma$  & MWh   \\ 
$d^\mathrm{EV +}_{\gamma}$  & Demand increase of charging profile $\gamma$   & MWh  \\ 
$d^\mathrm{EV -}_{\gamma}$  & Demand decrease of charging profile $\gamma$    & MWh \\
$d^{H_2}_\mathrm{el}$ & Hydrogen demand to generate electricity & MWh\textsubscript{H\textsubscript{2}} \\
$d^\mathrm{up}$ & Demand-side response - demand increase & MWh \\
$\mathrm{exch}$       & Power exchange between nodes & MWh              \\ 
$\mathrm{flow}$ & Power flow over a transmission line & MWh \\
$p^\mathrm{CP}$ &  Electricity market clearing price & \euro{}/MWh \\
$p^\mathrm{CP}_{H_2}$ &  Hydrogen market clearing price & \euro{}/MWh\textsubscript{H\textsubscript{2}} \\
$q^\mathrm{el}_{H_2}$ & Fuel cell electricity generation & MWh \\
$q^{H_2}_\mathrm{el}$ & Hydrogen generation & MWh\textsubscript{H\textsubscript{2}} \\
$q^\mathrm{in}$        & Storage charging power& MWh              \\ 
$q^\mathrm{missingInflow}$ & Missing Inflow & MWh \\
$q^\mathrm{out}$      & Storage discharging power& MWh              \\ 
$q^\mathrm{pu}$   & Pumped hydro-power storage pump mode & MWh    \\ 
$q_\mathrm{th}$   & Thermal power plant electricity generation & MWh   \\
$q^\mathrm{tu}$    & Pumped hydro-power storage turbine mode & MWh \\ 
$\mathrm{soc}^\mathrm{DSR}$ & Demand-side response state of charge & MWh \\
$\mathrm{soc}^{H_2}$ & Hydrogen storage state of charge & MWh\textsubscript{H\textsubscript{2}} \\
$\mathrm{spill}^\mathrm{RESE}$     & Renewable energy curtailment    & MWh              \\ 
$\mathrm{str}_\mathrm{th}$    & Thermal power plant start-up   & (0;1)            \\ 
$\mathrm{nse}$       & Not supplied energy  & MWh \\

\multicolumn{2}{l}{}       &       \\
\multicolumn{3}{l}{\textbf{Decision Variables - Redispatch model}} \\
$q^\mathrm{+ Redis} $      & \begin{tabular}[c]{@{}l@{}}Electricity generation increase\end{tabular}                & MWh  \\ 
$q^\mathrm{- Redis} $      & \begin{tabular}[c]{@{}l@{}}Electricity generation decrease\end{tabular}                & MWh  \\ 
$\mathrm{spill}^\mathrm{RESE}_\mathrm{Redis}$   & Additional renewable energy curtailment & MWh  \\ 

\multicolumn{2}{l}{}       &       \\
\multicolumn{3}{l}{\textbf{Calculation terms - Socio-economic welfare}} \\
$\mathrm{costs}^\mathrm{el}$ & Other cost term & \euro{} \\
$\mathrm{CR}^\mathrm{el}$ & Congestion rent & \euro{}\\
$\mathrm{CS}^\mathrm{el}$ & Electricity market consumer surplus & \euro{}\\
$\mathrm{CS}^{H_2}$ & Hydrogen market consumer surplus & \euro{}\\
$\mathrm{PS}^\mathrm{el}$ & Electricity market producer surplus & \euro{}\\
$\mathrm{PS}^{H_2}$ & Hydrogen market producer surplus & \euro{}\\
$\mathrm{SEW}^\mathrm{el}$ & Electricity market socio-economic welfare & \euro{}\\
$\mathrm{SEW}^{H_2}$ & Hydrogen market socio-economic welfare & \euro{}
\\ \hline
\caption{Model sets, parameters and decision variables}
\label{table:nomenclature}
\end{longtable}
\section{Case study and scenarios}\label{sec:methods_scenarios}
In order to answer this thesis's three main research questions, the described model uses different configurations and datasets for each of them. 
\subsection{Flexibility competition and synergy}
This section describes the case study that evaluates synergistic and competitive effects on \ac{SEW} between several flexibility options. For this purpose, different combinations of flexibility options are implemented using several scenarios. These options include storage expansion, transmission line expansion, \ac{EV} demand-side flexibility, and hydrogen sector coupling. These provide different system benefits constrained by each option's features. Flexibility characteristics of each option are presented in Table \ref{table:flexibility_character}. 

\begin{table}[h]
\centering
\begin{tabular}{llll}
\hline
\textbf{Flexibility option}  & \textbf{Energy}  & \textbf{Power} & \textbf{Duration} \\ \hline
Transmission line & not limited                   & limited & not limited       \\ \hline
Demand-side flexibility     & constant                      & limited & hours         \\ \hline
Sector coupling             & \begin{tabular}[c]{@{}l@{}}increases demand\end{tabular}  & limited  & not limited \\ \hline
Storage     & \begin{tabular}[c]{@{}l@{}}limited; increases demand\\due to conversion losses\end{tabular}                   & limited  & \begin{tabular}[c]{@{}l@{}}hourly\\to seasonal \end{tabular}     
\\ \hline
\end{tabular}
\caption{Flexibility characteristics}
\label{table:flexibility_character}
\end{table}
All 13 countries (Austria, Germany, Netherlands, Belgium, Luxembourg, Czech Republic, Slovenia, Switzerland, Poland, Slovakia, Hungary, Italy, and France) within the optimized area are expected to fulfil the European \ac{NECP} \cite{NECP2021} by 2030. This includes for example expanding \ac{RESE} and decommissioning coal-fired power plants. According to Austria's \ac{NECP}, \ac{RESE} compensates for 100\% of the yearly electricity demand. The used \textit{\ac{NT} 2030} dataset, provided by the \ac{ENTSOE}, is based on the fulfilment of these plans \cite{TYNDP2022}. It includes a power plant fleet with \ac{RESE}, fossil fuel-fired power plants (coal, oil, lignite and \ac{CCGT}) and other types of thermal power plants that do not emit CO\textsubscript{2} (nuclear, biogas and biomass).
\subsubsection{Scenarios}
A reference scenario is used to evaluate the changes in \ac{SEW}, \ac{PS}, \ac{CS}, and \ac{CR} of the whole system based on the integrated flexibility options. All scenarios evaluated for 2030 are listed in Appendix \ref{sec:scenario_overview} and described as follows:
\begin{itemize}
 \setlength\itemsep{-0.5em}
    \item The \textit{Baseline-1} and \textit{Baseline-2} serve as reference scenarios for the examined scenarios. The \textit{Baseline-1} scenario uses the current transmission grid, including a 4 900 MW \ac{NTC} between Austria and Germany (bold line in Figure \ref{fig:market_area}). This grid between Austria and Germany is expanded in several scenarios through the implementation of \ac{ENTSOE} \ac{TYNDP} 2022 projects TR47, TR187, and TR313 \cite{TYNDP2022}. These provide an overall capacity expansion of 4 100 MW. Based on these plans, the \textit{Baseline-2} scenario considers a total capacity of 9 000 MW. The case study between Austria and Germany's bidding zones was chosen as a representative European example. Centrally located in Europe, Austria has about 9 GW of turbine capacity from hydropower plants and \ac{PHS} due to the alpine arc. This compares to an annual electricity demand of 80 TWh in the chosen scenario. Conversely, Germany has an electricity demand of 584 TWh, which is more than seven times higher, but it has approximately the same turbine capacity of 10.3 GW. In addition, Germany possesses nearly 100 GW of installed wind power capacity. Thus, in windy periods, there is excess generation that should be exported or stored. Expanding the interconnectors between the two bidding zones can massively promote the storing and use of electricity generated by wind turbines in Europe. 
    \item A single storage expansion is evaluated in the scenarios \textit{Storage Austria-1} and \textit{Storage Germany-1}. Data published in the \ac{TYNDP} are used to apply grid and storage expansion with similar capital investment costs \cite{TYNDP2022}. This results in a representative \ac{PHS} with 700 MW pump- and turbine power. 7 GWh capacity is assumed as we use a typical relationship between capacity-to-power relation of 10 \cite{Diawuo2022}. In addition to storage, the \textit{Storage Austria-2} and \textit{Storage Germany-2} scenarios introduce interconnector expansion.
    \item The \ac{EV} demand-side flexibility combined with interconnector expansion is evaluated in the \textit{EVDSM only} scenario. The \textit{EVDSM Storage} scenario additionally includes the storage expansion. \ac{EV} demand-side flexibility does not change the overall electricity demand, but introduces the possibility to temporal shift demand. Hence, the benefits of the flexibility are evaluated. The 0.75 million \ac{EV}s charge using 11 kW of power. Overall, around 5.15 million passenger cars operate in Austria \cite{KFZAustria}. Thus, the electrification rate of Austria's private individual transport sector is 15\% \cite{Statharas2019}. The charging strategies prior to introducing flexibility are divided into 30\% charge immediately after the plug-in (\textit{immediately}) and 70\% charge with the lowest possible power to meet electricity demand (\textit{peak-shaving}).
    \item Sole sector coupling to the hydrogen market is evaluated in the \textit{Hydrogen only} scenario. In contrast, the \textit{Hydrogen Storage} scenario adds storage expansion and the \textit{Hydrogen EVDSM} scenario adds the \ac{EV} demand-side flexibility. A hydrogen price of 61.67 \euro{}/MWh and a $CO_2$ price of 70 \euro{}/t are assumed based on the \ac{TYNDP}2022 \ac{NT} dataset \cite{TYNDP2022}.
    \item The combined use of storage expansion and \ac{EV} demand-side flexibility is evaluated in the \textit{Flexibility-1 w/o H2} and \textit{EVDSM Storage} scenarios. Whereas the former does not include transmission grid expansion. All flexibility options are implemented in the \textit{Flexibility-1} scenario without transmission line expansion and in the \textit{Flexibility-2} scenario with transmission line expansion. 
\end{itemize}
\subsubsection{Model setup}

Figure \ref{fig:market_area_RQ1} illustrates the nodes and interconnectors used to answer this research question. The transmission line expansion in several scenarios is marked as a thick black line. The dispatch model uses all model components and flexibility options. This includes hydrogen production, storage and reconversion to electricity. The power exchange is calculated using the \ac{NTC} approach with one node per country. The redispatch calculation is not performed in this use case. 
\begin{figure}[ht]
\centering
      \includegraphics[width=0.48\textwidth]{graphics/RQ1/Europakarte_EDisOn.jpg}
      \caption{Nodes and grid of the electricity market model}
      \label{fig:market_area_RQ1}
\end{figure}
\FloatBarrier
\subsection{Hydrogen sector coupling}
This section describes the case study that evaluates the effects of the sector coupling between the European electricity and national hydrogen markets in detail. All 13 countries (Austria, Germany, Netherlands, Belgium, Luxembourg, Czech Republic, Slovenia, Switzerland, Poland, Slovakia, Hungary, Italy, and France) within the optimized area are expected to fulfil the European \ac{NECP} \cite{NECP2021} by 2030. This includes for example expanding \ac{RESE} and decommissioning coal-fired power plants. According to Austria's \ac{NECP}, \ac{RESE} compensates for 100\% of the yearly electricity demand. The used \textit{\ac{NT} 2030} and \textit{\ac{NT} 2040} scenarios provided by the \ac{ENTSOE} are based on the fulfilment of these plans \cite{TYNDP2022_scenarios}. The \ac{SRMC} of thermal power plants include a CO\textsubscript{2} price of 70 \euro{}/t in 2030 and 90 \euro{}/t in 2040.
\subsubsection{Scenarios}\label{sec:results_hydrogen_data}
The electrolyser and fuel cell capacities are based on the individual countries’ national hydrogen strategies. All installed electrolyser and fuel cell capacities are listed in Appendix \ref{sec:scenario_overview_RQ2}. As a result, the overall electrolyser capacity in 2030 is 26 GW and therefore slightly less than the European target of 40 GW \cite{su132313464}. All scenarios evaluated for 2030 and 2040 are listed in Appendix \ref{sec:scenario_overview_RQ2} and described as follows:
\begin{itemize}
 \setlength\itemsep{-0.5em}
    \item Each optimization is conducted with 8760 one-hour timesteps and is done for the years 2030 (\textit{2030}) and 2040 (\textit{2040}).
    \item  For both years, a distinction is made between using only electrolysers (\textit{Production}) or electrolysers, fuel cells and storage units (\textit{Complete}). This distinction allows to separately evaluate the impact of the two components, demand-side flexibility and seasonal storage. This results in four scenarios.
    \item Each scenario is optimised for a range of hydrogen prices as a sensitivity analysis (Table \ref{table:cost_assumptions}). Hence, five different hydrogen prices (\textit{0},\textit{50},\textit{100},\textit{150}, and \textit{200}) are assumed.
\end{itemize}
The \textit{Production-2030-0} and \textit{Production-2040-0} scenarios, with a hydrogen prices of 0 \euro{}/MWh\textsubscript{H\textsubscript{2}}, serve as a reference scenario for all other scenarios. There is no incentive to produce hydrogen in these scenarios. On the one hand, additional \ac{PS} or \ac{CS} cannot be obtained (Section \ref{sec:welfare_calculation}). On the other hand, there are no fuel cells to stimulate hydrogen production through sufficient electricity price fluctuations.
\subsubsection{Model setup}
Figure \ref{fig:market_area_RQ2} illustrates the nodes and interconnectors used to answer this research question. The dispatch model uses the hydrogen sector coupling as a flexibility option. This includes hydrogen production, storage and reconversion to electricity. \ac{EV}s and chargeable demand-side flexibility are not part of these evaluations. The power exchange is calculated using the \ac{NTC} approach with one node per country. The redispatch calculation is not performed in this use case. 
\begin{figure}[ht]
\centering
      \includegraphics[width=0.48\textwidth]{graphics/RQ2/Europakarte_EDisOn.jpg}
      \caption{Nodes and grid of the electricity market model}
      \label{fig:market_area_RQ2}
\end{figure}
\FloatBarrier
\subsection{Redispatch}
This section describes the case study that evaluates the influence of customer integration as aggregated \ac{EV} fleet demand-side flexibility on congestion and needed redispatch measures. All 13 countries (Austria, Germany, Netherlands, Belgium, Luxembourg, Czech Republic, Slovenia, Switzerland, Poland, Slovakia, Hungary, Italy, and France) within the optimized area are expected to fulfil the European \ac{NECP} \cite{NECP2021} by 2030. This includes for example expanding \ac{RESE} and decommissioning coal-fired power plants. This case study allows the evaluation of the effects of \ac{EV} demand-side flexibility in an electricity market with a high share of \ac{RESE}. The used \textit{\ac{NT} 2030} scenario provided by the \ac{ENTSOE} is based on the fulfilment of these plans \cite{NT2030}. The used parameters can be found in Appendix \ref{sec:appendix_parameters}. 


The used \ac{CT}s  and subsequently the \ac{CD}s of all \ac{EV}s are based on the statistical evaluation of real-world charging events \cite{Flammini2019} in combination with statistical data.
The charging strategies used are divided as follows: 70\% \textit{immediately}, 20\% \textit{partly peak-shaving} and 10\% \textit{peak-shaving}.
\subsubsection{Scenarios}
All scenarios evaluated for 2030 are presented in Appendix \ref{sec:appendix_parameters}.
The optimisation is conducted as a rolling horizon optimisation with 365 24 hour \ac{DA} optimisation steps. The market premium ($P^{Premium}$ ) as the price for \ac{RESE} curtailment as a redispatch measure is set to 60 \euro{}/MWh.
The redispatch is optimised with a different number of \ac{EV}s to demonstrate the influence of aggregated flexibilities as a congestion management compensation system. 


The cost-minimal redispatch is performed with 200 000 (\textit{Scen. 200k}), 600 000 (\textit{Scen. 600k}), and 2 000 000 (\textit{Scen. 2M}) \ac{EV}s as demand flexibility devices. Overall, around 5.15 million passenger cars operate in Austria \cite{KFZAustria}. Thus, the electrification rate of Austria's private individual transport sector is 4\%, 12\%, and 39\% \cite{KFZAustria}. Each transport electrification level is optimised with three different flexibility setups: 
\begin{itemize}
 \setlength\itemsep{-0.5em}
    \item Firstly, the flexibility is optimised in the redispatch. Austrian law allows aggregated consumers to participate in the congestion management market due to the increasing share of \ac{RESE} and to encourage active customer participation\cite{Osterreich2021}. Hence, the \ac{EV}s participate in the congestion management market to reduce redispatch needs (\textit{Scen. Redispatch DSM}).
    \item Secondly, \ac{EV}s do not actively participate in the electricity market. Demand-side flexibility can neither be used in the dispatch nor the redispatch (\textit{Scen. no DSM}). Hence, the summation of the \ac{CD}s serves as additional inflexible demand. This scenario forms the "baseline" to analyse the effect of the other two scenarios. 
    \item Thirdly, flexibility is available in the dispatch. Therefore, their \ac{CD} will be adapted to the \ac{DA} price (\textit{Scen. Dispatch DSM}). 
\end{itemize}
\FloatBarrier
\subsubsection{Model setup}
To answer this research question, the dispatch and the redispatch model are used. A reduced set of aggregated nodes is used in Austria to determine the optimal dispatch from an economic perspective. This set contains nodes that are connected via critical transport lines. Transmission lines within the new nodes in Austria are aggregated; however, the absence of congestion of these lines is not guaranteed. Meanwhile, transfer capacities between the bidding zones are allocated within the optimisation. Figure \ref{fig:Nodes_dispatch_RQ3} shows the nodes and transmission grid used within the market-based dispatch. 

\begin{figure}[h]
\centering
   \begin{minipage}[b]{.45\linewidth} % [b] => Ausrichtung an \caption
      \includegraphics[width=1\textwidth]{graphics/RQ3/Nodes_dispatch.jpg}
      \caption{Nodes and grid of the dispatch model}
      \label{fig:Nodes_dispatch_RQ3}
   \end{minipage}
   \hspace{0.05\linewidth}% Abstand zwischen Bilder
   \begin{minipage}[b]{0.45\linewidth} % [b] => Ausrichtung an \caption
        \includegraphics[width=1\textwidth]{graphics/RQ3/Nodes_redispatch.jpg}
        \caption{Nodes and grid of the redispatch model}
        \label{fig:Nodes_redispatch_RQ3}
   \end{minipage}
\end{figure}
After the dispatch model has minimised the overall electricity generation costs with an aggregated set of nodes ($N$) and transmission lines (Figure \ref{fig:Nodes_dispatch_RQ3}), these nodes are disaggregated (Figure \ref{fig:Nodes_redispatch_RQ3}) ($N^{All}$) before performing the cost minimal redispatch in Austria. Thus, the differences between the market-based optimum and the actual physical power flows that may lead to congested transmission lines and must be compensated are analysed. The redispatch needs are balanced cost-minimal by adjusting the generation of thermal power plants, additional \ac{RESE} curtailment, and the use of \ac{EV} demand-side flexibility. Hydrogen sector coupling and chargeable demand-side flexibility are not part of these evaluations to evaluate the effects of demand-side flexibility without competition with other flexibility options. The power exchange in both optimisations is calculated using the \ac{PTDF} approach with 9 (dispatch) and 17 (redispatch) nodes in Austria and one node per each other country.